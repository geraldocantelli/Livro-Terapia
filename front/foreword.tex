%%%%%%%%%%%%%%%%%%%%%%%%%%%%%%%%%%%%%
% Read the /ReadMeFirst/ReadMeFirst.tex for an introduction. Check out the accompanying book "Better Books with LaTeX" for a discussion of the template and step-by-step instructions. The template was originally created by Clemens Lode, LODE Publishing (www.lode.de), mail@lode.de, 8/17/2018. Feel free to use this template for your book project!
%%%%%%%%%%%%%%%%%%%%%%%%%%%%%%%%%%%%%

% The Foreword is by the publisher, only general statements about the book and the theme, not the contents themselves.

\begin{chapterpage}{Publisher's Note}{p1_foreword:cha}

\begin{myquotation}
Cada pessoa é livre para meditar sobre a vida, seus processos e significados da forma que quiser. Claro que há consequências para cada pensamento que temos portanto é importante fazê-lo com responsabilidade. Jamais se pode ferir a consciência alheia com ideias que ela não está preparada para assimilar, portanto se você não se sentir bem ao ler este livro, pare de ler, até que esteja preparado para tal ato. Mas se se sentir bem, acredito que será de grande ajuda.\end{myquotation}

\end{chapterpage}

The publisher's note is about giving the reader the context of other books the company has published, how this book was produced, and contact points (email, website, etc.) for the reader to report issues or ask questions.

\hfil\psvectorian[height=10mm]{46}\hfil
