%%%%%%%%%%%%%%%%%%%%%%%%%%%%%%%%%%%%%
% Read the /ReadMeFirst/ReadMeFirst.tex for an introduction. Check out the accompanying book "Better Books with LaTeX" for a discussion of the template and step-by-step instructions. The template was originally created by Clemens Lode, LODE Publishing (www.lode.de), mail@lode.de, 8/17/2018. Feel free to use this template for your book project!
%%%%%%%%%%%%%%%%%%%%%%%%%%%%%%%%%%%%%


% Replace Replace with Second Chapter Name
% Replace c2_secondchapter:cha with your chapter title label (no spaces, only lower case letters)
% Replace the text below \end{chapterpage} and insert your own text.

\begin{chapterpage}{Amizade com a minha sombra ?}{c2_secondchapter:cha}

\begin{myquotation}E, indo no caminho, aconteceu que, chegando perto de Damasco, subitamente o cercou um resplendor de luz do céu.
E, caindo em terra, ouviu uma voz que lhe dizia: Saulo, Saulo, por que me persegues?
E ele disse: Quem és, Senhor? E disse o Senhor: Eu sou Jesus, a quem tu persegues. Duro é para ti recalcitrar contra os aguilhões.
E ele, tremendo e atônito, disse: Senhor, que queres que eu faça? E disse-lhe o Senhor: Levanta-te, e entra na cidade, e lá te será dito o que te convém fazer". \par\vspace*{15mm}
\mbox{}\hfill \emdash{}(Atos 9:3-6)\index{Atos 9:3-6}
% Add the source.
%, \citetitle{bibitem}\index{@\citetitle{bibitem}} \ifxetex\label{famousperson-bibitem-quote}\else\citep[p.~123]{bibitem}\fi
\par\end{myquotation}

\end{chapterpage}

% -------------------- replace or remove text below and paste your own text ------


\section{Uma outra sessão}\label{c1_images:sec}

\emdash{}Hoje eu comecei o dia orando uma prece de Evangelho pelos espíritos que não estão entre meus amigos e que podem estar agindo em meu desfavor, pedindo que ouvissem a lição e se iluminassem e se libertassem de suas amarras, chegando mais perto de Deus e ficando mais felizes. A psicologia não fala Jungiana não fala que temos que ter amizade com nossa "sombra"?

\emdash{}Não sou exatamente especialista em Jung mas o que você quer dizer? Quer fazer amizade como esses "espíritos"?
 
\emdash{} O que está em nossa sombra é o que guardamos de nossa personalidade desde pequenos pois nos é reprimido pelos mais velhos, pela sociedade como errado, feio, e passamos a negar esse aspecto de nossa personalidade. Mas quando ficamos adultos esse aspecto encontra uma forma de "surgir", de alguma forma como fora reprimido ele eclode na forma de desequilíbrio. Eu também não sou especialista em Jung, apenas uma pessoa tentando entender a vida.

\emdash{}Mas o que isso tem a ver com a sua crença nos espíritos?

\emdash{}Talvez sejam formas de ver a mesma coisa, penso serem conceitos que estão interligados e que explicam cada um um pouquinho, um lado da questão. Quando eu era pequeno tive experiências de lidar com espíritos diferentes desse jeito saudável de lidar que os kardecistas têm que é de conversar com os obsessores e convencê-los a liberar a vítima para o bem de ambos.

\emdash{}E como era então?

\emdash{}Naquelas ocasiões os espíritos literalmente levaram uma "surra" para deixar os obsedados em paz.

\emdash{}Como os espíritos apanhavam se eles não têm mais corpo (não estão mais vivos)?

\emdash{}Os espíritos menos esclarecidos sentem as dores do corpo que tinham em vida, alguns nem chegam a saber que morreram. Por falta de adiantamento espiritual, eles continuam obstinados em perseguições às pessoas que julgavam ser seus desafetos em vida, necessitando de esclarecimento e de luz para se libertarem deste círculo vicioso.

\emdash{}E funciona? Esses espíritos que apanham vão embora?

\emdash{}Pela minha experiência, não. Toda semana havia outros e eu me lembro de um que foi reincidente. Claro que as primeiras tentativas eram na base da conversa mas acredito que se deva ficar nas tentativas de convencimento e nunca partir para algo agressivo pois Jesus pregou \textit{o amor a Deus sobre todas as coisas e ao próximo como a si mesmo} e disse que esta é toda a Lei e os profetas.

\emdash{}Onde eram essa reuniões?

\emdash{}Eram em casa mesmo. Antigamente muitas pessoas faziam as reuniões mediúnicas em suas casas (ainda há quem faça) mas as orientações da doutrina espírita é que se façam no Centro Espírita pois lá é um ambiente preparado para lidar com os irmãozinhos que se manifestam e pois há equipes cuidando da energia do local e quando se faz em casa podem ficar energias negativas na casa.

\emdash{}Você culpa seus pais por isso?

\emdash{}De jeito nenhum. Com a terapia veio a \textit{ressignificação}\footnote{processo psicológico que levou ao perdão dos pais, fruto da terapia} e eu vi que só tinha a perder culpando eles e não fazia sentido também pois eles fizeram o melhor que podiam com o conhecimento que tinham e as possibilidades que tinham. Então está tudo bem.

\emdash{}Mas antes...

\emdash{}Não gosto de lembrar, mas antes eu os culpava pelos meus problemas psicológicos e até discutia com eles. Mas como estamos conversando sobre a importância de fazer amizade com a sombra, não se trata de "passar a mão na cabeça" dos erros mas de se perdoar para poder continuar o caminho da vida. Se eu aprendi a lição e corrigi meu erro, para que ficar me castigando? Tem até vários exemplos bíblicos sobre personagens escolhidos por Deus que erraram mas corrigiram seus caminhos, se perdoaram e continuaram seus caminhos.

\emdash{}Quem por exemplo?

\emdash{}Saulo, ou melhor Paulo. Era perseguidor de cristãos, matou Estevão e estava indo para Damasco perseguir Ananias quando se encontra com Jesus que se apresenta em toda sua majestade e interroga "Porque me persegues?" Então Paulo cai em si, vê o enorme erro que estava cometendo e diz "Senhos, que queres que eu faça" e segue para sua missão. Ele faz amizade com sua sombra pois não fica se punindo pelas perseguições que fez, não se faz de coitado pois isso teria paralisado suas ações e ele não teria feito mais nada.

\emdash{}Mas assume a responsabilidade.

\emdash{}Sim, tanto que ele sabia que tudo teria que ser reparado pois suportou prisões, açoites e até a morte, tudo pela divulgação do Evangelho de Jesus. 

\emdash{}Então você se perdoa.

\emdash{}Se eu não for meu amigo, com as dificuldades que a vida naturalmente já tem, fica mais difícil.