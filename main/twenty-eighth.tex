\begin{chapterpage}{Salvação}{c28_twenty-eighthchapter:cha}

\begin{myquotation}Eu sou a ressurreição e a vida; quem crê em mim, ainda que esteja morto, viverá; E todo aquele que vive, e crê em mim, nunca morrerá.

\par\vspace*{15mm}
\mbox{}\hfill \emdash{}Jesus, João 11:25-26 \index{João 11:25-26, Jesus}
, %\citetitle{bibitem}\index{@\citetitle{bibitem}} %\ifxetex\label{famousperson-bibitem-quote}\else\citep[p.~123]{bibitem}\fi
\par\end{myquotation}

\end{chapterpage}

% -------------------- replace or remove text below and paste your own text ------


\section{Iluminação}\label{c1_basicformatting:sec}

\emdash{}O ser humano não está preso a tempo e espaço; não morre com a morte do corpo e sim continua sua jornada rumo à comunhão completa e perfeita com o Criador. O que vemos com os olhos carnais e em geral sentimos com os sentidos do corpo não são a verdadeira realidade; tudo o que é real não pode perecer e não pode ser ameaçado; tudo o que é verdadeiramente seu não pode ser tirado de você portanto o que se vai com o tempo nunca foi seu de verdade e só é instrumento para que você se desenvolva e chegue-se até o Senhor.

\emdash{}Palavras como essas soam como verdadeiras em nossas mentes porque na realidade já sabemos de tudo isso no nosso interior. Iluminados como Jesus e Buda vieram lembrar a humanidade de que ela não terá um fim no túmulo e que foi criada para algo muito maior. Iluminação como é dito no Budismo ou Salvação como é dito no Cristianismo têm o mesmmo efeito prático que é libertar o ser humano da ilusão do mundo e da morte, das limitações que a mente gera desde a queda do pecado original (pedagogicamente ensinado e colocado).

\emdash{}Não há nada que limite o ser humano a não ser a sua própria mente quando não conhece a Verdade e por isso Ela é libertadora. Como o Caminho se faz caminhando cada um terá que fazer sua experiência de vida com a Verdade para descobri-la e libertar-se por isso se diz que o temor do Senhor é o princípio da Sabedoria porque é só então se busca as coisas do Alto, ou seja, da Alma.

\emdash{}É como se a humanidade estivesse dormindo na sua ilusão de separação de Deus ou de que a sua Espiritualidade não existe, presa no materialismo; e viessem esses iluminados e trouxessem uma mensagem, um código libertadores para promover um despertar. Muitas das vezes só nos importamos com essas mensagens quando somos feridos pela vida mas essa é nossa forma atual de evoluir: existem dois professores o Amor e a Dor e parece que por enquanto estagiamos mais aprendendo com a professora Dor e, depois que atingimos uma evolução maior, estamos prontos para as aulas com o professor Amor.

\emdash{}Mas por que a maioria de nós escolhe a professora Dor antes do professor Amor? Acredito que por causa da inércia em nossa vontade de evoluir: achamos que está bom ficarmos na zona de conforto e deixamos de fazer caridade, de perdoar, de procurar as pessoas para se reconciliar, de evitar discussões frívolas e então vem a Lei do Progresso e nos força a evoluir quer queiramos ou não e como resultado estaremos assistindo as aulas na classe da professora Dor (mas não era esse o plano original).

\emdash{}Não lamentemos o tempo perdido pois nenhum tempo é perdido se a lição foi aprendida e as lições do passado bem aproveitadas são um valioso instrumento para um presente mais feliz. Que tal seguirmos o Evangelho de Jesus e estagiarmos na senda do Amor? A felicidade verdadeira está em levar felicidade aos outros pois quem a cuida dos outros, Deus cuida da sua Vida. E também quando estamos cuidando das dores dos outros, não temos tempo para sentir tanto e lamentar as nossas e essas parecem-se bem menores.

\emdash{}Para evoluir e chegar-se ao coração de Deus vamos ajudando as pessoas e no realizar destas tarefas de caridade estaremos automaticamente trabalhando as nossas características que precisam ser melhoradas para chegarmos à comunhão com o Senhor. Jesus disse: ``Sede perfeitos como vosso Pai Celestial é perfeito", ele provavelmente falava de colocarmo-nos no caminho que nos levará a nos tornarmos o que Deus sonhou para a humanidade.