\begin{chapterpage}{Consolador Prometido}{c5_fifhchapter:cha}

\begin{myquotation} Se me amais, guardai os meus mandamentos. E eu rogarei ao Pai, e Ele vos dará outro consolador, para que fique eternamente convosco, o Espírito da Verdade, a quem o mundo não pode receber, porque não o vê, nem o conhece. Mas vós o conhecereis, porque ele ficará convosco e estará em vós. -- Mas o Consolador, que é o Espírito Santo, a quem o Pai enviará em meu nome, vos ensinará todas as coisas, e vos fará lembrar de tudo que vos tenho dito. 
\par\vspace*{15mm}
\mbox{}\hfill \emdash{}Jesus, João 14:15 a 17;26 \index{João 14:15 a 17;26, Jesus}
, %\citetitle{bibitem}\index{@\citetitle{bibitem}} %\ifxetex\label{famousperson-bibitem-quote}\else\citep[p.~123]{bibitem}\fi
\par\end{myquotation}

\end{chapterpage}

% -------------------- replace or remove text below and paste your own text ------


\section{Dentro de nós está este Consolador!}\label{c1_basicformatting:sec}

\emdash{}Se Jesus disse que não nos deixaria sós, então Ele não nos deixou assim. Estamos na compania de algo ou alguém que Deus colocou dentro de nós e portanto é preciso saber aquietar-se para ouvi-lo, esse Consolador.

\emdash{}Jesus costumava retirar-se para orar em silêncio e assim conectar-se com o Pai mais intensamente. Se ele precisava fazer isso, acredito que nós mais ainda. Mas o mundo de hoje tem muito barulho e muita informação, precisamos de um tempo para nós e Deus, ouvindo nosso interior.

\emdash{}Se nosso interior é conturbado, procuremos terapia saudável e os conselhos do Evangelho de Jesus, e sem deixar-nos vencer pelo desânimo encher a alma de esperança nas palavras do mestre.

\emdash{}Esse Consolador é aquele que nos explica as passagens bíblicas de maneira a transbordar de amor, sem excluir ninguém do Reino de Deus, sem separar irmãos em Cristo e aplicando tolerância e humanidade no melhor sentido da palavra.

\emdash{}Esse Consolador é cheio de bom senso e não de secularismo. Ele quebra paradigmas e estruturas paralisantes e renova o fôlego das pessoas pra quem Jesus disse: "Vinde a mim vós que estais cansados e eu vos aliviarei".

\emdash{}Esse Consolador traz equilíbrio e revela os mistérios de Deus pois "vos ensinará todas as coisas". Chega de não entender a vida por não termos quem nos explique, lembremos da promessa do mestre.


\emdash{}Ao mesmo tempo tenhamos a humildade do filósofo que disse: "Só sei que nada sei", pois somos humanos e falhos, ainda na caminhada de aprendizado da vida e não será de uma hora para outra, ou ainda, de uma vida para outra, que vamos adquirir a sabedoria dos anjos.

\emdash{}Paz e bem hão de morar no coração de quem deixar-se guiar pelo Consolador Prometido, hoje e sempre. Assim seja. A doutrina Kardecista coloca o verdadeiro Espiritismo como o Consolador Prometido que vem relembrar tudo o que Jesus disse e revelar todas as coisas, trazendo assim alívio aos nossos corações e uma esperança e forças renovadoras, mas é preciso ter discernimento pois com o uso que se faz da mediunidade pois a santidade de seu mandato está em razão do uso que se faz em favor do Evangelho do Cristo e da instalação do Reino de Deus em todos nós.