\begin{chapterpage}{Amor de Mãe e Pai}{c30_thiertiethchapter:cha}

\begin{myquotation}Amor de mãe é a mais elevada forma de altruísmo.

\par\vspace*{15mm}
\mbox{}\hfill \emdash{}Machado de Assis \index{Assis, Machado de}
, %\citetitle{bibitem}\index{@\citetitle{bibitem}} %\ifxetex\label{famousperson-bibitem-quote}\else\citep[p.~123]{bibitem}\fi
\par\end{myquotation}

\end{chapterpage}

% -------------------- replace or remove text below and paste your own text ------


\section{Mães e Pais}\label{c1_basicformatting:sec}

\emdash{}Pesquisando a frase para iniciar esse capítulo encontrei algumas como: ``Tudo aquilo que sou, ou pretendo ser, devo a um anjo, minha mãe." (Abraham Lincoln) e também: ``Minha mãe foi a mulher mais bela que jamais conheci. Todo o que sou, lho devo a minha mãe. Atribuo todos meus sucessos nesta vida ao ensino moral, intelectual e física que recebi dela." (George Washington). Citações de dois grandes líderes de uma conhecida nação que, se não resumem, trazem as características mais importantes do amor das progenitoras: elas parecem moldar os destinos.

\emdash{}Primeiramente gostaria de analisar a situação de o que fazer quando as mães e/ou pais não foram conosco tão amáveis ou não puderam nos propiciar uma formação mental das mais saudáveis. Longe de vitimismo e muito menos de se entregar, nem de culpá-los pelo pouco que puderam fazer por estes filhos, gostaria aqui de trazer esperança e desfazer algumas ideias negativas determinísticas que há sobre o tema.

\emdash{}Importante assinalar: ``não importa o que temos e sim o que fazemos com o que temos", ou seja, como trabalhamos com os elementos iniciais que a Vida nos deu e a partir daí desenvolvemos o presente. O auto-conhecimento ajuda muito e para chegar a tanto pode-se usar técnicas como psicoterapia com psicólogos e meditação.

\emdash{}Aqui cabe um alerta: nem todos os(as) psicólogos(as) trabalham com ética, competência, de maneira que nos faça bem portanto sempre é preciso discernimento e estar atento se o tratamento está tendo resultados salutares e, se necessário, trocar de profissional. Digo discernimento pois às vezes o profissional dirá coisas que não nos agradarão (como um remédio que pode ser amargo) e não é só por isso que vamos trocá-lo; contudo tudo tem limite.

\emdash{}A mãe de Chico Xavier, uma senhora muito amável, morreu quando ele ainda não tinha nem 5 anos e depois ele foi criado por uma madrinha que lhe impunha castigos físicos como ``garfadas" na barriga. Então sua mãe lhe apareceu e disse que ele não tivesse mágoa dela pois ela estava lhe preparando para o mundo e que ela intercederia para que a madrinha suavizasse o tratamento que a ele dispensava. E assim foi.

\emdash{}Mais tarde seu pai uniu-se a outra mulher que tratou a ele e aos demais filhos muito bem. Lições de estoicismo que ficam, pois mais tarde na fase adulta, quando tinha problemas sérios, Chico dizia: ``que saudades dos garfos da madrinha". Não se trata se masoquismo mas sim de resignação pois há coisas na Vida que podemos mudar porém há outras que não o podemos e só nos resta aceitá-las, como saída.

\emdash{}Vale lembrar o valor do Poder do Agora, ou seja, não viver no passado. Não importa como foi sua criação se você se decidir a viver no presente e não no passado. Também não viva no futuro, fazendo projeções do que pode acontecer e sofrendo por antecipação por coisas que podem nem acontecer. Tudo o que temos é o Agora, por isso se chama Presente. Se possível fazer uma psicoterapia para entender e ressignificar sua história é interessante e jamais se prender ao passado; ficar no Agora é uma atitude libertadora.

\emdash{}Um importante psicólogo disse uma vez que não há mães perfeitas e que as mães tem que ser apenas ``suficientemente boas" pois se forem boas demais estragam seus filhos. Entenda-se boas demais fazendo todas suas vontades e sendo excessivamente permissivas. Uma mãe suficientemente boa saberá dar amor na medida certa, atenção na medida certa e carinho também e muitas das nossas mães e pais são assim, digo pais pois o que se diz das mães nesse capítulo vale para os pais também, pois ambos assumem o papel de progenitores.

\emdash{}Nossas experiências nesse mundo são antes de tudo de aprendizado para nos aproximarmos de Deus para um dia estarmos em plena e eterna comunhão com Ele. Esse fato explica em grande parte a fala de Paulo: ``Em tudo dai graças, porque esta é a vontade de Deus em Cristo Jesus para convosco" pois todos os seres vivos estão conectados em suas estórias de vida, ou seja, de vidas, como se cada encarnação fosse um livro e cada um tivesse uma biblioteca particular e Deus é o maestro que governa com justiça todas essas Vidas.

\emdash{}Por esse motivo às vezes vimos como filhos, às vezes como mães, às vezes como pais, irmãos, maridos, esposas, primos. Precisamos nos acertar, nos entender, nos amar pois a Bíblia é clara ao dizer que enquanto tivermos rusgas contra nossos irmãos não entraremos no Reino dos Céus; teremos que nos acertar com todo mundo, leve quantas existências levar. E o que é então entrar no Reino dos Céus? É estar puro e evoluído e entrar em comunhão eterna com o Senhor Deus e não ter mais necessidade de reencarnar para se purificar.

\emdash{}Daí o papel sagrado não só das mães e dos pais  que participam da co-criação de Deus e possibilitam nossa entrada nesse mundo para o santo ato da Vida.