\begin{chapterpage}{Laços Familiares}{c41_fortieth-firstchapter:cha}
 
\begin{myquotation}Qual seria para a sociedade o resultado do relaxamento dos laços de família? Resposta: Um recrudescimento do egoísmo.
\par\vspace*{15mm}
\mbox{}\hfill \emdash{}Questão 775 de ``O Livro dos Espíritos”\index{Kardec, Allan}
, %\citetitle{bibitem}\index{@\citetitle{bibitem}} %\ifxetex\label{famousperson-bibitem-quote}\else\citep[p.~123]{bibitem}\fi
\par\end{myquotation}

\end{chapterpage}

% -------------------- replace or remove text below and paste your own text ------


\section{A importância da família}\label{c1_basicformatting:sec}

\emdash{}``A Providência Divina, detentora da máxima sabedoria, organizou a família na Terra para que cada componente dela disponha de todos os recursos possíveis, tendo como meta a sua evolução e prosperidade espiritual.

\emdash{}No seu âmago, essa célula básica da sociedade reúne todas as condições para que o Espírito reencarnante desenvolva sua potencialidade rumo à perfeição a que está destinado.

\emdash{}Descuidar da família, relaxar em seus propósitos e objetivos será abandonar o educandário, menosprezando a grande chance que se tem para sair da vida física, rumo à espiritual, em condições bem superiores às que tínhamos quando aqui chegamos.

\emdash{}Além da afetividade, onde se reúnem os corações afins, a família tem a função de promover o aprendizado ideal para a correta convivência social, pois que ensina como vencer o orgulho e o egoísmo.

\emdash{}Pensando e agindo uns pelos outros, na defesa do interesse familiar, aprendemos a renunciar, a proteger a quem amamos e também a nos proteger, a defender os anseios alheios, a calar diante de uma exaltação, a trabalhar objetivando amparar os que estão sob nossos cuidados, a ser solidários, fraternos..., enfim, aprendemos a viver em família, fator que nos assegura, também, como viver em sociedade.

\emdash{}Longe da família, na solidão, no isolamento, vivendo descompromissadamente, tudo fica mais difícil, onde corremos maiores riscos de quedas espetaculares pelas estradas escorregadias do mundo.

\emdash{}Pais, mães, filhos, parentela variada se juntam na Terra, via de regra, depois de intensas e meticulosas programações efetuadas no mundo espiritual, onde manifestaram o desejo de usufruir de novas oportunidades de progresso espiritual, ensejando encontrar a paz que almejam e a felicidade que anseiam.

\emdash{}Diante dessa assertiva, no contexto familiar, os mais responsáveis, amadurecidos e conscientes, deverão se esforçar ao máximo para ajudar aqueles que seguem mais à retaguarda.

\emdash{}No tocante aos filhos, devem nascer as maiores preocupações nos genitores, pois que a Providência Divina, confiante, entrega aos pais aqueles que chegam ao mundo para novas e redentoras oportunidades de progresso.

\emdash{}Requererão, sim, bons médicos, hospitais, escolas, roupas, brinquedos, remédios, mas buscarão, acima de tudo, mãos amigas e determinadas que possam apontar-lhes um norte seguro e confiável, pelas estradas firmes da educação, pois que estão chegando de um passado não muito feliz e anseiam por um futuro promissor.

\emdash{}E educar é bem diferente de instruir. Instruir é dar conhecimentos, informações, tarefa essa compartilhada com a escola. Educar é formar caráter, edificar homens de bem, e essa missão é exclusivamente da família. Relegá-la a segundo plano ou transferi-la para outros setores da vida será, sem dúvida, escancarar um abismo diante dos filhos, com enorme possibilidade de queda dentro dele, muito provavelmente sem se encontrar o caminho de volta tão cedo.

\emdash{}Os pais são os primeiros e maiores exemplos para os filhos. A indiferença, o descaso e a irresponsabilidade diante dos reais e dignos valores da vida têm sido lições infelizes e inadequadas que os genitores vêm passando aos rebentos. Com isso crescem assustadoramente, no contexto infantojuvenil, o consumo de tóxicos pesados, o alcoolismo, o tabagismo, o uso prematuro e desequilibrado do sexo e outros comportamentos nocivos que põem em risco a vida desses “pequenos”, que reencarnaram pensando em receber socorro para vencer as más tendências que ainda abrigam.

\emdash{}O relaxamento dos laços de família poderá jogar por terra toda uma laboriosa programação espiritual, adredemente preparada, com muito esforço e dedicação, por Espíritos superiores, objetivando nos fazer melhores.

\emdash{}Somos, sim, livres, tendo autonomia para fazer ou deixar de fazer o que nos propusemos, mas, da mesma forma e na mesma proporção, receberemos o reflexo de ter feito ou não o que devíamos.

\emdash{}Recompensas felizes ou dores profundas nos esperarão nos dias do futuro. Pensemos nisso".

\emdash{}Esse artigo foi extraído da Revista Semanal de Divulgação Espírita O Consolador Ano 8 - N° 389 - 16 de Novembro de 2014 e seu autor é Waldenir Aparecido Cuin.