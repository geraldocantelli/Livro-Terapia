%%%%%%%%%%%%%%%%%%%%%%%%%%%%%%%%%%%%%
% Read the /ReadMeFirst/ReadMeFirst.tex for an introduction. Check out the accompanying book "Better Books with LaTeX" for a discussion of the template and step-by-step instructions. The template was originally created by Clemens Lode, LODE Publishing (www.lode.de), mail@lode.de, 8/17/2018. Feel free to use this template for your book project!
%%%%%%%%%%%%%%%%%%%%%%%%%%%%%%%%%%%%%


% Replace Replace with Third Chapter Name
% Replace c3_thirdchapter:cha with your chapter title label (no spaces, only lower case letters)
% Replace the text below \end{chapterpage} and insert your own text.

\begin{chapterpage}{A Sabedoria é cheia de dúvidas}{c3_thirdchapter:cha}

\begin{myquotation}Só sei que nada sei.\par\vspace*{15mm}
\mbox{}\hfill \emdash{}Sócrates\index{Sócrates}
% Add the source.
%, \citetitle{bibitem}\index{@\citetitle{bibitem}} \ifxetex\label{famousperson-bibitem-quote}\else\citep[p.~123]{bibitem}\fi
\par\end{myquotation}

\end{chapterpage}

% -------------------- replace or remove text below and paste your own text ------

\section{A Sabedoria tem dúvidas...}\label{c1_images:sec}

\emdash{}Quanto mais se conhece das coisas, surgem mais dúvidas do que certezas, já percebeu? E isso precisa ser consolador ao invés de desesperador, você não acha?

\emdash{}Sempre ouvi falar que conhecimento gera conhecimento e não ignorância, o que você quer dizer com gerar mais dúvidas?

\emdash{}Quanto mais se sabe, mais se percebe o quanto falta conhecer e surgem novas questões que antes não existiam. Ouvi uma frase numa palestra que dizia: "A Sabedoria tem dúvidas, já a ignorância tem certezas absolutas", o que é comprovado históricamente não precisa nem comentar mas o ponto é que se a sabedoria traz mais dúvidas do que certezas como podemos nos confortar com ela?

\emdash{}Nossa, filosófico isso...kkkkkk(lol). O que você acha?

\emdash{}Acho que o mundo é um barco (num mar, que é o universo) e que Jesus está no leme e Ele sim sabe o que está fazendo. Ele tem a sabedoria sem as dúvidas. E como Ele disse "nem uma folha cai de uma árvore sem a permissão de Deus-pai" então estamos todos bem cuidados e amparados mesmo sem ter domínio sobre a vida (que aliás nunca tivemos).

\emdash{}Essas reflexões te ajudam na sua caminhada?

\emdash{}Acredito que sim, tudo conta.

\emdash{}Mas se acredita que estamos todos bem amparados, quanto te acontece algo desagradável porque acha que acontece?

\emdash{}Eu acredito nas leis de ação e reação que Deus criou para não precisar ficar se imiscuindo em cada acontecimento mínimo na vida de cada ser do universo. O que tenho feito reflete  uma reação que volta para mim em um determinado momento e agora estou colhendo o que plantei há tempos atrás e ao mesmo tempo fazendo semeadura para colher em tempos futuros.

\emdash{}Esse é o conceito de justiça na sua crença?

\emdash{}Mais ou menos. Está na Bíblia que uma boa ação apaga uma multidão de pecados e o profeta Isaías disse que "as misericórdias do Senhor são as causas de não sermos consumidos". Eu não sou doutor nessas coisas mas nada é rígido ao pé da letra na misericórdia Divina mas justiça é feita sem dúvida.

\emdash{}E isso te conforta?

\emdash{}Sim, consola também.