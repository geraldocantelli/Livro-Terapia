\begin{chapterpage}{Não creiais em todos os Espíritos.}{c47_fortieth-seventhchapter:cha}
 
\begin{myquotation}Meus bem-amados, não creais em qualquer Espírito; experimentai se os Espíritos são de Deus, porquanto muitos falsos profetas se têm levantado no mundo.
\par\vspace*{15mm}
\mbox{}\hfill \emdash{}João, 1ª Epistola, 4:1\index{1ª Epistola João, 4:1}
, %\citetitle{bibitem}\index{@\citetitle{bibitem}} %\ifxetex\label{famousperson-bibitem-quote}\else\citep[p.~123]{bibitem}\fi
\par\end{myquotation}

\end{chapterpage}

% -------------------- replace or remove text below and paste your own text ------


\section{Prodígios dos falsos profetas}\label{c1_basicformatting:sec}

\emdash{}``Levantar-se-ão falsos cristos e falsos profetas, que farão grandes prodígios e coisas de espantar, a ponto de seduzirem os próprios escolhidos." Estas palavras dão o verdadeiro sentido do termo prodígio. Na acepção teológica, os prodígios e os milagres são fenômenos excepcionais, fora das Leis da Natureza. Sendo estas, exclusivamente, obra de Deus, pode
Ele, sem dúvida, derrogá-las, se lhe apraz; o simples bom senso, porém, diz que não é possível haja Ele dado a seres inferiores e perversos um poder igual ao seu, nem, ainda menos, o direito de desfazer o que Ele tenha feito. Semelhante princípio não no pode Jesus ter consagrado. Se, portanto, de acordo com o sentido que se atribui a essas palavras, o Espírito
do mal tem o poder de fazer prodígios tais que os próprios escolhidos se deixem enganar, o resultado seria que, podendo fazer o que Deus faz, os prodígios e os milagres não são privilégio exclusivo dos enviados de Deus e nada provam, pois que nada distingue os milagres dos santos dos milagres do demônio. Necessário, então, se torna procurar um sentido mais racional para aquelas palavras.

\emdash{}Para o vulgo ignorante, todo fenômeno cuja causa é desconhecida passa por sobrenatural, maravilhoso e miraculoso; uma vez encontrada a causa, reconhece-se que o fenômeno, por muito extraordinário que pareça, mais não é do que aplicação de uma Lei da Natureza. Assim, o círculo dos fatos sobrenaturais se restringe à medida que o da Ciência se alarga. Em todos os tempos, homens houve que exploraram, em proveito de suas ambições, de seus interesses e do seu anseio de dominação, certos conhecimentos que possuíam, a fim de alcançarem o prestígio de um pseudopoder sobre-humano, ou de uma pretendida missão divina. São esses os falsos cristos e falsos profetas. A difusão das luzes lhes aniquila o crédito, donde resulta que o número deles diminui à proporção que os homens se esclarecem. O fato de operar o que certas pessoas consideram prodígios não constitui, pois, sinal de uma missão divina, visto que pode resultar de conhecimento cuja aquisição está ao alcance de qualquer um, ou de faculdades orgânicas especiais, que o mais indigno não se acha inibido de possuir, tanto quanto o mais digno. O verdadeiro profeta se reconhece por mais sérios caracteres e exclusivamente morais.

\section{Não Creiais em Todos os Espíritos}

\emdash{}Os fenômenos espíritas, longe de abonarem os falsos cristos e os
falsos profetas, como a algumas pessoas apraz dizer, golpe mortal desferem
neles. Não peçais ao Espiritismo prodígios, nem milagres, porquanto ele
formalmente declara que os não opera. Do mesmo modo que a Física, a Química, a Astronomia, a Geologia revelaram as leis do mundo material,
o Espiritismo revela outras leis desconhecidas, as que regem as relações do
mundo corpóreo com o mundo espiritual, leis que, tanto quanto aquelas outras da Ciência, são Leis da Natureza. Facultando a explicação de
certa ordem de fenômenos incompreendidos até o presente, ele destrói o
que ainda restava do domínio do maravilhoso. Quem, portanto, se sentis-
se tentado a lhe explorar em proveito próprio os fenômenos, fazendo-se
passar por messias de Deus, não conseguiria abusar por muito tempo da
credulidade alheia e seria logo desmascarado. Aliás, como já se tem dito,
tais fenômenos, por si sós, nada provam: a missão se prova por efeitos
morais, o que não é dado a qualquer um produzir. Esse um dos resultados
do desenvolvimento da ciência espírita; pesquisando a causa de certos fenômenos, de sobre muitos mistérios levanta ela o véu. Só os que preferem
a obscuridade à luz, têm interesse em combatê-la; mas a verdade é como o
Sol: dissipa os mais densos nevoeiros.

\emdash{}O Espiritismo revela outra categoria bem mais perigosa de falsos
cristos e de falsos profetas, que se encontram, não entre os homens,
mas entre os desencarnados: a dos Espíritos enganadores, hipócritas,
orgulhosos e pseudossábios, que passaram da Terra para a erraticidade
e tomam nomes venerados para, sob a máscara de que se cobrem, facilitarem a aceitação das mais singulares e absurdas ideias. Antes que se
conhecessem as relações mediúnicas, eles atuavam de maneira menos
ostensiva, pela inspiração, pela mediunidade inconsciente, audiente ou
falante. É considerável o número dos que, em diversas épocas, mas,
sobretudo, nestes últimos tempos, se hão apresentado como alguns dos
antigos profetas, como o Cristo, como Maria, sua mãe, e até como Deus.
João adverte contra eles os homens, dizendo: ``Meus bem-amados, não
acrediteis em todo Espírito; mas experimentai se os Espíritos são de
Deus, porquanto muitos falsos profetas se têm levantado no mundo."
O Espiritismo nos faculta os meios de experimentá-los, apontando os
caracteres pelos quais se reconhecem os bons Espíritos, caracteres sempre
morais, nunca materiais. É à maneira de se distinguirem dos maus os
bons Espíritos que, principalmente, podem aplicar-se estas palavras de
Jesus: ``Pelo fruto é que se reconhece a qualidade da árvore; uma árvore boa não pode produzir maus frutos, e uma árvore má não os pode produzir bons." Julgam-se os Espíritos pela qualidade de suas obras, como
uma árvore pela qualidade dos seus frutos.

O conteúdo desse capítulo foi copiado da obra ``O Evangelho Segundo o Espiritismo", de Allan Kardec, capítulo XXI.