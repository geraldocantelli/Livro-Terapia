\begin{chapterpage}{Ação e Reação}{c13_thirteenthchapter:cha}

\begin{myquotation} Mas Jesus lhe ordenou: ``Embainha a tua espada; pois todos os que lançam mão da espada pela espada morrerão!"
 

\par\vspace*{15mm}
\mbox{}\hfill \emdash{}Jesus, Mateus 26:52 \index{Jesus, Mateus 26:52}
, %\citetitle{bibitem}\index{@\citetitle{bibitem}} %\ifxetex\label{famousperson-bibitem-quote}\else\citep[p.~123]{bibitem}\fi
\par\end{myquotation}

\end{chapterpage}

% -------------------- replace or remove text below and paste your own text ------


\section{Viver em paz o máximo possível}\label{c1_basicformatting:sec}

\emdash{}Neste mundo cada cabeça representa um pensamento de forma que é impossível viver nossa própria identidade sem que ocorram conflitos de interesse de alguma ordem. Contudo não devemos atacar a realidade alheia apenas para que se sobressaia a nossa pois a cada ação corresponde uma reação na mesma direção, em sentido oposto (como na física). 

\emdash{}Muitos dos males que nos afligem são causados por nós mesmos, desde doenças psicossomáticas a consequências funestas de atos impensados. Quando agimos mal para com o próximo estamos plantando uma erva daninha no nosso próprio quintal. Tudo o que damos volta para nós, inclusive palavras mesmo que as pessoas de quem falamos não as ouçam, pois todos estamos ligados, interconectados numa malha universal.

\emdash{}Aqui o conselho de Santo Agostinho é muito útil: fazer uma revisão mental toda noite sobre seu comportamento e ver se não há algo de censurável e que pode ser melhorado. Isso é até uma receita para o auto-conhecimento e para a felicidade neste mundo. Pensemos em todo o mal que podemos evitar que nos aconteça apenas corrigindo nosso proceder, isso se chama prudência.

\emdash{}O auto-conhecimento pode levar a alegrias maiores como uma vida em paz em família e ajudar ao próximo. Profissionalmente fazer algo de que se gosta é uma alegria e isso pode ser conseguido dando-se sentido ao seu trabalho alinhando-o com seus propósitos de vida. E como uma pessoa conheceria seus próprios propósitos sem o auto-conhecimento?

\emdash{}Todo estudo é trabalhoso, e neste caso requer passar um tempo consigo mesmo, fazer-se perguntas e ver quais respostas surgem, lapidar arestas como um diamante bruto que no início é apenas uma pedra amorfa e depois se torna uma jóia preciosa. Regar o terreno, adubar, plantar boa semente, não deixar bater muito sol, cuidar das plantações boas que nascerem com muito amor. Digo tudo isso em linguagem figurada mas cada um saberá fazer isso na prática na medida em que estiver praticando pois o caminho se faz caminhando e se necessário, procure a ajuda de um profissional de confiança.

\emdash{}Até a mais longa caminhada começa com um único passo e quando menos esperarmos estaremos colhendo os frutos desse plantio cuidadoso e amoroso.

\emdash{}Paz e bem. Deus abençoe a todos.