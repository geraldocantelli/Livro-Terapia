%%%%%%%%%%%%%%%%%%%%%%%%%%%%%%%%%%%%%
% Read the /ReadMeFirst/ReadMeFirst.tex for an introduction. Check out the accompanying book "Better Books with LaTeX" for a discussion of the template and step-by-step instructions. The template was originally created by Clemens Lode, LODE Publishing (www.lode.de), mail@lode.de, 8/17/2018. Feel free to use this template for your book project!
%%%%%%%%%%%%%%%%%%%%%%%%%%%%%%%%%%%%%


% Replace Replace with First Chapter Name
% Replace c1_firstchapter:cha with your chapter title label (no spaces, only lower case letters)
% Replace the text below \end{chapterpage} and insert your own text.

\begin{chapterpage}{Depois de várias sessões}{c1_firstchapter:cha}

\begin{myquotation} Quem olha para fora sonha, quem olha para dentro desperta.
\par\vspace*{15mm}
\mbox{}\hfill \emdash{}Carl Gustav Jung \index{Jung, Carl Gustav}
, %\citetitle{bibitem}\index{@\citetitle{bibitem}} %\ifxetex\label{famousperson-bibitem-quote}\else\citep[p.~123]{bibitem}\fi
\par\end{myquotation}

\end{chapterpage}

% -------------------- replace or remove text below and paste your own text ------


\section{Diálogo}\label{c1_basicformatting:sec}

\emdash{}Eu fico tentando entender as coisas como são e às vezes obtenho êxito, pelo menos acho, e fico feliz com isso!! Queria contar da última que eu acho das mais importantes.

\emdash{}Primeiro me diga, como é esse processo de você ficar tentando entender as coisas como são, você gasta muito tempo com isso? que importância você dá para essas tentativas, chega a ficar preocupado com isso?

\emdash{}Acho que nos meus anos de terapia os questionamentos da vida se tornaram uma coisa natural pra mim mas não é o tempo todo, sei da importância de não desconectar dos afazeres do dia a dia para ter esses momentos de reflexão. E sobre se preocupar, isso era no começo do tratamento agora tudo flui com mais tranquilidade, graças a Deus, literalmente.

\emdash{}Como assim?

\emdash{}Acredito que nossos esforços para melhorar são uma das formas que Deus usa para nos ajudar, além da ajuda que recebemos das outras pessoas. No meio do processo vamos nos curando graças à ajuda Dele que vai colocando ideias luminosas de o que fazer para sairmos das situações difíceis na medida que vamos caminhando mas é um passo de cada vez, não vem tudo pronto de cara. Como uma experiência de fé: damos um passo, vem uma solução, então damos outro passo e vem outra solução, sempre na medida da situação. O caminho se faz caminhando é o que dizem.

\emdash{}E foi isso que você entendeu e queria contar?

\emdash{}Não, foi outra coisa: eu estava assistindo uma palestra que dizia que a Verdade não é algo que se lê ou que se ouve mas sim é uma pessoa, Jesus. Provavelmente todos já ouvimos falar na máxima \textit{"Conhecereis a Verdade e a Verdade vos libertará"} e acho que ficamos esperando que nos digam essa verdade em alguma igreja ou que a leiamos na Bíblia em algum trecho do Evangelho, mas na realidade Ela é uma Pessoa: Jesus, que também disse que Ele é o Caminho, a Verdade e a Vida. É muito pessoal a experiência espiritual de cada um mas entendi que quando Ele diz que é a Vida não quer dizer que vivemos Nele? ou seja na Vida através Dele? quando diz vinde a mim todos vós que estais fadigados e eu vos aliviarei e que Seu fardo é leve e suave não quer dizer que Ele nos ajudar a carregar o nosso fardo pois vivemos através Dele?

\emdash{}Mas pensar essas coisas não te deixa alienado e se sentindo esquisito? Como você está se sentindo?

\emdash{}Muito bem, melhor do que o melhor que já estive. E pensar assim acredito que me devolve pra mim mesmo ao invés de me alienar, me dá a oportunidade de fazer as coisas com intenção e não no automático, com mais responsabilidade e sabor (sentido).

\emdash{}É importante que uma pessoa tenha uma crença, seja ela qual for, mas não vá fixar demais nessas ideias pois tudo em excesso não faz bem.

\emdash{}Eu sei que o equilíbrio não deve ser perdido de vista mas como teria alcançado equilíbrio sem ser "dono" de mim mesmo? Sem terapia, sem espiritualidade, sem reconciliação com o próximo e sem auto-conhecimento. É uma caminhada longa pois as variáveis são muitas mas estar fazendo um bocadinho de cada vez pra não afobar me dão a noção de não estar metendo os pés pelas mãos\footnote{figura de linguagem}. 


