\begin{chapterpage}{No amor não existe medo}{c66_sixtieth-sixthchapter:cha}
 
\begin{myquotation} No amor não existe medo; antes, o amor perfeito lança fora o medo. Ora, o medo produz tormento; logo, aquele que teme não é aperfeiçoado no amor. 
\par\vspace*{15mm}
\mbox{}\hfill \emdash{}1 João 4,18\index{1 João 4,18}
, %\citetitle{bibitem}\index{@\citetitle{bibitem}} %\ifxetex\label{famousperson-bibitem-quote}\else\citep[p.~123]{bibitem}\fi
\par\end{myquotation}

\end{chapterpage}

% -------------------- replace or remove text below and paste your own text ------


\section{Primeira Carta de São João}\label{c1_basicformatting:sec}

\emdash{}Às vezes podemos nos pega preocupados, num sentido negativo, com as coisas do Céu e sobre Deus mas isso não é bom e não faz bem. A boa preocupação seria no sentido de desejar as coisas do Alto e se esforçar por buscá-las mas em nenhum momento isso pode ser penoso ou pesaroso.

\emdash{}Deus é Vida, é o Senhor da Vida. As coisas santas precisam nos elevar, fazer bem. O Amor cura, liberta, a Palavra de Deus liberta nossas consciências e nossas almas e não o contrário. Muitas pessoas não se dedicam a pensar em Deus pois o associam os assuntos do céu com outros como pecado e punição mas se dedicamos nossa alma inteiramente a Deus, Ele nos guiará por um Caminho que nada tem de pecado e perdição.

\emdash{}O Senhor nos guia se a Ele nos entregamos. Nós mesmos com nossas mentes limitadas não podemos pensar em tudo o que é necessário para sermos bons filhos de Deus mas Ele mesmo nos completa e guia e esse é o sentido de termos uma vida espiritual: vivermos Nele. A mente humana pode gerar mais problemas do que solução e frequentemente leva a pessoa a sofrimento mas quando vivemos em Deus acontece a paz que excede os problemas gerados pela mente: somos salvos.

\emdash{}Já parou para pensar no conceito de Salvação? A palavra salvação quer dizer que uma pessoa que está presa ou em situação difícil se livrou dessa situação: Jesus nos salva de nós mesmos, de nossas próprias mentes. O conceito de pecado original foi a identificação com a mente e passamos a pensar que somos ela, que somos nosso passado e futuro e as coisas que aconteceram ou acontecerão com a gente, mas não somos nada disso: somos uma Unidade com o Senhor e a queda da humanidade foi parar de reconhecer essa Unidade e passar a reconhecer apenas a si mesma (suas mentes) como realidade (então a morte entrou no mundo).

\emdash{}Jesus é a Verdade, o Caminho e a Vida pois Ele nos religa à essa Unidade com o Criador. Ele nos salvou através do seu sacrifício e da Verdade que pregou: ''conheceis a Verdade e a Verdade vos libertará". Não deixe que seus pensamentos religiosos seja pesarosos ou conflitantes pois nós somos muito mais do que nossos pensamentos e pela nossa mente ser limitada corremos o risco de nos identificar com esses pensamentos e pensarmos até que Deus é limitado também a ter defeitos humanos como ira e vingança.

\emdash{}A palavra Deus foi muito mal empregada quando a associou a uma pessoa com defeitos humanos. Ele é muito mais do que isso, é transcendente. Formamos com Ele uma Unidade de Amor: acordar para essa realidade é uma libertação, uma salvação. Dessa Unidade brota paz, saúde, amor, perdão, caridade, compaixão, louvor, tudo de bom. Jesus disse que a boa árvore não dá mau fruto e que a má árvore não dá bom fruto. Deus é a boa árvore e dele só vem bons frutos; não pode vir nada de Deus que tenha a ver com nossos defeitos pois Ele não os tem.

\emdash{}Pode-se perguntar: e a Justiça? Onde está? Os judeus acreditavam que nos arredores de Jerusalém havia uma região de um fogo eterno onde quem morresse e fosse mau expiaria suas imperfeições por um tempo e depois seguiria na vida eterna. Quando foi traduzido para o grego e o latim esse fogo eterno se transformou em pena individual eterna então criou-se o conceito de inferno, que não existia para os hebreus. Mas é o fogo que é eterno e não as penas. Portanto a cada um será dado conforme suas obras e quem fizer o mal vai pagar caro por isso mas não eternamente pois Deus é justo e não mau.

\emdash{}Nós devemos ser responsáveis por nossos atos na medida que tudo o que fizermos para os outros, nos será feito igualmente: o que damos para o Universo volta para nós. E nos motivar a sermos bons pelo reconhecimento da nossa Unidade com o Pai que dá Alegria e Completude: fazer o bem pelo Bem, amar pelo Amor. Até porque medo do inferno não impede as pessoas de pecar quando elas realmente querem fazer alguma coisa errada. Pelo contrário, esse conceito de inferno apenas leva algumas pessoas a duvidar da existência de Deus pois não poderia haver um Deus pior do que nós mesmos; e essa incredulidade leva a um relativismo perigoso para a conduta.

\emdash{}Jesus disse que aquele que peca é escravo do pecado. Mas se nos libertarmos pela exata noção que estamos unidos a Deus e somos Verdade junto com Ele, deixamos de mau proceder. E passamos a viver uma plenitude de paz e alegria, como Ele disse: ``Eu vim para trazer Vida e Vida em plenitude".