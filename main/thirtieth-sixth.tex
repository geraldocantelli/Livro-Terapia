\begin{chapterpage}{Compaixão e o Nosso Futuro}{c36_thirtieth-sixthchapter:cha}

\begin{myquotation}Felicidade e bem-estar são, na verdade, mais bem considerados como habilidades.
\par\vspace*{15mm}
\mbox{}\hfill \emdash{}Richard Davidson\index{Davidson, Richard}
, %\citetitle{bibitem}\index{@\citetitle{bibitem}} %\ifxetex\label{famousperson-bibitem-quote}\else\citep[p.~123]{bibitem}\fi
\par\end{myquotation}

\end{chapterpage}

% -------------------- replace or remove text below and paste your own text ------


\section{As Pesquisas Científicas do Dr. Richard Davidson}\label{c1_basicformatting:sec}

\emdash{}Estava este doutor citado no título em vias de concluir seu PhD. quando foi acometido de sérios problemas que não estavam se resolvendo por meio de terapias convencionais quando lhe sugeriram a meditação, que foi a solução. Empolgado com o fato, numa viagem ao Tibet, conversando com o \textit{Dalai Lama}, este lhe disse que era um fã de seu trabalho com pesquisas em neurociência e que a ciência ocidental pesquisava muito sobre ansiedade, doenças em geral da mente e mesmo depressão mas que ele nunca havia visto uma pesquisa sobre ternura, gentileza e bondade e se o senhor Richard poderia lhe informar se conhecia alguma pesquisa sobre essas realidades.

\emdash{}O doutor Richard, que é hoje um dos 100 cientistas mais influentes do mundo, respondeu que também não conhecia em sua área pesquisa sobre essas coisas mas que as procuraria para enviar-lhe assim que retornasse à América. Quando chegou, pediu a seu assistente que pesquisasse não só trabalhos concluídos mas também pré-projetos sobre ternura, gentileza e bondade na área de neurociência e depois de alguns dias o assistente nada encontrou. Não desistindo, ele pediu a um amigo encarregado de microfilmar na biblioteca do congresso americano todo o acervo já escrito portanto todo o acervo possível e fez a mesma solicitação: trabalhos sobre ternura, gentiliza e bondade. O amigo lhe disse que essas coisas ele garantia que não encontraria na área de neurociência pois se tratavam de fenômenos sociais e não de realidades e sugeriu que procurasse em sociologia.

\emdash{}Mas o doutor insistiu na área de neurociência e mais uma vez nada foi encontrado, então decidiu ele mesmo fazer a pesquisa. Definiu ternura como sendo ajudar pessoas a quem amamos e queremos bem, gentileza como ajudar pessoas a quem desconhecemos até então ou que nos são ``neutras" em termos de afetividade (modo de falar) e bondade como ajudar pessoas a quem temos antipatia e de quem não gostamos. Para que a pesquisa não tivesse influência nenhuma de cunho social escolheu crianças de 4 anos e colheria material genético delas antes e depois de expostas a certas situações assim como imagens de seus cérebros durante essas situações.

\emdash{}A pessoa que amam seriam por exemplo suas mães, as pessoas desconhecias, voluntários que nunca tiveram contato com as crianças e o antipatizante seria um(a) voluntário(a) que implicaria e provocaria inimizade com a criança de propósito para que ela não gostasse dele(a). Iniciou-se, por exemplo, o seguinte diálogo: ``Você gosta muito da mamãe, não gosta?" E claro, a resposta era positiva, e falava-se do que tinham feito de especial naqueles dias como passear, conversar, etc. Enquanto isso ia-se visualizando as áreas do cérebro da criança que ficavam ativas, até aí tudo normal. Então falou-se: ``A mamãe está com problemas, você gostaria de ajudar a mamãe?" Então a criança sentia empatia pela mãe e queria beijá-la, ajudá-la de alguma forma e ativava-se o lobo frontal do seu cérebro, que é uma parte que quase não é usada em adultos. Na verdade nenhum estudo anterior havia sido feito sobre o lobo frontal a não ser um sobre monges tibetanos e freiras carmelitas.

\emdash{}Nesse ponto estava-se avaliando a ternura. Então quanto mais se falava em ajudar a mamãe mais se ativava o lobo frontal do cérebro da criança e propôs-se ir até ela para a ajuda e no caminho foram colocadas pessoas desconhecidas voluntárias em situação de necessidade de ajuda e o entrevistador propôs à criança se queria ou não (deixou livre) ajudar essas pessoas e quando ela ajudava o lobo frontal ficava mais ativo ainda. E depois disso cada vez ajudava com maior frequência. Nesse ponto estava-se avaliando a gentileza. Em algumas crianças a área do lobo frontal tornou-se ``acesa" ou ativa como que permanentemente e, nessas, quando apareceu a figura do antipatizante em situação de necessidade, elas também o ajudaram; mas as crianças onde o lobo frontal não se tornou ativo permanentemente não ajudaram o antipatizante e até gostaram de vê-lo em dificuldades. Perguntadas por que o ajudaram as primeiras disseram: ``porque ele precisava" mas ao mesmo tempo perguntadas se gostavam dele responderam ``não gosto dele" portanto tiveram uma reação altruísta porque isto está dentro do ser humano, só precisa ser incentivado, ensinado, potencializado mas é natural em nós.

\emdash{}Quando os pais dizem para os filhos coisas como ``não seja boboca" para na verdade dizer que não sejam bondosos ou que não partilhem seus brinquedos pois se os primos ou amigos quebrarem os brinquedos, eles apanharão em casa, esses pais estão ensinando esses seres humanos a serem egoístas, frios, ou seja o contrário da ternura, gentileza e bondade que é a natureza e o Evangelho que Jesus trouxe à Terra. E depois esses mesmos pais se surpreendem quando esses filhos não os vão visitar na velhice e os abandonam mas foram eles mesmo que os ensinaram a ser frios e calculistas. O amor também se ensina, a felicidade e o bem-estar são habilidades que se desenvolvem. A grande pista é seguir os conselhos evangélicos, não é piegas, e não torna ninguém fraco, pelo contrário, fortalece em todos os sentidos.

\emdash{}Como disse, também amostras genéticas foram colhidas das crianças e a conclusão a que se chegou é que desenvolvendo-se a ternura, a gentileza e a bondade as pessoas tornam-se mais resistentes a doenças, com melhor imunidade, a taxa de erros no metabolismo é menor, ou seja são pessoas mais fortes e saudáveis. A fonte das informações deste capítulo estão numa palestra de Anete Guimarães proferida na Mansão do Caminho que está disponível na plataforma www.espiritismoplay.com que ajuda a manter a obra de caridade do médium Divaldo Franco.

\emdash{}A compaixão é a chave que possibilitou a ativação do lobo frontal nessas pessoas e pode possibilitar em todos nós. Como visto na obra de André Luiz psicografada por Chico Xavier, o lobo frontal está relacionado com nosso futuro e com a Compaixão, ele anula nossas programações do passado e nos faz atuar livremente no presente como Deus quer no Amor. É preciso que demos o primeiro passo, ou seja, que ``treinemos" nosso cérebro com compaixão, ternura, gentileza, bondade, até esse comportamento fique automático, se assim podemos dizer, em nós.