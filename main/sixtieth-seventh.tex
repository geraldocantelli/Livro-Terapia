\begin{chapterpage}{Parábolas de S. Luís}{c67_sixtieth-seventhchapter:cha}
 
\begin{myquotation} O amor resume inteiramente a doutrina de Jesus, porque é o sentimento por excelência, e os sentimentos são os instintos elevados à altura do progresso realizado.
\par\vspace*{15mm}
\mbox{}\hfill \emdash{}Lázaro (Paris, 1862)\index{Lázaro (Paris, 1862)}
, %\citetitle{bibitem}\index{@\citetitle{bibitem}} %\ifxetex\label{famousperson-bibitem-quote}\else\citep[p.~123]{bibitem}\fi
\par\end{myquotation}

\end{chapterpage}

% -------------------- replace or remove text below and paste your own text ------


\section{Dissertação Moral ditada por S. Luís à Srta. Ermance Dufaux}\label{c1_basicformatting:sec}

\begin{center}
I
\end{center}
 

\emdash{}Um homem soberbo possuía algumas jeiras[1] de boa terra. Sentia-se orgulhoso das pesadas espigas que cobriam o seu campo e lançava o olhar desdenhoso sobre o campo estéril do humilde. Esse levantava-se ao cantar do galo e ficava o dia todo curvado sobre o solo ingrato; recolhia pacientemente os seixos e ia atirá-los à beira do caminho; revolvia profundamente a terra e arrancava com dificuldade os espinheiros que a cobriam. Ora, seu suor fecundou o campo e ele colheu o melhor trigo.

\emdash{}Entretanto, o joio crescia no campo do homem soberbo e abafava o trigo, enquanto o dono se vangloriava de sua fecundidade e olhava com piedade os esforços silenciosos do humilde.

\emdash{}Em verdade vos digo que o orgulho é semelhante ao joio que afoga o bom grão. Aquele dentre vós que se julga mais que seu irmão e que se vangloria, é insensato. Sábio é o que trabalha por si mesmo, como o humilde em seu campo, sem se envaidecer de sua obra.

[1] Medida agrária com 0,2 hectare.

\begin{center}
II
\end{center}

 \emdash{}Havia um homem rico e poderoso que desfrutava o favor do prín­cipe. Morava em palácios e numerosos servos esforçavam-se por adivinhar-lhe os desejos.

\emdash{}Um dia em que suas matilhas acuavam um cervo nas profundezas da floresta, ele avistou um pobre lenhador vergando ao peso de um feixe de lenha. Chamou-o e lhe disse:

\emdash{} Vil escravo! Por que passas pelo caminho sem te inclinares perante mim? Sou igual ao Senhor: nos conselhos minha voz decide a paz e a guerra, e os grandes do reino curvam-se em minha presença. Saiba que sou sábio entre os sábios, poderoso entre os poderosos, grande entre os grandes e minha elevação é obra de minhas mãos.

\emdash{} “Senhor! ─ respondeu o pobre homem ─ temi que minha saudação humilde vos fosse uma ofensa. Sou pobre e o único bem que possuo são os meus braços, mas não desejo vossas grandezas enganosas. Durmo o meu sono e não temo, como vós, que o prazer do senhor me faça cair em minha obscuridade.

\emdash{}Ora, o príncipe entediou-se do orgulho da soberba. Os grandes humilhados ergueram-se contra ele, que foi precipitado do pináculo de seu poder, como uma folha seca que o vento varre do cume da montanha. Mas o humilde continuou pacificamente seu rude trabalho, sem preocupação pelo dia seguinte.

 
\begin{center}
III
\end{center}
 

\emdash{}Soberbo, humilha-te, porque a mão do Senhor curvará o teu orgulho até a poeira!

\emdash{}Escuta! Nasceste onde te lançou a sorte; saíste do seio materno fraco e nu como o último dos homens. Por que levantas a fronte mais alto que os teus semelhantes, tu que como eles nasceste para a dor para a morte?

\emdash{}Escuta! Tuas riquezas e tuas grandezas, vaidades das vaidades, escaparão de tuas mãos quando vier o grande dia, como as águas inconstantes da torrente que o sol evapora. Não levarás de tuas riquezas mais que as tábuas do esquife, e os títulos gravados na lápide funerária serão palavras vazias de sentido.

\emdash{}Escuta! O cão do coveiro brincará com os teus ossos, e eles serão misturados aos do mendigo; a tua poeira confundir-se-á com a dele, porque um dia vós ambos sereis apenas pó. Então amaldiçoarás os dons que recebeste, quando vires o mendigo revestido de sua glória, e chorarás o teu orgulho.

\emdash{}Humilha-te, soberbo, porque a mão do Senhor curvará o teu orgulho até o pó.


\begin{center}
(19 e 26 de Janeiro de 1858)    
\end{center}
