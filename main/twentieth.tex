\begin{chapterpage}{Lição de uma vida}{c20_twentiethchapter:cha}

\begin{myquotation}Não deixe o barulho da opinião dos outros abafar sua voz interior. E mais importante, tenha a coragem de seguir seu coração e sua intuição. Eles de alguma forma já sabem o que você realmente quer se tornar. Tudo o mais é secundário.


 
\par\vspace*{15mm}
\mbox{}\hfill \emdash{}Steve Jobs \index{Jobs, Steve}
, %\citetitle{bibitem}\index{@\citetitle{bibitem}} %\ifxetex\label{famousperson-bibitem-quote}\else\citep[p.~123]{bibitem}\fi
\par\end{myquotation}

\end{chapterpage}

% -------------------- replace or remove text below and paste your own text ------


\section{Mensagem deixada por Steve Jobs}\label{c1_basicformatting:sec}

\emdash{}Em 2011 Steve Jobs morre aos 56 anos de câncer de pâncreas, deixando uma fortuna de 7 bilhões de dólares e estas são algumas das suas últimas palavras..

" neste momento, deitado na cama, doente e lembrando toda a minha vida, percebo que todo o reconhecimento e riqueza que tenho não faz sentido diante da morte iminente. Eu tenho o dinheiro para contratar o melhor na tarefa que for, mas não é possível contratar alguém para carregar a minha doença. O dinheiro pode obter todos os tipos de coisas materiais, mas há uma coisa que não pode ser comprada: "a vida".

À medida que cresci percebi que um relógio de $ 300 e um de $ 3.000.000 mostram a mesma hora. Que com um carro de $ 150,000 e um de $ 15.000.000 podemos chegar ao mesmo destino. Que um vinho de $ 150 ou um de $ 1500, geram a mesma " ressaca ". que em uma casa de 300 metros quadrados, ou em uma de 3000, a solidão é a mesma ".

" a verdadeira felicidade não vem das coisas materiais, vem do afeto que nos dão os nossos entes queridos."

Então, espero que você entenda que quando você tem amigos ou alguém com quem falar, é a verdadeira felicidade!

Não eduque seus filhos para que eles sejam ricos. Educá-los para serem felizes. - então, quando crescerem, saberão o valor das coisas, não o preço.

Coma sua comida como medicina, caso contrário você deve comer a medicina como comida.

Quem te ama nunca vai te deixar, mesmo que tenha 100 motivos para desistir. Ele / ela sempre encontrará um motivo para se apegar.

Há uma grande diferença entre ser humano e se humano.

Se você quiser ir rápido, vá sozinho! Mas se você quiser ir longe, vá acompanhado.

Os seis melhores médicos do mundo são:

\begin{enumerate}
    \item a luz do sol
    \item o descanso
    \item o exercício
    \item a dieta
    \item a confiança em si mesmo
    \item os afetos
\end{enumerate}

Em qualquer etapa da vida em que você se encontre agora, agradeça e aproveite ao máximo das pequenas coisas e valorize o amor do seu casal, sua família e seus amigos, para que quando chegar o dia em que a cortina baixar, você possa levar Com você a verdadeira riqueza deste mundo!!

Steve Jobs.

\emdash{}Esta mensagem possui uma riqueza imensa e nos motiva a viver melhor qualquer que seja a posição em que nos encontremos no momento atual.

\emdash{}Precisamos deixar de lado discussões inúteis e aborrecimentos como ofensas pessoais pois essas coisas nos trazem um prejuízo incalculável. Não vemos o mundo espiritual (a maioria de nós) que nos rodeia mas sentimos as presenças e energias boas e ruins que provem de atos de amor ou de desamor. Desde essa vida até o momento da partida e depois colhemos os frutos de uma palavra mal dita e atos não pensados portanto cortemos já essas atitudes.

\emdash{}Mensagens como esta de Jobs mostram o que realmente importa na vida e com certeza o que realmente importa não é ter a última palavra numa discussão. Jesus trouxe uma proposta de vida de amor e perdão para nos libertar. O Evangelho liberta! Se ainda não somos livres em nossos sentimentos é porque ainda não vivemos o Evangelho de Jesus como ele deveria ser vivido, é sintomático. 

\emdash{}E às vezes essas energias negativas somatizam no corpo em forma de doença e até podemos morrer por isso. Mas tudo é um aprendizado, se aprendermos a conviver em amor e perdão então não adoeceremos das enfermidades fruto do ódio e da falta de perdão.

\emdash{}Às vezes uma pessoa tão boa desenvolve um câncer e não entendemos os motivos e claro, não podemos julgar, mas a causa pode estar em outra vida (passada) ou ela pode ter pedido a doença como provação. Uma coisa é certa: os infortúnios que julgamos desgraças podem ser apenas instrumentos de Deus para nosso aperfeiçoamento moral, como uma esquizofrenia por exemplo. 

\emdash{}Afinal, aprendemos mais lições de vida passando por dificuldades do que em tempos de bonança. Infelizmente é assim que nós aprendemos. Se nós aprendêssemos mais lições pelo amor sem necessidade da dor, assim seria. Daí a importância de um trabalho voluntário, da caridade, de ajudar o próximo como forma de começar a inverter essa lógica e começarmos a aprender mais pelo amor até que um dia estaremos livres da necessidade da dor como professora e esse dia chegará.