\begin{chapterpage}{Realidades Espirituais}{c52_fiftieth-secondchapter:cha}
 
\begin{myquotation}Há mais coisas entre o céu e a terra do que sonha nossa vã filosofia.
\par\vspace*{15mm}
\mbox{}\hfill \emdash{}William Shakespeare\index{Shakespeare, William}
, %\citetitle{bibitem}\index{@\citetitle{bibitem}} %\ifxetex\label{famousperson-bibitem-quote}\else\citep[p.~123]{bibitem}\fi
\par\end{myquotation}

\end{chapterpage}

% -------------------- replace or remove text below and paste your own text ------


\section{De onde viemos, quem somos, para onde vamos?}\label{c1_basicformatting:sec}

\emdash{}Há muito tempo a humanidade se faz essas perguntas, desde quando se olhava para as estrelas procurando entender nosso mundo até hoje. O materialismo foi criado para tentar ter-se um modelo que explicasse a Vida e que pudesse nortear os pensamentos ávidos de respostas, mas não foi suficiente nem adequado pois várias realidades que acontecem com as pessoas escapam às leis até agora conhecidas da ciência. O avanço da desta mesma ciência nos possibilita pensar com mais liberdade do que na Idade Média, quando estávamos no obscurantismo de pesadas opressões para a alma que até hoje tem suas consequências na sociedade.

\emdash{}Depois veio o Iluminismo, ou a idade das luzes, que possibilitou uma abertura de pensamento como que a preparar a humanidade para a chegada no século XIX dos estudos de Espiritismo de Allan Kardec. A certeza da continuidade da Vida após a morte e das relações entre o mundo espiritual e o material trouxeram outra perspectiva à ciência, à filosofia e à religião. Mistérios antes ocultos são revelados e podem ser provados pela lógica e a razão. 

\emdash{}Muitas das vezes combatido como coisa herege pelas outras religiões ou também como ilusão por alguns médicos e cientistas, esse novo pensamento revoluciona a maneira de a pessoa elevar-se moralmente e entender o Evangelho de Jesus. Trechos da Bíblia ininteligíveis até então são explicados de maneira fácil de entender e a Palavra torna-se acessível a todos que se abrirem a Deus. Está escrito que se calarem as bocas dos profetas, as pedras falarão e hoje aqueles que nos precederam no além túmulo vêm falar do Evangelho do Mestre e pregar o Bem, a Paz e a Caridade.

\emdash{}O próprio Jesus foi acusado de não ser de Deus pelos seus contemporâneos mas ele disse que uma casa edificada contra si mesma não fica de pé e que se Satanás expulsa Satanás como poderá o seu reino subsistir? Portanto se o Espiritismo não fosse de Deus, não falaria do Evangelho de Jesus, não ensinaria a moral e o bem e a caridade. Jesus também disse que a árvore se reconhece pelos frutos, que a árvore boa não dá mau fruto e a má árvore não dá bom fruto; quantos realmente bons frutos vem do Espiritismo: pessoas são curadas, largam dos vícios, fazem caridade, são esclarecidas sobre questões básicas da vida e assim não precisam mais se sentirem perdidas. 


O Espiritismo relembra as coisas que Jesus disse e as explica levando as pessoas a praticar seus ensinamentos e serem pessoas melhores, boas, caridosas. Nele vários são curados e muitos demônios (ou espíritos imperfeitos) são não somente expulsos mas convencidos a deixar em paz suas vítimas (por vezes passam até a ajudá-las).

\emdash{}Com o avanço das ciências, o surgimento da Teoria da Relatividade e a Física Quântica pode-se entender mais da nossa realidade e vimos por exemplo que as energias se conservam além da matéria condensada, tendo no espiritismo a certeza da manutenção da individualidade da consciência na energia mais sutil ou alma, depois de desligar-se do corpo. Quantos suicídios foram evitados por saber a pessoa de antemão que com o fim do corpo seus problemas não acabariam mas sim seriam agravados com as consequências no pós-vida de tão funesto ato.

\emdash{}O Espírito é a centelha Divina que possui a capacidade de pensar e sentir e é todo imaterial assim como o próprio Deus é imaterial. Nós fomos criados à Sua imagem e semelhança portanto se Ele é espírito, nós também o somos afinal fomos nós que fomos criados à Sua imagem e semelhança e não Deus à nossa (antromorfização de Deus).

\emdash{}A energia mais condensada que é a matéria se liga a esse espírito por um corpo intermediário, que Kardec chamou de perispírito. Esse corpo semi-material nos acompanha depois da morte do corpo físico e será assim por muito tempo. Durante nossa evolução precisamos desse corpo mas um dia, quando atingirmos a perfeição relativa, nem dele precisaremos.

\emdash{}Sem afobação, não tenhamos pressa pois a evolução não dá saltos, como diz Emmanuel, espírito que era o mentor de Chico Xavier. Com paz no coração vamos seguindo em frente, sem medo. O Reino dos Céus é paz e gozo no coração do homem, como diz na Bíblia e podemos já no Caminho para a perfeição sentir felicidade pois como disse Buda: Não há caminho para a felicidade, a Felicidade é o Caminho.