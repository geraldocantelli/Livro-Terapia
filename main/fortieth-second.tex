\begin{chapterpage}{Estar Por Inteiro}{c42_fortieth-secondchapter:cha}
 
\begin{myquotation}Onde quer que você esteja, esteja por inteiro.
\par\vspace*{15mm}
\mbox{}\hfill \emdash{}Eckhart Tolle\index{Tolle, Eckhart}
, %\citetitle{bibitem}\index{@\citetitle{bibitem}} %\ifxetex\label{famousperson-bibitem-quote}\else\citep[p.~123]{bibitem}\fi
\par\end{myquotation}

\end{chapterpage}

% -------------------- replace or remove text below and paste your own text ------


\section{A importância de estar por inteiro}\label{c1_basicformatting:sec}

\emdash{}A oração que inicia esse capítulo (frase não tem verbo, oração é o nome que se dá quando apresenta verbo) é mais importante do que se pode imaginar a primeira vista. Precisamos estar presentes em nossa própria vida ou seja conscientes de nossos atos e presença de vida e da Presença de Deus em nossa vida para que não sejamos assaltados por coisas ou pessoas desse ou de outro mundo. 

\emdash{}Jesus certa feita disse: ``Quando um espírito imundo sai de um homem, passa por lugares áridos procurando descanso, e, não o encontrando, diz: ‘Voltarei para a casa de onde saí’.  Quando chega, encontra a casa varrida e em ordem.  Então vai e traz outros sete espíritos piores do que ele, e entrando passam a viver ali. E o estado final daquele homem torna-se pior do que o primeiro”. Jesus não queria nos assustar com esse alerta mas apenas mostrar a importância da Presença, no nosso estado de consciência para que tal coisa não nos aconteça.

\emdash{}Quem manda em nós somos nós e Deus somente mas para manter essa soberania é importante seguir o Evangelho de Jesus pois o contrário disso, ou seja, o orgulho, o egoísmo, o fanatismo são portas abertas a que caiamos na inconsciência e possamos perder o próprio controle: então entramos num automático mas que na verdade é obsessão.

\emdash{}É preciso autoridade moral para afastar as más influências, mais autoridade moral do que elevar a voz, e Deus está conosco sempre. Também está escrito: ``Buscai em primeiro lugar as coisas do Alto e tudo o mais voz será acrescentado". Orar, pedir a Jesus que acalme o mar agitado de nossa mente como fez com o mar da Galiléia quando os discípulos o acordaram pedindo socorro por causa das ondas gigantes que quase viraram o barco; e depois o mar ficou calmo e cristalino como um lago.

\emdash{}É o que Deus faz com nossa vida toda, com nossa mente, ele acalma, acerta, coloca nos eixos mas temos que dar uma chance para Ele, também é bom buscar conhecê-Lo e auto-conhecimento, praticar 5 minutos de meditação por dia (por exemplo atenção plena). Nesse ponto como em tudo na vida Equilíbrio é a chave: nem tanto ao mar nem tanto a terra, mas com perseverança e constância. 

\emdash{}Se for possível e julgar necessário procure ajuda psiquiátrica/psicológica/espiritual, sempre com discernimento e vá avaliando sua melhora com o passar do tempo e avaliando a própria ajuda também, não vai se arrepender pois o único arrependimento seria ficar parado sem nada fazer para melhorar-se. Ainda que encontre-se em uma ajuda que você avalie como inadequada, troque-a por outra e siga em frente com as ajudas, mas faça alguma coisa.

\emdash{}O tempo é amigo de quem persevera e ele vai passar de qualquer jeito.