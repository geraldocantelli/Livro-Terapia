\begin{chapterpage}{As 100 frases}{c44_fortieth-forthchapter:cha}
 
\begin{myquotation}A felicidade real começa em fazer a felicidade dos outros.
\par\vspace*{15mm}
\mbox{}\hfill \emdash{}Chico Xavier\index{Xavier, Chico}
, %\citetitle{bibitem}\index{@\citetitle{bibitem}} %\ifxetex\label{famousperson-bibitem-quote}\else\citep[p.~123]{bibitem}\fi
\par\end{myquotation}

\end{chapterpage}

% -------------------- replace or remove text below and paste your own text ------


\section{As 100 frases mais compartilhadas de Chico Xavier}\label{c1_basicformatting:sec}

\emdash{}Em centenas de entrevistas e, em seus mais de 430 livros, Chico Xavier trouxe ensinamentos que, se aplicados, podem guiar a todos no caminho do bem. Conheça as 100 frases mais compartilhadas de Chico Xavier.

1 ``Embora ninguém possa voltar atrás para fazer um novo começo, qualquer um pode começar agora a fazer um novo fim.”

2 ``Não exijas dos outros qualidades que ainda não possuem. A árvore nascente aguarda-te a bondade e a tolerância para que te possa ofertar os próprios frutos em tempo certo.”

3 ``Deixe algum sinal de alegria, onde passes.”

4 ``Ninguém quer saber o que fomos, o que possuíamos, que cargo ocupávamos no mundo; o que conta é a luz que cada um já tenha conseguido fazer brilhar em si mesmo.”

5 ``Sempre recebi os elogios como incentivos dos amigos para que eu venha a ser o que tenho consciência de que ainda não sou.”

6 ``A gente deve lutar contra o comodismo e a ociosidade; caso contrário, vamos retornar ao mundo espiritual com enorme sensação de vazio. Dizem que eu tenho feito muito, mas, para mim, não fiz um décimo do que deveria ter feito.”

7 ``A questão mais aflitiva para o espírito no Além é a consciência do tempo perdido.”

8 ``A felicidade real começa em fazer a felicidade dos outros.”

9 ``A vida é sempre o resultado de nossa própria escolha.”

10 ``Em matéria de felicidade convém não esquecer que nos transformamos sempre naquilo que amamos.”

11 ``Quem se aceita como é, doando de si à vida o melhor que tem, caminha mais facilmente para ser feliz como espera ser.”

12 ``Se Allan Kardec tivesse escrito que ‘fora do Espiritismo não há salvação’, eu teria ido por outro caminho. Graças a Deus ele escreveu ‘Fora da Caridade’, ou seja, fora do Amor não há salvação.”

13 ``Onde existe amor não há lugar para ressentimentos.”

14 ``A maior revelação de teu amor aparece brilhando quando permites que o Cristo em ti e contigo possa amar e servir aos outros sem procurar saber quem são e como são.”

15 ``A melhor maneira de aprender a desculpar os erros alheios é reconhecer que também somos humanos, capazes de errar talvez ainda mais desastradamente que os outros.”

16 ``Cada dia que amanhece assemelha-se a uma página em branco, na qual gravamos os nossos pensamentos, ações e atitudes. Na essência, cada dia é a preparação de nosso próprio amanhã.”

17 ``Cada boa ação que você pratica é uma luz que você acende em torno dos próprios passos.”

18 ``Cada minuto é uma semente de amor que podes cultivar ou uma abençoada luz que podes acender para o grande futuro.”

19 ``Confiemos na Providência Divina e aceitemos no serviço do bem a nossa mais bela e melhor oportunidade a que denominamos: agora.”

20 ``Depressão? Alma querida, se tens apenas tristeza, se te sentes indefesa, contra mágoa e dissabor, sai de ti mesma e auxilia aos que mais sofrem na estrada. A depressão é curada pelo trabalho de amor.”

21 ``Nunca somos tão pobres de bens materiais e espirituais que não possamos doar alguma coisa ao companheiro necessitado, seja o pão ou a palavra de consolo e solidariedade.”

22 ``O amor verdadeiro auxilia sem perguntar.”

23 ``Se quiser realmente ver o teu maior inimigo, pare por alguns instantes à frente de um espelho.”

24 ``Sempre que chamados à crítica, respeitemos o esforço nobre dos semelhantes. Para construir, são necessários amor e trabalho, estudo e competência, compreensão e serenidade, disciplina e devotamento. Para destruir, porém, basta, às vezes, uma só palavra.”

25 ``Sem a ideia da reencarnação, sinceramente, com todo respeito às demais religiões, eu não vejo uma explicação sensata, inclusive, para a existência de Deus.”

26 ``Berço e túmulo são simples marcos de uma condição para a outra. Somos responsáveis por nossa tragédia e por nossa glória.”

27 ``Hoje auxiliamos, amanhã seremos os necessitados de auxílio.”

28 ``A desilusão de agora será benção depois.”

29 ``Nenhuma atividade no bem é insignificante. As mais altas árvores são oriundas de minúsculas sementes.”

30 ``Muitos ficam na expectativa do socorro do Alto, mas não querem nada com o esforço de renovação; querem que os espíritos se intrometam na sua vida e resolvam seus problemas.”

31 ``Devemos orar pelos políticos, pelos administradores da vida pública. A tentação do poder é muito grande. Eu não gostaria de estar no lugar de nenhum deles.”

32 ``Sem Deus no coração, as futuras gerações colocarão em risco a Vida no planeta. Por maior que seja o avanço tecnológico da Humanidade, impossível que o homem viva em paz sem que a ideia de Deus o inspire em suas decisões.”

33 ``Pela força do exemplo vencerás.”

34 ``Na realidade, toda doença no corpo é processo de cura para a alma.”

35 ``Uma das mais belas lições que tenho aprendido com o sofrimento: não julgar, definitivamente não julgar a quem quer que seja.”

36 ``O exemplo é uma força que repercute de maneira imediata, longe ou perto de nós. Não podemos nos responsabilizar pelo que os outros fazem de suas vidas; cada qual é livre para fazer o que quer de si mesmo, mas não podemos negar que nossas atitudes inspiram atitudes, seja no bem ou no mal.”

37 ``Fico triste quando alguém me ofende, mas, com certeza, eu ficaria mais triste se fosse eu o ofensor. Magoar alguém é terrível!”

38 ``Perante Deus toda pessoa é importante.”

39 ``O bem que praticares em qualquer lugar é seu advogado em toda a parte.”

40 ``A criança desprotegida que encontramos na rua não é motivo para revolta ou exasperação, e sim um apelo para que trabalhemos com mais amor pela edificação de um mundo melhor.”

41 ``Às vezes, naquele minuto de oração deixamos de tomar uma atitude precipitada, de proferir uma palavra agressiva, de permitir que a cólera nos induza a qualquer atitude infeliz…”

42 ``A alegria do próximo começa muitas vezes no sorriso que você lhe queira dar.”

43 ``A crítica dos outros só poderá trazer-lhe prejuízo se você consentir.”

44 ``A dor é uma luz acesa no apoio da evolução.”

45 ``A hora que passa é preciosa demais para que lhe percamos a grandeza.”

46 ``A humildade é a chave de nossa libertação.”

47 ``O Cristo não pediu muita coisa, não exigiu que as pessoas escalassem o Everest ou fizessem grandes sacrifícios. Ele só pediu que amássemos uns aos outros.”

48 ``A marcha será medida pelo passo do serviço ao próximo.”

49 ``Toda a vida futura, no entanto, depende inevitavelmente da vida presente, como toda a colheita próxima se deriva da sementeira atual.”

50 ``A melhora de tudo para todos começa na melhora de cada um.”

51 ``A pedra colocada em disciplina é o agente que te assegura firmeza na construção.”

52 ``A serenidade e o apreço para com os inimigos são os melhores antídotos para que as preocupações com eles não nos destruam.”

53 ``A tarefa parece fracassar? Siga adiante trabalhando, que, muitas vezes é necessário sofrer, a fim de que Deus nos atenda à renovação.”

54 ``A Terra é uma embarcação cósmica de vastas proporções e não podemos olvidar que o Senhor permanece vigilante no leme.”

55 ``Toda migalha de amor está registrada na Lei, em favor de quem a emite.”

56 ``A vitória na luta pelo bem contra o mal caberá sempre ao servidor que souber perseverar com a Lei Divina até o fim.”

57 ``Aceita-te como és e aceita a vida em que deves estar, na condição em que te vês, a fim de que faças em ti o burilamento possível.”

58 ``Acentuemos, na própria vida, a disposição de aprender e auxiliar.”

59 ``Ajude conversando. Uma boa palavra auxilia sempre.”

60 ``Alma corajosa não é aquela que se dispõe a revidar o golpe recebido e sim aquela que sabe desculpar e esquecer.”

61 ``Amigo, continua servindo e não temas. Onde viste o lavrador que deitasse as sementes na terra e as visse germinar, no mesmo instante? O serviço que te confiei é aquele mesmo que o Pai me deu a fazer… Nenhum gesto de bondade e nenhuma palavra de amor se perdem na construção do Reino do Bem-Eterno.”

62 ``Ampara aos que se acham perseguidos pela ignorância ou pela crueldade.”

63 ``Ante às crises da vida, não te revoltes. Serve.”

64 ``Façamos da caridade o pão espiritual da vida.”

65 ``As almas afins se engrandecem constantemente repartindo as suas alegrias e os seus dons com a Humanidade inteira, não existindo limitações para o amor, embora seja ele também a luz divina a expressar-se em graus diferentes nas variadas esferas da vida.”

66 ``As mães e os pais terrestres foram convocados a negócios de renúncia, exemplificação e devotamento.”

67 ``Auxilia aos outros, tanto quanto puderes. Cada pessoa que hoje te encontra talvez seja amanhã a chave de que necessitas para a solução de numerosos problemas.”

68 ``Você nem sempre terás o que desejas, mas enquanto estiveres ajudando os outros encontrarás os recursos que pre-cisa.”

69 ``Cada criatura constrói na própria mente e no próprio coração o paraíso que a erguerá ao nível sublime da perfeita alegria, ou o inferno que a rebaixará aos mais escuros antros de sofrimento.”

70 ``Centraliza-te no esforço de auxiliar no bem comum, seguindo com a tua cruz, ao encontro da ressurreição divina. Nas surpresas constrangedoras da marcha, recorda que antes de tudo importa orar sempre, trabalhando, servindo, aprendendo, amando e nunca desfalecer.”

71 ``Colocar-te-ás na posição dos que sofrem, a fim de que faças por eles tudo aquilo que a ti desejarias nas mesmas circunstâncias.”

72 ``Comecemos nosso esforço de soerguimento espiritual desde hoje e, amanhã, teremos avançado consideravelmente no grande caminho!”

73 ``Compreendamos que unicamente cooperando na paz dos outros é que o concurso da paz virá ao nosso encontro.”

74 ``Compreender constantemente. Trabalhar sempre. Descansar, quando se mostre necessária a pausa de refazimento. Parar nunca.”

75 ``Confia em Deus, mas não te esqueças de que Deus confia em ti.”

76 ``Corrijamos a nós mesmos, antes que o mundo nos corri-ja.”

77 ``De tudo o que semeares, efetivamente colherás.”

78 ``Dentro da visão espírita-cristã, céu, inferno e purgatório começam dentro de nós mesmos. A alegria do bem praticado é o alicerce do céu. A má intenção já é um piso para o purgatório e o mal devidamente efetuado, positivado, já é o remorso que é o princípio do inferno.”

79 ``Deus colocou a esperança em cada realização da Natureza, por que haveremos nós de desesperar?”

80 ``Dificuldades que te surpreendam são os testes aconselháveis em que te cabe encontrar aproveitamento.”

81 ``Enquanto houver um gemido na paisagem em que nos movimentamos, não será lícito cogitar da felicidade isolada para nós mesmos.”

82 ``Esquece injúrias e ofensas. Não lastimes o passado. Não censures a ninguém. Segue sempre para diante e não temas. Deus vigia.”

83 ``Estenda a mão ao que necessita de apoio. Chegará seu dia de receber cooperação.”

84 ``Examina o sentido, o modo e a direção de tuas palavras, antes de pronunciá-las.”

85 ``Nada se realiza de útil e grande sem a coragem.”

86 ``Não critiques. A lâmina de nossa reprovação volta-se, invariavelmente, contra nós, expondo-nos as próprias deficiências.”

87 ``Não desesperes. O raio de nossa inconformação aniquilará a sementeira de nossos melhores sonhos.”

88 ``Não exija perfeição nos outros e nem mesmo em você, mas procure melhorar-se quanto possível.”

89 ``Não firas. O golpe da nossa crueldade brandido na direção dos outros, retornará a nós mesmos, inevitavelmente, fazendo chagas de dor e aflição no corpo de nossa vida.”

90 ``Não nos esqueçamos de que o filho descuidado, ocioso ou perverso é o pai inconsciente de amanhã e o homem inferior que não fruirá a felicidade doméstica.”

91 ``Não se esqueças de que casar é tarefa para todos os dias, porquanto somente da comunhão espiritual gradativa e profunda é que surgirá a integração dos cônjuges.”

92 ``Não te encolerizes. O punhal de nossa ira alcança-nos a própria saúde, impondo-nos o vírus da enfermidade.”

93 ``Ninguém recolhe o bem sem conquistá-lo e ninguém recebe o mal sem atraí-lo.”

94 ``Cada hora na vida é recurso potencial para a criação de novos destinos.”

95 ``Na vida, não vale o que temos nem tanto importa o que somos. Vale o que realizamos com aquilo que possuímos e, acima de tudo, importa o que fazemos de nós.”

96 ``O seu pior momento na vida é sempre o instante de melhorar.”

97 ``Recorda: felicidade é uma construção a fazer. O alicerce está em ti mesmo.”

98 ``Ouve os que te busquem a presença ou a palavra, com bondade e simpatia.”

99 ``Podes contar com Deus na solução de todos os teus problemas, entretanto, não te esqueças de que Deus conta contigo em todos os teus caminhos.”

100 ``Quem perdeu a própria fé nada mais tem a perder.”

\emdash{}Texto retirado do site Mensagem Espirita, em 2 de maio de 2020, no endereço https://www.mensagemespirita.com.br/chico-xavier/ad/as-100-frases-mais-compartilhadas-de-chico-xavier.