\begin{chapterpage}{Carta aos Romanos 8}{c71_seventieth-firstchapter:cha}
 
\begin{myquotation}Quem vive segundo a carne tem a mente voltada para o que a carne deseja; mas quem, de acordo com o Espírito, tem a mente voltada para o que o Espírito deseja.
\par\vspace*{15mm}
\mbox{}\hfill \emdash{}Paulo, Romanos 8:5\index{Romanos 8:5, Paulo}
, %\citetitle{bibitem}\index{@\citetitle{bibitem}} %\ifxetex\label{famousperson-bibitem-quote}\else\citep[p.~123]{bibitem}\fi
\par\end{myquotation}

\end{chapterpage}

% -------------------- replace or remove text below and paste your own text ------


\section{Pensar como o Espírito é Libertador}\label{c1_basicformatting:sec}

\emdash{}Os hebreus acreditavam muito na restauração e na implantação do Reino de Deus e que isso só poder-se-ia dar com a Justiça que viria no julgamento das nações, por isso o texto descrito no Apocalipse. Mas essa justiça se daria na carne para que o espírito pudesse ficar livre e voltar a viver sob Deus. Os homens que, na estória de Adão e Eva inicialmente viviam sob esse maravilhoso Deus, decidiram por viver de acordo com as próprias ideias e serem os governantes de si mesmos: foi o pecado original e desde então auto impuseram-se o reino dos homens e saíram do Reino de Deus, ou seja, Deus deixou de ser Rei para eles.

\emdash{}Haroldo Dutra Dias, palestrante, conta que se pegarmos as falas de Jesus, muitas vezes Ele diz o termo ```Reino dos Céus''. Ele veio para nos colocar de novo sob o reinado de Deus, mas isso não é possível se continuarmos com a cabeça pensando como o mundo nos ensinou a pensar e os pensamentos do mundo levam à morte e os do Espírito levam a vida. Pode verificar por si mesmo: se um pensamento nos entristece ou angustia então ele é fruto de se pensar como o mundo ensina, já se um pensamento é libertador e vivificante, ele é segundo Deus.

\emdash{}Paulo diz na sua carta aos romanos 8:

``Portanto, agora já não há condenação para os que estão em Cristo Jesus,
porque por meio de Cristo Jesus a lei do Espírito de vida me libertou da lei do pecado e da morte.
Porque, aquilo que a lei fora incapaz de fazer por estar enfraquecida pela carne, Deus o fez, enviando seu próprio Filho, à semelhança do homem pecador, como oferta pelo pecado. E assim condenou o pecado na carne,
a fim de que as justas exigências da lei fossem plenamente satisfeitas em nós, que não vivemos segundo a carne, mas segundo o Espírito.

Quem vive segundo a carne tem a mente voltada para o que a carne deseja; mas quem, de acordo com o Espírito, tem a mente voltada para o que o Espírito deseja.

A mentalidade da carne é morte, mas a mentalidade do Espírito é vida e paz;
a mentalidade da carne é inimiga de Deus porque não se submete à lei de Deus, nem pode fazê-lo.
Quem é dominado pela carne não pode agradar a Deus.
Entretanto, vocês não estão sob o domínio da carne, mas do Espírito, se de fato o Espírito de Deus habita em vocês. E, se alguém não tem o Espírito de Cristo, não pertence a Cristo.
Mas se Cristo está em vocês, o corpo está morto por causa do pecado, mas o espírito está vivo por causa da justiça.
E, se o Espírito daquele que ressuscitou Jesus dentre os mortos habita em vocês, aquele que ressuscitou a Cristo dentre os mortos também dará vida a seus corpos mortais, por meio do seu Espírito, que habita em vocês.

Portanto, irmãos, estamos em dívida, não para com a carne, para vivermos sujeitos a ela.
Pois se vocês viverem de acordo com a carne, morrerão; mas, se pelo Espírito fizerem morrer os atos do corpo, viverão,
porque todos os que são guiados pelo Espírito de Deus são filhos de Deus.
Pois vocês não receberam um espírito que os escravize para novamente temer, mas receberam o Espírito que os adota como filhos, por meio do qual clamamos: "Aba, Pai".

O próprio Espírito testemunha ao nosso espírito que somos filhos de Deus.
Se somos filhos, então somos herdeiros; herdeiros de Deus e co-herdeiros com Cristo, se de fato participamos dos seus sofrimentos, para que também participemos da sua glória.
Considero que os nossos sofrimentos atuais não podem ser comparados com a glória que em nós será revelada''.