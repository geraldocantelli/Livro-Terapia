\begin{chapterpage}{O Amoroso Apelo}{c35_thirtieth-fifthchapter:cha}

\begin{myquotation}Venho eu, vosso Salvador e vosso Juiz. Venho, como outrora, aos filhos transviados da casa de Israel. Venho trazer a Verdade e dissipar as trevas. Escutai-me: o Espiritismo, como outrora a minha palavra, tem de lembrar aos materialistas que acima deles reina a imutável Verdade: o Deus bom, o Deus grande, que faz germinar a planta e que levanta as ondas.  Revelei a doutrina divina. Como o ceifeiro, atei em  feixes o bem esparso na humanidade e disse: — Vinde a mim vós todos que sofreis!  Mas, ingratos, os homens se desviaram do caminho reto e largo que conduz ao reino de meu Pai e se perderam nas ásperas veredas da impiedade! Meu Pai não quer aniquilar a raça humana; quer, não mais por meio de profetas; não mais por meio de apóstolos, porém que, ajudando-vos uns aos outros, mortos e vivos, isto é, mortos segundo a carne, porquanto a morte não existe, vos socorrais e que a voz dos que desencarnaram se faça ouvir, clamando-vos: - Orai e crede. A morte é a ressurreição. E a vida é a prova escolhida, durante a qual, cultivadas, vossas virtudes têm de crescer e de se desenvolver como o cedro. Crede nas vozes que vos respondem. São as próprias almas dos que evocais.  Só muito raramente me comunico. Meus amigos, os que assistiram a minha vida e a minha morte, são os intérpretes divinos da vontade de meu Pai. Homens fracos que acreditais no erro de vossas inteligências obscuras, não apagueis o facho que a demência divina vos coloca nas mãos para clarear a estrada e reconduzir-vos, filhos perdidos, ao regaço de vosso Pai! 
\par\vspace*{15mm}
\mbox{}\hfill \emdash{}Espirito da Verdade\index{Espírito da Verdade}
, %\citetitle{bibitem}\index{@\citetitle{bibitem}} %\ifxetex\label{famousperson-bibitem-quote}\else\citep[p.~123]{bibitem}\fi
\par\end{myquotation}

\end{chapterpage}

% -------------------- replace or remove text below and paste your own text ------


\section{Mensagem}\label{c1_basicformatting:sec}

\emdash{}Continua o texto psicografado: ``Em verdade vos digo: Crede na diversidade, na multiplicidade dos espíritos que vos cercam; estou infinitamente tocado de compaixão pelas vossas misérias, pela vossa imensa fraqueza, para deixar de estender mão protetora aos infelizes transviados que, vendo o céu, caem no abismo do erro. Crede, amai, compreendei as verdades que vos são reveladas. Não mistureis o joio com o bom grão, os sistemas com as verdades! Espíritas, amai-vos, eis o primeiro ensino. Instruí-vos, eis o segundo. Todas as verdades se encontram no Cristianismo; são de origem humana os erros que nele se enraizaram. Eis que do além-túmulo que julgáveis o nada vos clamam vozes: - Irmãos, nada perece. Jesus Cristo é o vencedor do mal, sede os vencedores da impiedade!".

\emdash{}Quando Kardec recebeu essa mensagem, não a publicou de imediato, mas a avaliou sob todos os aspectos para verificar se era possível que o conteúdo da mesma fosse de quem dizia ser o remetente. Jesus disse que não nos deixaria órfãos e que rogaria ao Pai para que nos enviasse o Espírito da Verdade, que nos lembraria tudo o que Ele nos disse e diria tudo o mais que havia para ser ensinado. Hoje em dia, pela boca dos espíritos e com a supervisão da espiritualidade superior, este ensino é realizado em todas as partes do mundo para todas as classes sociais e explicado de forma sem alegorias e complexidade para que todos possam compreender.

\emdash{}É chegada a hora de se desanuviar as trevas da ignorância sobre vários pontos em que jazíamos por milênios. Como disse Jesus: conhecereis a Verdade e a Verdade vos libertará, a morte não é o fim de nada, apenas uma passagem. Chega de viver sem sentido e esperança. A fé torna-se uma realidade palpável a todos pois é provada, explicada, em outras palavras: raciocinada; tirando então a dúvida que gerava o medo e as angústias tiranas da alma.

\emdash{}Pavimenta-se a estrada para o reinado da Paz, do Príncipe da Paz, onde o entendimento entre as pessoas, fruto do conhecimento das leis de causa e efeito e a humildade praticada fruto da consciência da realidade da Vida adoçam a existência. Longo trajeto até essa situação beatífica? O importante é que começamos e estamos caminhando para lá. Como será até que se realize? Confiemos em Deus. É caminhando que se faz o Caminho: a Providência se faz presente também no Progresso.