\begin{chapterpage}{Enchendo-se de Luz}{c61_sixtieth-firstchapter:cha}
 
\begin{myquotation}Não ergas alto um edifício sem fortes alicerces, se o fizeres viverás com medo.
\par\vspace*{15mm}
\mbox{}\hfill \emdash{}Sabedoria Persa\index{Persa, Sabedoria}
, %\citetitle{bibitem}\index{@\citetitle{bibitem}} %\ifxetex\label{famousperson-bibitem-quote}\else\citep[p.~123]{bibitem}\fi
\par\end{myquotation}

\end{chapterpage}

% -------------------- replace or remove text below and paste your own text ------


\section{Enchendo a mente com Luz}\label{c1_basicformatting:sec}

\emdash{}Contou Divaldo Pereira Franco uma estória da qual podemos tirar valiosas lições: Um pai muito rico tinha dois filhos e pretendia deixar a maior parte da herança para aquele que fosse mais zeloso. Então mostrou-lhes dois quartos e propôs a cada um encher o espaço do quarto da melhor maneira possível com a maior economia. No outro dia, o primeiro filho mostrou ao pai o quarto cheio de feno, dizendo ``Querido pai, o quarto está cheio e não custou quase nada pois o feno eu encontrei no pasto de nossa propriedade" e o pai lhe disse ``muito bem meu filho". Depois o segundo filho chamou o pai e mostrando o quarto e acendendo a luz disse ``Meu pai, enchi o quarto com Luz, está todo preenchido até os mais recôndidos cantos e frestas".

\emdash{}O pai ficou muito feliz com a solução desse filho e asserverou ``Devemos encher-nos, encher nossa mente e corpo, com Luz para sermos felizes e servirmos melhor a Deus pois Jesus disse se teu olho é são, a Luz entrará e todo o teu ser será Luz (ele dizia em sentido alegórico, espiritual)". Por outro lado, se enchermos nosso ser com feno, os animais ruminantes virão pastar sobre nós.

\emdash{}Como se diz: a Vida é uma Escola. E a dor é uma professora muito eficiente. Não fique triste se estiver passando pela dor, Jesus era uma pessoa feliz e passou por situações dolorosas e nos trouxe a Boa Nova, segundo Divaldo a Boa Nova só poderia ter sido trazido por uma pessoa feliz, não poderia ser diferente.

