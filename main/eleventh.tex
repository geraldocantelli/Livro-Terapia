\begin{chapterpage}{Resiliência}{c11_eleventhchapter:cha}

\begin{myquotation} Pois não é contra homens de carne e sangue que temos de lutar, mas contra os principados e potestades, contra os príncipes deste mundo tenebroso, contra as forças espirituais do mal (espalhadas) nos ares. 
Tomai, por tanto, a armadura de Deus, para que possais resistir nos dias maus e manter-vos inabaláveis no cumprimento do vosso dever. 

\par\vspace*{15mm}
\mbox{}\hfill \emdash{}Paulo, Efésios 6:11-12 \index{Paulo, Efésios 6:11-12}
, %\citetitle{bibitem}\index{@\citetitle{bibitem}} %\ifxetex\label{famousperson-bibitem-quote}\else\citep[p.~123]{bibitem}\fi
\par\end{myquotation}

\end{chapterpage}

% -------------------- replace or remove text below and paste your own text ------


\section{Visão Espiritual}\label{c1_basicformatting:sec}

\emdash{}Nós somos tentados e isso é normal. Às vezes pensamentos que querem dizer que não estamos no caminho certo vêm para querer justamente nos desviar do Caminho certo, é a tentação. 
Por mais que meditemos e nos esforcemos jamais seremos oniscientes,  e seremos tentados, como aliás o próprio Jesus o foi. Jamais desanimar nesses incidentes e haurir mais forçar ainda em Deus para prosseguir porque não haveria resistência se estivéssemos no caminho errado, pelo contrário, os pensamentos frívolos fluiriam constantemente e livremente.

\emdash{}Pode-se ver a importância de uma ideia pela força que fazem contrária a ela, isso seja manifestado internamente ou externamente a nós e esse pode ser mais um sinal de estar no caminho certo se é em direção e sentido ao Amor.

\emdash{}Há espalhados nos ares muitas forças, como disse São Paulo. Entendo como espíritos que nos inspiram pensamentos, atitudes e palavras. Cabe a nós filtrar muito bem para não sermos joguetes de ninguém. O modelo é Jesus sempre. 

\emdash{}Não estamos aqui para agradar ninguém que busque nos desviar, encarnado ou desencarnado, apenas para ser autênticos e seguir ao Mestre da Vida, o Amor, Agora.

\emdash{}Hoje agradeci a Deus toda a Verdade que me foi revelada até o momento e toda a libertação que ela gerou. Gratidão. Bendito seja o Senhor.

\emdash{}Temos tudo em nossa mente e podemos acessar de uma vez a qualquer momento. Todas as Verdades de Jesus ouvidas por nós até hoje estão em nossa memória e como o Espírito Consolador (de Verdade) nos sopra mais verdades a medida que o indagamos, vamos nos enriquecendo para a Vida e podemos acessar imediatamente.

\emdash{}Ouvimos mais quando participamos de um estudo bíblico, assistimos a uma palestra, meditamos as palavras do Mestre Jesus. Sempre com discernimento, isso é muito importante, pois muitas vozes falam sobre o Alto mas nem todas em consoante com o Alto.
Uma forma de discernir é ver se mostram um Jesus com doçura e amor, então provavelmente são procedentes, caso contrário, equívocos.

\emdash{}As pessoas precisam ser acostumadas desde pequenas com a imagem de Jesus doce, sereno, bondoso, amoroso e compassivo. Parece estranho falar mas sinto que no inconsciente das pessoas há uma outra imagem do Mestre devido ao mau emprego da palavra Deus nos últimos milênios. Os hebreus nem utilizavam essa palavra, eles diziam "o Senhor" e há até um mandamento que regula sua utilização justamente para que não acontecesse o que aconteceu e foi explicado no começo deste parágrafo. E se as pessoas pensam que Deus não é amável e doce então elas se desculpam para também não o ser e aí ficamos distantes do mundo que Jesus veio inaugurar.

\emdash{}Paz e bem. Deus abençoe a todos.