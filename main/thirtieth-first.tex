\begin{chapterpage}{Educação}{c32_thiertieth-firstchapter:cha}

\begin{myquotation}É na educação dos filhos que se revelam as virtudes dos pais.

\par\vspace*{15mm}
\mbox{}\hfill \emdash{}Coelho Neto \index{Neto, Coelho}
, %\citetitle{bibitem}\index{@\citetitle{bibitem}} %\ifxetex\label{famousperson-bibitem-quote}\else\citep[p.~123]{bibitem}\fi
\par\end{myquotation}

\end{chapterpage}

% -------------------- replace or remove text below and paste your own text ------


\section{Um tanto mais sobre Educação}\label{c1_basicformatting:sec}

\emdash{}Há alguns dias enviei uma cópia desse livro para meu bom  amigo Kleber André Carlos Moreira que, em resposta, presenteou-me com as palavras que publico aqui nesse capítulo por considerar que elas muito tem a contribuir a quem vier a lê-las:

\emdash{}A ideia da punição como consequente regulador de comportamento remota a tempos antigos, mito longínquo. Forças da natureza, incontroláveis, devastavam tudo e todos. De certa forma, nos remete a ideia de punição por algo maior do que nós, que nos controla, nos dando falsa ideia de educação.

\emdash{}Na época de Moisés, a lei Mosaica nos trouxe a ideia de um Deus punitivo, forçando-nos a moldar comportamentos mais adequados para vivência social. Trazemos esse conceito nas religiões cristãs até os dias atuais.

\emdash{}As relações sociais foram moldadas, ao longo dos tempos, por métodos punitivos. Em todos os tempos tivemos escravos, os quais eram moldados em seu comportamento através de punição. Tivemos povos dominantes, que exerciam poderio através de punição severa quando da não obediência. Em menor grau, de forma velada, isso perdura até os dias atuais.

\emdash{}Uma vez que fui punido, não podendo ``descontar” em quem me puniu, faço isso com menores que eu – filhos, netos, sobrinhos, subordinados etc. E aqui não coloco isso de forma maldosa. Na maioria dos casos é inconsciente. Faz-se por semelhança, espelho. Recebeu, distribui.

\emdash{}No que tange a educação, temos de separar em duas vertentes: a educação que nos torna seres mais inteligentes, por meio da ciência, e a educação moral, que nos torna seres mais conscientes de nossos deveres para conosco e com o próximo. Nesse sentido, a primeira nos é transmitida através dos estudos na escola, faculdade, universidade etc., não sendo de responsabilidade dessas entidades, nos fornecer a educação moral. Essa, é de dever e responsabilidade única dos pais, seres que foram designados a cuidar da formação desse espírito, filho de Deus, colocado sob seus cuidados. Os cuidados pelos responsáveis não se resumem apenas aos de cunho material. É dever dos pais atuar para reprimir as más inclinações trazidas de experiências anteriores desse ser espiritual, as quais, em momento oportuno insurgir-se-ão, concorrendo para o declínio moral do espírito encarnado. Essas más inclinações já se despertam em tenra idade. As duas educações andam de mãos dadas. A educação da inteligência, nos fornece o caminho para conhecermos. A educação moral nos fornece o caminho do discernimento. Conhecer o bem, não significa fazer o bem. Conhecer o bem, é de cunho inteligente. Fazer o bem, em detrimento do mal, é de cunho da moral.

\emdash{}A relação de Deus para com os homens é de amor e disciplina. Deus nos criou simples e ignorantes, para que conquistemos nossa felicidade chegando a espíritos puros, livres das imperfeições morais que nos assolam a existência. A todo momento Deus envia seus emissários para nos ajudar nessa caminhada. O amor dele por nós é incondicional. Essa jornada de forma alguma passa pelo conceito de que temos liberdade total de ação. Temos limites claros e definidos. Aqui mora a disciplina. Para resumir, o principio básico que rege a relação entre os homens é: faça para os outros o que queríeis que vos fizessem. Ninguém quer o mal para si, somente o bem, portanto, também façamos o bem aos outros. Uma ave não consegue sobrevoo com apenas uma asa. Precisamos da asa do amor e da disciplina - limites para que sejamos completos. Como por eras, entendemos que punição molda comportamento, acabamos de forma equivocada, associando isso a implantação dos limites na educação dos seres sob nossos cuidados.

\emdash{}Estamos encontrando o caminho. Passamos de gerações que se utilizavam da punição física como forma de educar, para gerações que não dão os devidos limites. A confusão continua. Vamos entender ainda que o limite mora na educação moral, e não na punição física. Esse, nos fornece disciplina e a oportunidade de podermos conviver na sociedade. Em qualquer tempo tem de ser utilizado. É a base de nossa formação como seres. A punição física não tem mais espaço, já teve.

\emdash{}Também é necessário lembrar que somente a cerca de 200 a 300 anos para cá, começamos a entender a criança como criança, com necessidades diferentes, e não como um adulto em miniatura. Sendo um adulto em miniatura, poderia receber tudo quanto um adulto, inclusive punição física. Jean-Jacques Rousseau defendeu que devemos respeitar a natureza da criança e o mundo da criança com suas especificidades. Foi o precursor da ideia de que a criança não deveria ser considerada como um adulto em miniatura.

\emdash{}Por tempos, tínhamos a ideia de que deveríamos deixar um mundo melhor para as gerações. Eu defendo a ideia de que devemos é deixar pessoas melhores para o mundo, conscientes de suas responsabilidades consigo mesmo e com o próximo. Somente assim, teremos uma sociedade melhor, mais consciente.

\emdash{}É preciso fazer um outro ponto sobre educação. Os pais nunca deveriam ter saído da posição de pais. Nos dias atuais, assumiram o conceito de que são amigos de seus filhos. Desceram da posição de pais para de amigos. E ainda querem ser respeitados como pais. Pais são para colocar os filhos no caminho reto. Deus confiou aos pais essa missão. Acabaram que por confundir aceitação com amor. É comum vermos pais perguntando aos filhos: Filho, você me ama? E para ter uma resposta positiva, afrouxam limites, responsabilidades, dão tudo de material aos rebentos, substituem o não pelo sim. Amar é uma escolha. Amamos alguém porque respeitamos. Amor é o sentimento mais sofisticado que temos. Somente a espécie humana é capaz de sentir amor. Para que ele aconteça, existem outros sentimentos que lhe são precursores. O amor somente pode ser conquistado se antes o objeto de amor tiver sido objeto de respeito e admiração. Quem não é respeitado e admirado, jamais será amado. 