\begin{chapterpage}{Virtude}{c18_eigthteenthchapter:cha}

\begin{myquotation}O homem que é firme, paciente, simples, natural e tranquilo está perto da virtude.
 
\par\vspace*{15mm}
\mbox{}\hfill \emdash{}Confúcio \index{Confúcio}
, %\citetitle{bibitem}\index{@\citetitle{bibitem}} %\ifxetex\label{famousperson-bibitem-quote}\else\citep[p.~123]{bibitem}\fi
\par\end{myquotation}

\end{chapterpage}

% -------------------- replace or remove text below and paste your own text ------


\section{Transcendente}\label{c1_basicformatting:sec}

\emdash{}Pensar na Virtude ou pensar em Deus com muita frequência não parece agradável para muitos pois existe uma impressão que para se pensar em Deus é necessário pensar na morte e ninguém gosta de pensar nela. Mas a imagem que se fez de Deus e dessa palavra ao longo dos séculos causou esse efeito levando a não se querer pensar com frequência na própria Vida pois Deus é Vida, Amor.

\emdash{}Como está escrito que Ele nos fez à sua Imagem e Semelhança, logo se pensou em imagem e semelhança física (forma humana, antromórfica) mas como o Reino de Deus não é desse mundo, é possível que nós sejamos à sua Imagem e Semelhança Espiritual, ou seja, nós sejamos à Sua Imagem e não Ele à nossa imagem (o que é uma visão limitada).

\emdash{}E como limitou-se, nessa ideia, Deus a algo parecido com o que nós somos, logo se lhe atribuiu características humanas como vingança e outras características menos nobres da humanidade. Mas Deus, na verdade, é Amor. Não podemos afirmar com certeza como Ele é fisicamente pois Ele não é físico e sim transcendente: cada vez que se faz o que Jesus mandou: amar, perdoar, trabalhar honestamente, fazer caridade para o próximo (como o bom Samaritano) se está mais próximo de quem é Deus.

\emdash{}Assim, com esses conceitos aplicados na vida prática: amor, perdão, caridade, paciência, honestidade é que pode-se ter uma ideia do que e de quem seja Deus e não com uma ideia antropomórfica carregada de pré-conceitos incluindo defeitos de caráter. Até porque desses defeitos Jesus não tinha nenhum e Ele sim é um exemplo mais próximo de Deus que tivemos na Terra.

\emdash{}A Virtude praticada é a Comunhão com esse Deus que está em nós, nunca saiu de nós. Apenas nos esquecemos de como estamos o tempo todo com Ele e por isso não o vemos mas podemos senti-lo no amor aos irmãos e, porque não, a nós mesmos também. Compartilhar sentimentos puros é compartilhar a presença de Deus pois para acessar Sua presença não é preciso morrer pois a morte na verdade não existe, apenas uma passagem e estaremos sempre com Deus pois Ele já está aqui.

\emdash{}Diga-me se pensar isso não é reconfortante e se não dá energias para o dia-a-dia. A Salvação de Jesus traz Libertação, Iluminação desde já e passamos a ser criaturas novas e a agir no Amor, por isso está escrito: ``assim reconhecerão os meus discípulos, aqueles que amarem-se uns aos outros". Se não houver transformação na Vida da pessoa depois de conhecer o Cristo, então ela não o conheceu verdadeiramente, tem que continuar tentando... `` Àquele que bate, abrir-se-á, quem procura, acha, a quem pede, será dado". Paz e bem. Assim Seja.