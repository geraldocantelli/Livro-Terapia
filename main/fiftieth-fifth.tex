\begin{chapterpage}{Semeador}{c55_fiftieth-fifthchapter:cha}
 
\begin{myquotation}A Força do semeador vem da confiança que se tem em suas sementes.
\par\vspace*{15mm}
\mbox{}\hfill \emdash{}Desconhecido\index{Desconhecido}
, %\citetitle{bibitem}\index{@\citetitle{bibitem}} %\ifxetex\label{famousperson-bibitem-quote}\else\citep[p.~123]{bibitem}\fi
\par\end{myquotation}

\end{chapterpage}

% -------------------- replace or remove text below and paste your own text ------


\section{Parábola do Semeador}\label{c1_basicformatting:sec}

\emdash{}No Evangelho de São Marcos (capítulo 4:3-9) está escrito: ``Ouvi. O semeador saiu a semear; quando semeava, uma parte da semente caiu à beira do caminho, e vieram as aves e comeram-na. Outra parte caiu nos lugares pedregosos, onde não havia muita terra; logo nasceu, porque a terra não era profunda, e tendo saído o sol, queimou-se; e porque não tinha raiz, secou-se. Outra caiu entre os espinhos; e os espinhos cresceram, e sufocaram-na, e não deu fruto algum. Mas outras caíram na boa terra e, brotando e crescendo, davam fruto, um grão produzia trinta, outro sessenta e outro cem. Disse: Quem tem ouvidos para ouvir, ouça".

\emdash{}O próprio Jesus, quem contou essa parábola, a explica: ``Escutai, pois, vós outros a parábola do semeador. – Quem quer que escuta a palavra do reino e não lhe dá atenção, vem o espírito maligno e tira o que lhe fora semeado no coração. Esse é o que recebeu a semente ao longo do caminho. – Aquele que recebe a semente em meio das pedras é o que escuta a palavra e que a recebe com alegria no primeiro momento. – Mas, não tendo nele raízes, dura apenas algum tempo. Em sobrevindo reveses e perseguições por causa da palavra, tira ele daí motivo de escândalo e de queda. – Aquele que recebe a semente entre espinheiros é o que ouve a palavra; mas, em quem, logo, os cuidados deste século e a ilusão das riquezas abafam aquela palavra e a tornam infrutífera. – Aquele, porém, que recebe a semente em boa terra é o que escuta a palavra, que lhe presta atenção e em quem ela produz frutos, dando cem ou sessenta, ou trinta por um". (S. MATEUS, 13:18-23.)

\emdash{}Estamos todos sob a lei de causa e efeito. Tudo o que plantamos, seja ações, omissões, bons sentimentos, acolhimento de uma pessoa necessitada, consolações, enfim o que ```damos para o universo" retorna a nós e nós colheremos cedo ou tarde. Mas esse dar ao universo pode parecer muito genérico então digo que o universo é sempre quem está diante de nós no presente. Pensando assim nunca desprezaremos ninguém, o que é louvável pois todos somos filhos de Deus.

\emdash{}Especial missão tem os pregadores da Palavra de Deus que podem alcançar a muitos e levar a força e o consolo do Evangelho a seus corações. Em São Mateus (capítulo 5:14-16), Jesus diz: ``Vós sois a luz do mundo; não se pode esconder uma cidade edificada sobre um monte; nem se acende a candeia e se coloca debaixo do alqueire, mas no velador, e dá luz a todos que estão na casa. Assim resplandeça a vossa luz diante dos homens, para que vejam as vossas boas obras e glorifiquem a vosso Pai, que está nos céus". Ou seja, a palavra tem que ser proclamada, anunciada e nós precisamos ser bom terreno fértil para ela, que é a mais linda semente.

\emdash{}Temos também que tomar cuidado com nosso proceder para não causar escândalo a nossos irmãos pois se eles nos vêem em contato com a Palavra de Deus e se mesmo assim nós fizermos algo de vexaminoso, a fé de muitos pode se esfriar. 

\emdash{}Falando especificamente dos espíritas, estes não devem procurar pregar senão para os prosélitos, ou seja, aqueles que nos procurarem para se esclarecer sobre a vida espiritual e que já estão abertos para novos conceitos pois nem todas as pessoas quereriam nos ouvir e elas tem o direito de abrirem-se quando estiverem preparadas, não antes. A paz entre as diferentes consciências é uma boa semente a ser plantada e que dá frutos cem por cento.

\emdash{}Nem todas as pessoas estão maduras para as coisas do Reino dos Céus. Muitos buscam a Deus apenas para pedir, sem pensar em se doar. Ledo engano que leva as pessoas a plantarem más sementes e até se perderem no caminho da vida. Há mais alegria em dar do que em receber, isso vale para nossa relação com Deus também.

\emdash{}Reflexão faz-se necessária para uma vida proveitosa, frutuosa. Quem não reflete sobre a própria existência e o que está plantando pelo caminho pode colher tempestades e mazelas. Entretando, quem pede a Deus Sabedoria para suas ações, sentimentos e pensamentos será um bom semeador.

\emdash{}No O Evangelho segundo o Espiritismo, capítulo XVII (Sede perfeitos), o tema é abordado com as seguintes reflexões: 1. A parábola do semeador exprime perfeitamente os matizes existentes na maneira de serem utilizados os ensinos do Evangelho. Quantas pessoas há, com efeito, para as quais não passa ele de letra morta e que, como a semente caída sobre pedregulhos, nenhum fruto dá!.

\emdash{}2. Não menos justa aplicação encontra ela nas diferentes categorias espíritas. Não se acham simbolizados nela os que apenas atentam nos fenômenos materiais e nenhuma consequência tiram deles, porque neles mais não vêem do que fatos curiosos? Os que apenas se preocupam com o lado brilhante das comunicações dos Espíritos, pelas quais só se interessam quando lhes satisfazem à imaginação, e que, depois de as terem ouvido, se conservam tão frios e indiferentes quanto eram? Os que reconhecem muito bons os conselhos e os admiram, mas para serem aplicados aos outros e não a si próprios? Aqueles, finalmente, para os quais essas instruções são como a semente que cai em terra boa e dá frutos?
