\begin{chapterpage}{Imagem}{c19_nineteenthchapter:cha}

\begin{myquotation}Não farás para ti imagem de escultura, nem semelhança alguma do que há em cima no céu, nem em baixo na terra, nem nas águas debaixo da terra;


 
\par\vspace*{15mm}
\mbox{}\hfill \emdash{}Deuteronômio 5:8 \index{Deuteronômio 5:8}
, %\citetitle{bibitem}\index{@\citetitle{bibitem}} %\ifxetex\label{famousperson-bibitem-quote}\else\citep[p.~123]{bibitem}\fi
\par\end{myquotation}

\end{chapterpage}

% -------------------- replace or remove text below and paste your own text ------


\section{A figura do que transcende}\label{c1_basicformatting:sec}

\emdash{}Qual a finalidade útil de haver o Senhor proibido de que se fizesse imagens de escultura ou semelhança alguma do que havia no céu ou em baixo da terra? É um cuidado para que não nos perdêssemos nos símbolos e os tomássemos como reais, minimizando e entorpecendo o conceito das realidades que eles representam.

\emdash{}Se fosse feita uma imagem de Deus, por exemplo um velhinho barbudo ou ainda um musculoso herói, Ele, em toda a sua Imaterialidade e Transcendência, Senhor e Criador do Universo seria reduzido na cabeça das pessoas à forma de um mortal e logo se pensaria que ele teria nossas fraquezas morais como raiva, vingança, ódio, tomar partido a favor de uma pessoa ao invés de outra numa disputa. Ele está muito acima de nossas disputas particulares, governa o universo através das leis naturais que criou e não tem que se imiscuir na vida particular de cada um cada vez que alguma ``justiça" é reivindicada sob a óptica da suposta vítima.

\emdash{}A beleza, a onipotência, a providência, a magnificência, a glória, as benesses, a perfeição do Senhor são infinitas. Por isso Ele merece ser louvado e bendito a todo momento que pudermos. Não é um juiz de pequenas causas pronto a satisfazer aos humores humanos e nossos conceitos curtos de olho por olho e dente por dente. Aliás como disse Gandhi nessa história, todo mundo acabaria cego.

\emdash{}O próprio nome de Deus foi protegido por outra lei para que não fosse pronunciado a todo momento e desrespeitosamente para que as pessoas não perdessem a noção da Sua Santidade e Grandeza inclusive em momentos de pedir Sua ajuda. Mas isso aconteceu e hoje em dia muitos descreem de pedir ajuda a Ele pois após tantos séculos de Seu nome na boca do povo sendo mal usado, o conceito de Deus ficou lesado do seu sentido original.

\emdash{}E as pessoas ainda caíram no erro de criar imagens e vários nomes para um inimigo de Deus, criando uma dualidade que muitas vezes é lembrada com certo equilíbrio de forças, o que seria absurdo. Isso leva medo ou descrença ao coração de muitos, tirando a fé e tornando vacilantes pessoas que poderiam ser crentes fiéis.

\emdash{}Mas quando se diz que tem que se ter fé na Vida, na verdade está se dizendo que tem que se ter fé em Deus, pois a Vida vem de Deus e Deus é Vida, é Amor, Vida é Amor. A Sabedoria de Deus governa a Vida. Se diz também para confiar na Sabedoria do Universo, é a mesma coisa dita de outra forma. No Kardecismo se diz que Deus é inteligência suprema, causa primária de todas as coisas. 

\emdash{}Na Bíblia está escrito que o temor ao Senhor é o princípio da Sabedoria. Começa-se pelo temor pois pedagogicamente é o nosso jeito de aprender mas São João nos diz que aquele que teme castigo não conheceu verdadeiramente o Amor e não vive no Amor. Acredito que começamos no temor ao Senhor mas que depois de conhecê-lo, experimentá-lo, podemos passar a vivenciar uma experiência espiritual mais tranquila embora ainda precisemos de tribulações para evoluir por mais um bom tempo mas com confiança.

\emdash{}Jesus disse: ``Ònde está o teu tesouro, aí está o teu coração", focarmos nosso coração em Deus com discernimento é saber valorizar a Vida equilibradamente. No Oráculo de Delphos na antiga Grécia também estava escrito: ``Conhece-te a ti mesmo; mas nada em excesso", o auto-conhecimento é um Caminho para chegar-se a Deus e passa por carregar a própria cruz a que se referia o Divino Mestre. É caminhando que se faz o Caminho e quem não tem dificuldades no Caminho mas Ele nos auxilia na medida em que nos intui soluções, em que Sua Providência nos providencia ajuda propriamente falando, na medida em que crescemos em atitude e capacidade para resolvermos ``nós mesmos" nossos problemas, contudo sempre com resignação, humildade, caridade e amor.