\begin{chapterpage}{Teologia}{c56_fiftieth-sixthchapter:cha}
 
\begin{myquotation}A tarefa do teólogo: Estudar Deus e sua revelação e, em seguida, todas as demais coisas 'à luz de Deus' (\textit{sub ratione Dei}), pois Ele é o princípio e fim de tudo.
\par\vspace*{15mm}
\mbox{}\hfill \emdash{}São Tomás de Aquino\index{Aquino, São Tomás de}
, %\citetitle{bibitem}\index{@\citetitle{bibitem}} %\ifxetex\label{famousperson-bibitem-quote}\else\citep[p.~123]{bibitem}\fi
\par\end{myquotation}

\end{chapterpage}

% -------------------- replace or remove text below and paste your own text ------


\section{Porque estudar Teologia}\label{c1_basicformatting:sec}

\emdash{}Guilherme de Ockham (1285-1349) disse: ``Somente a fé consegue esclarecer assuntos da revelação divina. O entendimento da revelação divina vai além da capacidade humana e portanto não é racionalmente compreensível". É muito profunda essa afirmação, sou Espírita e acredito numa fé raciocinada, mas mesmo o nosso raciocínio a respeito das coisas de Deus é por Ele guiado quando sinceramente pedimos Sua orientação, ou seja, quando nos entregamos para sermos guiados por Ele.

\emdash{}Muitas são as complexidades se formos ver os textos Bíblicos e a conjectura atual do mundo e a maneira de sermos fiéis ao Senhor é bebermos da fonte do Espírito Consolador que Jesus prometeu que enviaria para repetir tudo o que ele havia dito e ensinar todas as coisas. Não estaremos abandonados se vivermos em comunhão com Deus mas é verdade que isso exige de nós estarmos alertas o tempo todo e meditarmos bastante.

\emdash{}Segundo Elberfran Oliveira, ``Todo mistério tem sua revelação, a vida nos ensina que temos que ser paciente, tomar decisões por impulso é precipitar-se para uma derrota na batalha, comedimento com as palavras, moderação nas atitudes, cautela nos passos a serem dados, é o segredo para uma boa caminhada rumo às vitórias e conquistas da vida"! Se desejamos ser guiados por Deus, não podemos ser precipitados nas conclusões a respeito de Sua Vontade; entreguemo-nos a Ele inteiramente e consequentemente faremos Sua Vontade.

\emdash{}Feliz é aquele que faz a Vontade do Senhor, paz e bênçãos o seguirão. A humanidade será inteiramente feliz e saudável quando todos forem ovelhas do redil do Senhor. Então não haverá lobos nem dissensões entre os irmãos. Jesus disse ``Não pensem que vim abolir a Lei ou os Profetas; não vim abolir, mas cumprir. Digo-lhes a verdade: Enquanto existirem céus e terra, de forma alguma desaparecerá da Lei a menor letra ou o menor traço, até que tudo se cumpra. Todo aquele que desobedecer a um desses mandamentos, ainda que dos menores, e ensinar os outros a fazerem o mesmo, será chamado menor no Reino dos céus; mas todo aquele que praticar e ensinar estes mandamentos será chamado grande no Reino dos céus". Aqui precisamos fazer distinção de o que no Velho Testamento é Lei de Deus e o que foram leis particulares e momentâneas que por exemplo Moisés, precisou legislar por causa da dureza do coração dos homens da época e para controlar o povo contra suas iniquidades.

\emdash{}Havia por exemplo a pena de morte por espada para certa transgressão sexual, havia o apedrejamento até a morte para as mulheres adúlteras (para as mulheres, não para os homens) e outras coisas que hoje são vistas como barbaridades e selvageria pelas pessoas mais civilizadas e não combinam como determinações vindas do Deus de infinito amor que Jesus pregou, mas foram necessárias naquele tempo, naquele momento e muitas devem ter sido inspiradas pelo Alto para as pessoas santas da época mas hoje em dia com os costumes abrandados pela melhor compreensão de Deus, já não cabem mais. Essas leis de apedrejamento por exemplo desapareceram da sociedade civilizada e Jesus disse que nenhum traço da Lei desapareceria, logo o apedrejamento não é Lei de Deus e sim foi uma lei humana localmente colocada.

\emdash{}Cabe a cada um a reflexão a partir da Revelação. É um direito e uma responsabilidade da pessoa humana mas ela não está sozinha nessa tarefa pois sendo sincero de coração e procurando instruir-se e meditar, o próprio Deus nos ajuda através de Seu Espírito. Tenhamos ânimo e calma e sejamos pacientes conosco mesmo pois por mais que nos dediquemos a essa missão, nunca teremos toda a certeza pois a plenitude da Revelação pertence a Deus (Ele é perfeito). Somos apenas homens e mulheres, portanto limitados e jamais poderemos dizer que sabemos o suficiente para impor a outrem nossas conclusões a respeito de teologia. 

\emdash{}A liberdade de cada filho de Deus é fundamental: liberdade de crença, liberdade de coração. Há uma Igreja fundada por Cristo cujo corpo é composto de todos os cristãos unidos em comunhão, vivos e mortos. Essa Igreja é a que não erra, essa Igreja celeste da qual Cristo é a cabeça. Não são os movimentos religiosos da Terra, compostos por pensamentos humanos, que não erram e sim essa Igreja de comunhão com Jesus. Ainda que julguemos participar da Vida de Jesus, sejamos prudentes de aplicar nossas conclusões apenas a nós mesmos e deixar a consciência de nossos irmãos intacta. Isso não nos impede de pregar, de falar do Reino e de fazer toda manifestação religiosa santa, só o que não podemos é impô-las para que a Paz seja perfeita.