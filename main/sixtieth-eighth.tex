\begin{chapterpage}{Realidades Espirituais}{c68_sixtieth-eighthchapter:cha}
 
\begin{myquotation}Há mais coisas entre o céu e na terra, Horácio, do que sonha a nossa vã filosofia.
\par\vspace*{15mm}
\mbox{}\hfill \emdash{}William Shakespeare (1600)\index{Shakespeare, William}
, %\citetitle{bibitem}\index{@\citetitle{bibitem}} %\ifxetex\label{famousperson-bibitem-quote}\else\citep[p.~123]{bibitem}\fi
\par\end{myquotation}

\end{chapterpage}

% -------------------- replace or remove text below and paste your own text ------


\section{Nossos dons como características biológicas}\label{c1_basicformatting:sec}

\emdash{}Há muito tempo que a Biologia trata de assuntos da Vida com relevância singular. Desde que Darwin trouxe a teoria da Evolução, e diga-se de passagem ele era um homem religioso e não quis se contrapor às Sagradas Escrituras, que o impacto da ciência na vida das pessoas é presente. A própria medicina que alivia muitos dos nossos males deve à Biologia e também à Química a descoberta da penicilina (antibióticos).

\emdash{}A capacidade de perceber o mundo espiritual é uma característica biológica e todos, segundo Kardec, a tem em maior ou menor grau. Por uma alteração na glândula da hipófise, uma pessoa pode desenvolver uma sensibilidade de ver, ouvir e até se comunicar com pessoas que se dizem que já fizeram parte do nosso mundo.

\emdash{}Antigamente essas coisas eram vistas como ``sobrenatural" e essas pessoas do além, como fantasmas e seres à parte da criação às vezes. Mas se fizermos uma educação espiritual e encararmos esses fatos com naturalidade, estaremos dando mais um passo em direção a nossa evolução e felicidade.

\emdash{}Porque haveria de tudo acabar no túmulo? Porque haver apenas dois destinos: céu e inferno, ambos eternos, onde a justiça no caso de suposta condenação? O que fazer eternamente na beatitude sem as maravilhas do trabalho edificante? Se houvesse apenas uma vida, como explicar que alguns nascem com características que facilmente os levariam à salvação e outros cheios de imperfeições que dificilmente os levaria a entrar na eternidade em boa situação?

\emdash{}A capacidade de se comunicar com o mundo espiritual não é exclusiva de pessoas virtuosas, como podemos notar, justamente por ser uma condição biológica como a visão, o tato, o paladar. Essa oportunidade de interação deve servir a nos aproximar mais de Deus e praticarmos o Evangelho de Jesus, aceitando-o como Mestre Maior. A mediunidade é um mandato santo mas não é em si garantia de que a pessoa vá cumprir com suas obrigações diante do Pai. Por isso é cada vez mais importante o conselho de Jesus: ``Orai e vigiai".

\emdash{}Nossa responsabilidade é muito grande e deve ser vivida com tranquilidade por isso há a necessidade de estudar e se informar. Os fenômenos espirituais não vão desaparecer se os ignorarmos, portanto é preciso aprender a lidar com eles. Por muitos anos neguei minha espiritualidade e os resultados foram amargos. 

\emdash{}O negacionismo em relação às realidades espirituais faz parte do ser humano, está arraigado às instituições da sociedade mas devemos fazer acender uma luz em nossa alma para que, de olhos abertos, vejamos o que é claro e óbvio mas que não queremos enxergar. A pluralidade das existências, a infinita Sabedoria, Amor e Justiça de Deus, a maestria de Jesus são a vitória do Bem sobre o mal em nossas vidas.