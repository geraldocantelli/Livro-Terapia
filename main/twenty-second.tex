\begin{chapterpage}{A chave}{c22_twenty-secondchapter:cha}

\begin{myquotation}Não desista. Geralmente é a última chave no chaveiro que abre a porta.


 
\par\vspace*{15mm}
\mbox{}\hfill \emdash{}Paulo Coelho \index{Coelho, Paulo}
, %\citetitle{bibitem}\index{@\citetitle{bibitem}} %\ifxetex\label{famousperson-bibitem-quote}\else\citep[p.~123]{bibitem}\fi
\par\end{myquotation}

\end{chapterpage}

% -------------------- replace or remove text below and paste your own text ------


\section{Reconforto ao Coração}\label{c1_basicformatting:sec}

\emdash{}O mundo é mister em criar situações para nos embaraçar, verdadeiras armadilhas mentais e físicas para os filhos de Deus (que somos todos nós, a humanidade). Mas Jesus disse ``Tenham calma, eu venci o mundo." Ele venceu primeiro para que nós pudéssemos vencer a nosso turno pois nossa batalha está acontecendo aqui e agora pois estamos no caminho da evolução e não há como negar que apesar de Jesus já haver vencido o mundo, desafios ainda batem à nossa porta.

\emdash{}Para trazer paz à mente lembremos de buscar ajuda na força Divina e na Sua Luz que está dentro de cada um de nós. Nas páginas dos livros sagrados também estão as verdades sagradas mas Deus as colocou inclusive no interior das pessoas, na consciência. Desenvolvemo-la praticando o bem, evitando o mal, fazendo caridade, sendo justos e honestos, fiéis, sinceros, amorosos, compassivos, misericordiosos. Tudo isso está ao alcance de qualquer pessoa que deseje conectar-se com o Altíssimo.

\emdash{}Jesus também disse ``Destruí esse templo e eu o reconstruirei em três dias". Ele falava da destruição do templo de Jerusalém que era de pedra e depois do templo do seu corpo, onde ficaria para sempre o espaço Sagrado, isto é, Separado para Deus pois o conceito de sagrado é algo separado só para Deus. Assim Ele transferiu toda a sacralidade do templo de pedra para dentro do corpo do ser humano e todos nós passamos a conter a ``Arca da Aliança" e o espaço do ``Santo dos Santos" com todos os seus significados agora contidos em nós.

\emdash{}Resumindo, o templo agora somos nós, por isso Jesus veio para destruir o pecado que era acreditarmos estar a parte de Deus e agindo desse modo. A partir do momento que tomamos consciência que a centelha Divina brilha em nós e passamos a agir de acordo com Sua Vontade, guiados pela Sua Sabedoria e Luz, está agindo a obra que Jesus veio realizar em nossa Vida. 

\emdash{}E quantas voltas nós damos nessa vida até chegar a esse ponto! Às vezes nos perdemos e nos reencontramos várias vezes pois a própria busca da iluminação é um caminho árduo; na Bíblia simbolizado pelo deserto em que o povo passa quarenta anos antes de chegar à terra prometida. O próprio Cristo foi tentado no deserto; esse é um período de seca espiritual mas precisamos manter a fé em Deus e nos alimentar de Sua Sabedoria dentro de nós (o que inclui buscar fontes como nas nossas religiões para ativá-la).

\emdash{}A recomendação de Paulo Coelho no início deste capítulo é bastante válida pois o nosso deserto pode ser demorado e podemos precisar tentar várias chaves para sair do problema mas nunca desistir é um conselho sábio. Podemos tentar meditação, sessões de terapia em psicólogos(as), ajuda espiritual em nossa religião ou outra ajuda que seja adequada e a esperança sendo alimentada como uma planta que precisa de um tanto de água certo e luz do sol certo e precisa ser colocado adubo e muito bem cuidada, pois nesse ponto a esperança é muito importante.

\emdash{}Nosso lado humano precisa ajudar nosso lado espiritual, mas também precisa aceitar a ajuda do lado espiritual. Nós não somos só lado humano nem só lado espiritual: somos ambos. E juntos com Deus formamos uma Unidade, ou melhor, formaremos quando tivermos completado nossa evolução, por enquanto estamos a caminho e não há pressa pois a pressa é inimiga da perfeição e não devemos nos cobrar demais por causa de nosso atual estado de evolução.

\emdash{}Cada passo que dermos demos no amor para Deus e já estaremos no Caminho da Vida, vivendo na Verdade. É o casamento de Jesus com a humanidade, tão aguardado através dos séculos. O lado espiritual deve dar lugar ao Espírito Santo ou Espírito da Verdade, que Jesus prometeu que enviaria, e o nosso lado humano deve se realizar com Deus através Dele.

\emdash{}A fé não pode e não deve ser imposta. O ideal é que seja raciocinada, que se possa fazer perguntas e ser cientificamente provada pela razão para que não haja dúvidas e isso não traga o negativismo e então o ceticismo. O Espírito da Verdade que Jesus prometeu e enviou promove essa fé raciocinada, onde a razão permite perguntar, comparar e estudar o mundo espiritual de maneira que ele não esteja mais no plano do maravilhoso, fantástico mas do real.

\emdash{}Precisamos parar de ver as coisas espirituais como coisas que não devem ser faladas ou tocadas porque na verdade elas elas fazem parte da Vida tanto quanto as coisas relativas aos cinco sentidos da matéria. Jesus mostrou que a morte não existe então paremos de nos comportar como se ela existisse realmente, o que morre, ou melhor se transforma, é apenas o corpo físico.

\emdash{}A continuidade da Vida, inclusive mantendo a individualidade da personalidade está mais do que provada nas em diversos artigos de fé que são sim conhecimento com construção de argumentos e lógica inegáveis, apesar de os materialistas desejarem negar sem conseguirem provar suas teorias. Há cientistas que afirmam sobre a continuidade da energia além da matéria do corpo, por exemplo.

\emdash{}Termos como ``fantasmas" não devem assustar mais as pessoas que não precisam ver mais no além um lugar tenebroso e assustador. O além na verdade coexiste com o aqui, apenas não o vemos pois são planos existenciais diferentes: o material e o espiritual. Estamos todos juntos o tempo todo e isso é muito naturalmente comprovado por fatos narrados em várias partes do mundo.

\emdash{}É preciso manter mente e corpo sãos. A importância de buscar o Equilíbrio. Não se assustar com a realidade pois é apenas a realidade, nada além dela, nada de fantástico ou mágico. Deus está no comando de tudo e nem uma folha cai de uma árvore sem a permissão Dele, o que equivale a dizer que nada acontece sem que Ele permita e Ele não permitirá que nada nos aconteça que não esteja nos planos Dele.

\emdash{}Quando um acontecimento funesto nos acontece, não quer dizer que Deus concorda com ele mas o permitiu porque vamos poder crescer com esse acontecimento, evoluir ou estamos resgatando um débito do passado (no primeiro caso prova ou no segundo, expiação). Se uma coisa for somente má e sem proveito para nós, Deus automaticamente a afasta do nosso caminho e para tudo isso ele o faz através de suas Leis Divinas que regem tudo o que acontece no Universo, como por exemplo a Lei de Ação e Reação e a de Progresso.