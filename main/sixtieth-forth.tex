\begin{chapterpage}{A Lei de Deus}{c64_sixtieth-forthchapter:cha}
 
\begin{myquotation} Não penseis que eu tenha vindo destruir a lei ou os profetas: não os vim destruir, mas cumpri-los: – porquanto, em verdade vos digo que o céu e a Terra não passarão, sem que tudo o que se acha na lei esteja perfeitamente cumprido, enquanto reste um único iota e um único ponto. 
\par\vspace*{15mm}
\mbox{}\hfill \emdash{}S. Mateus, 5:17 e 18\index{S. Mateus, 5:17 e 18}
, %\citetitle{bibitem}\index{@\citetitle{bibitem}} %\ifxetex\label{famousperson-bibitem-quote}\else\citep[p.~123]{bibitem}\fi
\par\end{myquotation}

\end{chapterpage}

% -------------------- replace or remove text below and paste your own text ------


\section{A Lei de Deus está escrita em nossa consciência}\label{c1_basicformatting:sec}

\emdash{}Jesus disse: "Eu não vim destruir a Lei, mas dar-Lhe cumprimento". Logo se Jesus precisou dizer isso, é porque as verdades que ele veio trazer, essas realidades eram contrárias a várias leis dos judeus, que eles consideravam leis de Deus mas que eram muitas mais do que os 10 mandamentos, eles tinham mais de 600 leis segundo seu rito e credo, inclusive a que proibia as manifestações mediúnicas. Mas Jesus veio falar da verdadeira Lei de Deus, ele também disse: se vossa justiça não for maior que a dos fariseus e saduceus, não entrareis no Reino dos Céus, ou seja, temos que ser justos e honestos moralmente. Os judeus tinham uma lei que dizia que devia-se lavar as mãos antes de comer e Jesus disse: "não é o exterior que torna o homem impuro e sim o interior", é como se estivesse contra essa lei mas na verdade estava mostrando um valor espiritual acima de uma convenção social.  

E onde está a Lei? Está na consciência das pessoas, escrita em suas mentes. Ninguém pode alegar desconhecimento da Lei de Deus pois Jesus fez do nosso corpo templo do Espírito Santo e onde estamos está também Deus conosco. Por isso devemos ter um proceder reto diante desse Deus de Misericórdia, ou seja, diante da Vida. 

Várias vezes as palavras de Jesus se contrapunham à lei dos judeus mas isso é porque nem tudo que está na lei dos judeus é a Lei de Deus e é essa Lei que Jesus veio dar cumprimento que também estava contida na lei da época, essa Lei maior que ele colocou como o Amor: "amarás a teu Deus de todo o coração e de todo entendimento e ao próximo como a si mesmo" essa, disse Ele, é toda a Lei e os profetas. 

Naquela época, as pessoas abusavam de seus dons mediúnicos e só os usavam para abusos, fazer o mal ou ficar na dependência exclusiva dos espíritos, não havia uma doutrina santa sobre o uso da mediunidade, que fosse baseada nos valores cristão do Evangelho e que levasse o homem para Deus. Isso só veio acontecer com Kardec, cumprindo-se o que Jesus havia dito: ``não vos deixarei sós mas enviarei o Espírito Consolador que repetirá tudo o que vos tenho dito e dirá todas as coisas". Tudo o que está escrito na lei deve se cumprir, o que foi dito pelos profetas, mas nem toda lei por exemplo de Moisés se aplica nos dias de hoje por causa da vinda de Jesus e do Espírito Consolador. Moisés foi a primeira revelação, Jesus a segunda e o Espiritismo a terceira.