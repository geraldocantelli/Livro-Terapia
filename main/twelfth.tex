\begin{chapterpage}{Dentro de nós}{c12_twelfthchapter:cha}

\begin{myquotation} Sendo Jesus interrogado pelos fariseus sobre quando viria o reino de Deus, respondeu-lhes: O reino de Deus não vem com aparência exterior; nem dirão: Ei-lo aqui! ou Ei-lo ali! pois o reino de Deus está dentro de vós.  

\par\vspace*{15mm}
\mbox{}\hfill \emdash{}Jesus, Lucas 17:20,21 \index{Jesus, Lucas 17:20,21}
, %\citetitle{bibitem}\index{@\citetitle{bibitem}} %\ifxetex\label{famousperson-bibitem-quote}\else\citep[p.~123]{bibitem}\fi
\par\end{myquotation}

\end{chapterpage}

% -------------------- replace or remove text below and paste your own text ------


\section{O Todo está em Tudo}\label{c1_basicformatting:sec}

\emdash{}Quantos desafios já enfrentamos nessa vida? Desde o primeiro, o nascimento que dizem os estudiosos não é nada fácil, até cada contratempo que tivemos na família, na escola, no trabalho, na sociedade em geral, e passamos por tudo isso e estamos aqui. Procure se lembrar no geral de como foram dadas soluções para grande parte das situações, o que possibilitou que você chegasse até agora e lesse essas páginas. Pode não ter sido a solução idealizada na maioria das vezes mas note que uma Força Maior esteve com você sempre e repito em todos os momentos da sua vida como que conduzindo para coisas muito piores não acontecessem e que você saísse mais forte de cada obstáculo.

\emdash{}Afinal de contas, os obstáculos vêm para nos burilar e para ficarmos mais fortes, mais bondosos, mais misericordiosos e sabermos partilhar mais. Quando tivermos aprendido todas as lições que há para aprender, não haverá mais necessidade nem de estarmos nessa realidade mais (nem de corpo físico e mente). Estamos numa jornada como que numa jangada atravessando um lago. Quando chegarmos à outra margem, que é o objetivo, poderemos andar em terra firme e não precisaremos mais da jangada, na verdade em terra firme arrastar a jangada até atrapalharia. Assim é nossa evolução e nosso estado atual: precisamos de corpo e mente para aprendermos tudo que é necessário e quando, daqui a talvez milhares de anos, esse ciclo tiver se completado é possível que vivamos intensamente em outra realidade em direta comunhão com Deus.

\emdash{}Mas esse Reino em que viveremos já pode existir dentro de nós, como disse Jesus: "O Reino de Deus está dentro de vós". Todos os ensinamentos Dele são sobre Amor e nos levam a exteriorizar atos de amor logo esse reino só pode ser de Amor. Religião vem do latim \textit{religare} que significa religar, religar o mundo exterior que atualmente está tão deturpado com esse mundo de amor interior de que a humanidade se esqueceu desde o que se chamou de pecado original. Por isso a importância de uma espiritualidade sadia e não somente religiosidade simplesmente pois nossos atos devem refletir essa realidade de Amor interior para isso tomando consciência de que viemos Dele e para Ele voltaremos, é voltar-se para dentro de si, ser senhor de si, autoconhecimento para conhecer melhor a Deus. 


\emdash{}A história narrada na Bíblia fala de realidades exteriores que refletem o interior do ser humano. Deus nos conhece muito bem e nós pertencemos a Ele, está dentro de nós. O fato de ter nos criado mostra que não é nem um pouco egoísta pois quis compartilhar o dom da Vida pois viu que é um dom maravilhoso, tanto para Ele quanto para nós. Por isso pôs Seu Reino dentro de nós para que ao nos encontrarmos exteriorizássemos a realidade verdadeira em que a Vida é linda, abundante como disse Jesus. Conhece-te a ti mesmo vem de encontro à realização do plano de Deus portanto.

\emdash{}O Aqui e Agora é onde esse plano de Deus pode se manifestar, cada segundo é tão valioso que não se repete nunca mais em toda a Eternidade. Devemos valorizar a Presença Dele em nossa Vida desde os primeiros momentos em que nos reconhecemos em apuros e fomos salvos até os momentos tranquilos onde gozamos de Sua Paz. Gratidão é um sentimento libertador: gratidão aos irmãos, aos problemas, às doenças, aos professores, aos pais, a tudo que vem para nos fazer enxergar nossa história que assim como a história bíblica pode ser usada para aprender sobre uma realidade superior e interior onde a Vida pulsa e nos chama faz tempo e Ela nos aguarda há muito tempo. Quando essa vida interior flui harmoniosa e organizada, a vida exterior será um reflexo da mesma, isso os orientais já sabiam de pronto.

\emdash{}Paz e bem. Deus abençoe a todos.