\begin{chapterpage}{O Princípio da Sabedoria}{c45_fortieth-fifthchapter:cha}
 
\begin{myquotation}O temor do Senhor é o princípio da sabedoria; todos os que cumprem os seus preceitos revelam bom senso. Ele será louvado para sempre!
\par\vspace*{15mm}
\mbox{}\hfill \emdash{}Salmos 111:10\index{Salmos 111:10}
, %\citetitle{bibitem}\index{@\citetitle{bibitem}} %\ifxetex\label{famousperson-bibitem-quote}\else\citep[p.~123]{bibitem}\fi
\par\end{myquotation}

\end{chapterpage}

% -------------------- replace or remove text below and paste your own text ------


\section{O temor do Senhor é o princípio da Sabedoria}\label{c1_basicformatting:sec}

\emdash{}Em primeiro lugar, é preciso saber distinguir o salutar temor do Senhor do patogênico medo de Deus que acabou gerando mais incrédulos e depressivos do que crentes desde a Idade Média, principalmente. O temor de Deus consiste em conhecermos Sua Onipotência, Onisciência e Onipresença e que Ele tem a última palavra em todos os assuntos do Universo, independente do que possa parecer no mundo aparente, pois o que vemos é só uma página de um livro que está em uma vasta biblioteca de livros, que só lendo essa biblioteca toda poderíamos conhecer os motivos verdadeiros de porque as coisas acontecem como acontecem.

\emdash{}Nós somos criaturas imortais mas não eternas, pois como espíritos não pereceremos mas tivemos um dia de criação portanto não existimos desde sempre. Já o Criador é eterno pois existe desde sempre e não terá fim. Essas colocações são um paradoxo para aquela nossa afirmação terrena de que ``tudo o que tem começo tem um fim", contudo essa afirmação pode estar baseada apenas em coisas materiais. Seria um triste pensamento considerar o túmulo como o fim da experiência humana, possivelmente isto geraria uma falta de sentido existencial, ou seja, um vazio na alma que a pessoa procuraria preencher com inúmeras experiências e até coisas materiais ou mesmo pessoas em busca de compensação.

\emdash{}Mas esse vazio é como um saco sem fundo e o que pode preenchê-lo é Deus, ou seja, a experiência com o numinoso que uma pessoa tem dentro de si (termo que Jung trouxe à luz na psicologia). Até porque os relacionamentos com as pessoas e até mesmo coisas e experiências não devem ser compensatórios e sim integrais, ou seja, precisamos ser completos em nós mesmos, senhores de si para estar inteiros e nos dar nos relacionamentos e não usar as pessoas e as coisas como ``muletas" para nossa falta de sentido na vida. Inclusive quando estamos usando as pessoas assim como objetos, na menor contrariedade que elas nos apresentam, somos agressivos e podemos até ser violentos pois não estamos sendo humanos para com elas, nem as considerando como pessoas humanas dignas de direitos fundamentais.

\emdash{}Aí entra o temor a Deus como princípio da Sabedoria, sabermos que Ele, que é só Amor, criou suas Leis Divinas para que o Universo fosse plenamente governado à Sua Vontade dentro de Harmonia e Retidão. Uma dessas Leis é a de Ação e Reação, que diz que responderemos por nossos atos praticados e tudo que plantarmos, colheremos. É o famoso: o que damos ao Universo, ele nos retribui, de outra forma explicado. Jesus também explicou isso quando disse que a pessoa ficaria na prisão até que pagasse o último centavo da sua dívida (esse trecho do Evangelho é pouco divulgado atualmente).

\emdash{}Deus é Amor mas é Justiça também. Ele só sabe fazer bondade, mas criou leis para que o comportamento das pessoas não corrompesse a criação e desvirtuasse a boa vontade e moral das boas pessoas. É preciso haver, e há, recompensas aos bons e medidas pedagógicas para corrigir os maus atos. A virtude estimulada e praticada como base da sociedade de Paz e Amor. 

\emdash{}E esse conhecimento das coisas de Deus também passa pela pessoa conhecer a si mesma para que se torne uma pessoa melhor para si e para os irmãos e irmãs. Há um conselho antigo, conhecido desde a Grécia: ``Conhece-te a ti mesmo". Às vezes a pessoa se convence que tem que fazer a vontade de Deus mas esquece de melhorar-se como pessoa, que é um dever de todos. E só melhora a si mesmo quem se conhece.

\emdash{}A pessoa que não se conhece pode ter traços ditatoriais e isso unido aos conteúdos da Bíblia pode virar fanatismo e se a pessoa chegar a um posto de pregador da palavra pode ser um desastre pois as coisas que serão ditas serão de um Deus terrível e não de um Deus de Amor. Aí geraria nos ouvintes medo doentio de Deus, que não é nada saudável e que afasta as pessoas de Deus. Esses pregadores precisariam de terapia com psicólogos para curarem-se e então voltar para pregar um Deus de amor como sempre deveria ter sido.