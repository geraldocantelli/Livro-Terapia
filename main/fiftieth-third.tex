\begin{chapterpage}{Iluminismo}{c53_fiftieth-thirdchapter:cha}
 
\begin{myquotation}A razão forma o homem, o sentimento o conduz.
\par\vspace*{15mm}
\mbox{}\hfill \emdash{}Jean-Jacques Rousseau\index{Rousseau, Jean-Jacques}
, %\citetitle{bibitem}\index{@\citetitle{bibitem}} %\ifxetex\label{famousperson-bibitem-quote}\else\citep[p.~123]{bibitem}\fi
\par\end{myquotation}

\end{chapterpage}

% -------------------- replace or remove text below and paste your own text ------


\section{Razão e Emoção}\label{c1_basicformatting:sec}

\emdash{}O filósofo Immanuel Kant deixou algumas pérolas fruto da sabedoria que aqui compartilho com vocês a respeito da Vida, para podermos refletir:

\emdash{}Podemos julgar o coração de um homem pela forma como ele trata os animais.

\emdash{}Não somos ricos pelo que temos, e sim pelo que não precisamos ter.

\emdash{}O homem não é nada além daquilo que a educação faz dele.

\emdash{}Age sempre de tal modo que o teu comportamento possa vir a ser princípio de uma lei universal.

\emdash{}A moral, propriamente dita, não é a doutrina que nos ensina como sermos felizes, mas como devemos tornar-nos dignos da felicidade.

\emdash{}Quem não sabe o que busca, não identifica o que acha.

\emdash{}Toda reforma interior e toda mudança para melhor dependem exclusivamente da aplicação do nosso próprio esforço.

\emdash{}Ciência é conhecimento organizado. Sabedoria é vida organizada.

\emdash{}É no problema da educação que assenta o grande segredo do aperfeiçoamento da humanidade.

\emdash{}Quanto mais amor temos, tanto mais fácil fazemos a nossa passagem pelo mundo.

\emdash{}O sábio pode mudar de opinião. O ignorante, nunca.

\emdash{}Não se ensina filosofia; ensina-se a filosofar.

\emdash{}No reino dos fins, tudo tem ou um preço ou uma dignidade. Quando uma coisa tem preço, pode ser substituída por algo equivalente; por outro lado, a coisa que se acha acima de todo preço, e por isso não admite qualquer equivalência, compreende uma dignidade.

\emdash{}Somos todos iguais perante o dever moral.

\emdash{}Todo o conhecimento humano começou com intuições, passou daí aos conceitos e terminou com ideias.

\emdash{}Duas coisas me enchem a alma de crescente admiração e respeito, quanto mais intensa e frequentemente o pensamento delas se ocupa: o céu estrelado sobre mim e a lei moral dentro de mim.

\emdash{}A inumanidade que se causa a um outro, destrói a humanidade em mim.

\emdash{}É absolutamente necessário persuadir-se da existência de Deus; mas não é necessário demonstrar que Deus existe.

\emdash{}Só a crítica pode cortar pela raiz o materialismo, o fatalismo, o ateísmo, a incredulidade dos espíritos fortes, o fanatismo e a superstição, que se podem tornar nocivos a todos e, por último, também o idealismo e o cepticismo, que são sobretudo perigosos para as escolas e dificilmente se propagam no público.

\emdash{}Se vale a pena viver e se a morte faz parte da vida, então, morrer também vale a pena...

\emdash{}Avalia-se a inteligência de um indivíduo pela quantidade de incertezas que ele é capaz de suportar.

\emdash{}É minha fé na Bíblia que me serviu de guia em minha vida moral e literária. Quanto mais a civilização avance, mais será empregada a Bíblia.

\emdash{}Ages de tal maneira que uses a humanidade, tanto na tua pessoa como na pessoa de qualquer outro, sempre e simultaneamente como fim e nunca simplesmente como meio.

\emdash{}O mesmo acontece ao mérito e à inocência: perde-se, desde que deles nos sustentemos.

\emdash{}Pensamentos sem conteúdos são vazios; intuições sem conceitos são cegas.

\emdash{}Só há uma religião verdadeira, mas podem haver muitas espécies de fé.

\emdash{}Citações tiradas do site https://www.frasesfamosas.com.br/frases-de/immanuel-kant/ em 16/06/2020.