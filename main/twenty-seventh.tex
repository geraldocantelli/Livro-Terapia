\begin{chapterpage}{Sentimentos}{c27_twenty-seventhchapter:cha}

\begin{myquotation}O primeiro sentimento que surgiu na humanidade foi o medo, então veio a ira e depois desenvolveu-se o amor. Portanto o amor é o sentimento mais recente dentre estes. 

\par\vspace*{15mm}
\mbox{}\hfill \emdash{}Divaldo Franco \index{Franco, Divaldo}
, %\citetitle{bibitem}\index{@\citetitle{bibitem}} %\ifxetex\label{famousperson-bibitem-quote}\else\citep[p.~123]{bibitem}\fi
\par\end{myquotation}

\end{chapterpage}

% -------------------- replace or remove text below and paste your own text ------


\section{Ego ``Eu estou" e Self ``Eu Sou"}\label{c1_basicformatting:sec}

\emdash{}São interessantes as informações que a antropologia nos traz sobre o nosso desenvolvimento, ou seja, o desenvolvimento da humanidade, nos permitindo uma visão mais ampla a cerca da sociedade e potencializando nossa tomada de decisão para sermos melhores pessoas, seja para nós mesmos, seja para os outros e para Deus.

\emdash{}Não sendo eu especialista no assunto, peço perdão por eventuais imprecisões, apresentarei o que se estuda no geral: quando o \textit{homo sapiens} tomou consciência de que sua comunidade ou sua ``morada" estava exposta a ser atacada por outros seres humanos de então ou por predadores, conheceu o medo e começou a pensar em formas de evitar que isso acontecesse. Então sentiu a ira pelos potenciais invasores e predadores e armou-se para a defesa imediata.

\emdash{}A infância da humanidade consistiu-se de proteger-se das intempéries, da busca por alimentos de maneira extrativa e depois pelo desenvolvimento da agricultura, e pela defesa contra inimigos. E tudo isso está até hoje em nosso inconsciente, medo e ira estão lá, assim como a capacidade de desenvolver-se e criar soluções, de aprimorar-se.

\emdash{}E felizmente, principalmente graças, à pregação de Jesus e de outros iluminados, surgiu o Amor (que também está em nosso inconsciente). Jesus coloca que o amor a Deus sobre todas as coisas e ao próximo como a si mesmo significa toda a Lei e os profetas portanto feliz é aquele que pauta suas atitudes pelo Amor.

\emdash{}Carl Gustav Jung fala dos arquétipos que são aqueles conceitos primordiais que estão na mente de todos. Já falei da Sombra em um dos capítulos iniciais, agora falarei do Ego e do Self. O Self é nossa parte que nos liga com o divino, é o ``Eu Sou", o que me lembra que na Bíblia quando Moisés perguntou à Entidade que estava na sarça ardente qual o seu nome? Deus disse ``Eu sou o Eu sou". Já o Ego é o mordomo da mansão da mente, enquanto o Self é o dono dessa mansão. Mas o Ego, que tem sua importância com certeza, é quem executa as ações práticas mas quando ele está imaturo ``pensa" e age como se fosse o dono.

\emdash{}É preciso haver um equilíbrio entre Ego, Self, Sombra e os outros arquétipos e o Amor parece ser a chave para aprendermos a equilibrar essa balança ao que tudo indica e a palavra de Jesus. Digo isso pois, vejo Jesus não apenas como um líder religioso mas uma pessoa que promove ``a vida e a vida em abundância" e que utilidade tem a Lei se não for para o nosso bem e promover a vida? Portanto se o Amor é toda a Lei e os profetas, então o Amor é a chave para o nosso equilíbrio interno, externo, espiritual, psíquico e material.

\emdash{}Para a humanidade atingir sua maioridade é preciso que aprenda a amar por isso é importante saber diferenciar religiosidade de espiritualidade. Toda religião que ensina o amor dá a oportunidade de a pessoa viver uma espiritualidade saudável mas isso vai depender da própria pessoa pois não basta estar fisicamente em uma igreja e não viver a espiritualidade pregada por tal igreja. No caso dos cristãos existe o conceito da Salvação, já para os budistas tem-se a Iluminação e poder-se-ia citar vários outros exemplos mas é o Amor que integra o ser humano com sigo mesmo e com Deus: tem-se então a Unidade refeita, religada, que vem do latim \textit{religare} que deu origem à palavra Religião. Se não se consegue religar a Unidade com Deus vivendo a espiritualidade sadia, então a religião não atingiu seu objetivo e para tal a chave é o Amor.