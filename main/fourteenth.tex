\begin{chapterpage}{O Espírito Vivifica}{c14_fourteenthchapter:cha}

\begin{myquotation}E Eu rogarei ao Pai, e Ele vos dará outro Advogado, a fim de que esteja para sempre convosco, o Espírito da verdade, que o mundo não pode receber, porque não o vê, nem o conhece; vós o conheceis, porque Ele vive convosco e estará dentro de vós.
 
\par\vspace*{15mm}
\mbox{}\hfill \emdash{}Jesus, João 14:16-17 \index{Jesus, João 14:16-17}
, %\citetitle{bibitem}\index{@\citetitle{bibitem}} %\ifxetex\label{famousperson-bibitem-quote}\else\citep[p.~123]{bibitem}\fi
\par\end{myquotation}

\end{chapterpage}

% -------------------- replace or remove text below and paste your own text ------


\section{O Espírito da Verdade Sopra dentro de Nós}\label{c1_basicformatting:sec}

\emdash{}Quando Jesus estava fisicamente no mundo os discípulos não temiam nada pois depositavam Nele toda a fé e confiança inclusive para os assuntos temporais; tinham a certeza de que nenhum mal aconteceria àqueles que o seguissem pois Ele curava a todos e demonstrava um poder divino. Jesus chegou a afirmar: ``Enquanto estou no mundo, sou a luz do mundo" (João 9:5). Qual não foi a surpresa e o medo que se apoderaram dos apóstolos quando Jesus se entregou aos soldados para ser preso, flagelado e morto.

\emdash{}Tudo melhorou com a Ressurreição e a esperança voltou a seus corações, apesar de Jesus ter anunciado todos esses fatos com antecedência. Mas novamente Jesus não ficaria fisicamente (em corpo) com a humanidade para sempre e pensando nisso Ele disse que não nos deixaria órfãos e pediria ao Pai e Ele nos enviaria o Espírito de verdade, que o mundo não conhece mas que nós conhecemos e que vive em nós.

\emdash{}Esse Espírito sopra as verdades de Deus para nós em nosso interior. Se silenciarmos poderemos ouví-lo. É importante conhecer a palavra de Deus para dar subsídios para nosso entendimento entender as intuições desse Santo Espírito. Como uma voz interior que guia a humanidade para tudo de bom que faz, como já comentado anteriormente: ``Quem olha para fora sonha, quem olha para dentro desperta!"

\emdash{}A Luz do mundo agora está dentro de cada um mas se não olharmos para ela e mergulharmos profundamente dentro de nós, se ficarmos apenas na superfície e ignorando tesouros preciosos de nossa alma, a mudança edificante não ocorre. Buscar o auto-conhecimento é um passo importante: meditação, reflexão, espiritualidade saudável. E aí entram os conselhos do Evangelho de Jesus como o principal: Amar a Deus sobre todas as coisas e ao próximo como a si mesmo, pois se tivermos apenas religiosidade mas não cumprirmos esse maior mandamento, seremos apenas como um sino que retine como disse São Paulo.

\emdash{}Há também aquela meditação que consiste em concentrar-se na própria respiração, nos sons da natureza como forma de atenção plena que pode ajudar para entrarmos em contato com nossa Vida interior e com toda a Sabedoria da Vida que está dentro de cada um de nós. Deus colocou toda Sua Lei em nossa consciência por isso não podemos nos eximir de fazer o bem por pretensa ignorância, mas colocou também e coloca a Força, a Paz, a Sabedoria para vivermos a Vida. Quando não encontramos essas maravilhas é porque não procuramos ainda dentro de nós, lembrem-se que o Temor do Senhor é o princípio da Sabedoria e que quando buscamos a Deus passamos a procurar ter mais contato com nossa espiritualidade, que está dentro de nós.

\emdash{}Outra técnica interessante é escrever sobre seus pensamentos, reflexões, sentimentos, acontecimentos em uma espécie de diário, seja no papel ou no celular, computador, aonde for. É conhecida como ``journaling" ou escrita terapêutica. São formas de buscar a si e encontrar-se  com a pessoa da qual você mais precisa conviver bem: você mesmo(a). Amar-se é fundamental para viver bem, mas como amar uma pessoa que mal se conhece, que não admira? As pessoas em geral pensam que se conhecem mas na verdade só é possível conhecer a si quem olha para dentro conscientemente e ``por querer".

\emdash{}O mais longo relacionamento que teremos na vida é conosco mesmo então tem que fazer dar certo esse contato, buscando entender as reações do corpo, dos sentimentos, o porquê de elas terem acontecido e mesmo se não forem positivas, perguntar-se o que elas estão querendo dizer? O corpo fala pois mente e corpo estão intimamente ligados (um manifesta no outro). Se estamos tristes, buscarmos as causas dessa tristeza e também as soluções para ficarmos felizes, racionalmente, sim, mas também com o coração.

\emdash{}Muitas pessoas não se amam e não se conhecem o suficiente para isso. Daí a importância de meditar, de escrever e de continuar tentando, pois parte muito importante do caminho é não parar nem estacionar. Continuar na caminhada sempre buscando a Deus da forma como você o compreende mas sempre buscando melhorar pois há uma enxurrada de soluções esperando por quem pede ao Senhor. Às vezes quando pedimos essas soluções, elas se apresentam durante a caminhada portanto comece da maneira que for possível, com os recursos de que dispõe atualmente e fazendo o que pode fazer agora. Não se preocupe com o ``depois", viva o Agora.

\emdash{}Não sabemos mas quando nos colocamos no caminho da Vida, é o próprio Espírito Consolador quem nos conduz. Lembremo-nos de que nunca estivemos sozinhos, Deus está conosco. Às vezes fazemos coisas nesse Caminho que não entendemos porque fizemos mas o Senhor sabe todas as coisas. Muitas pessoas conseguem esse suporte de Vida através da fé que praticam em suas religiões. Deus não desampara ninguém e é exatamente isso que está sendo falado quando se diz que ``Deus ajuda quem se ajuda". O caminho se faz caminhando e buscar conhecer-se é uma forma de abrir as torneiras do Céu para as bênçãos jorrarem.

\emdash{}Outras pessoas também podem nos ajudar mas quem está conosco o tempo todo somos nós mesmos. Devemos aceitar toda ajuda que seja benéfica mas também buscar o fortalecimento interior, que nos ajudará em todos os sentidos. a Fé nasce, brota no nosso interior e se fortalece conosco diante das experiências diárias que são como lições pedagógicas na escola da Vida por isso tudo é para nosso bem, se pensarmos que podemos aproveitar os acontecimentos para aprimorarmo-nos e ficamos bem melhor a cada dia.

\emdash{}Quando é que enxergamos esse processo terapêutico da Vida? Quando olhamos para dentro de nós mesmos e começamos a comparar acontecimentos com reações, com soluções, com sentimentos, com o conhecimento que temos das coisas de Deus, e submetemos tudo a uma análise ou seja uma auto-análise. Vemos então a riqueza da Vida e dos recursos vastos de que dispomos para bem viver, tudo sob a tutela do Espírito Consolador. ``A Vida é pela Vida", desde que nos colocamos no caminho de Deus, da Vida e continuamos em frente mesmo diante de obstáculos, tudo tende à solução, a melhorar.

\emdash{}O poder de Deus é imenso. Ele tudo pode. Quando colocamos nossas forças à serviço das forças da Vida, todo tipo de milagres acontecem em nossa Vida. Desde as curas físicas e psicológicas até a mudança em nosso comportamento para nos tornarmos pessoas mais éticas e morais, amando, abençoando, trabalhando para o bem de todos que conhecemos e através de uma das mais belas faces da Vida: a Caridade.

\emdash{}Ajudar aos outros é outra forma de ajudar-nos a nós mesmos. Ajudando as pessoas também trabalhamos conteúdos que nos são necessários ao aprimoramento próprio. Afinal se vivêssemos como eremitas em grutas não teríamos a dimensão social que é importantíssima. Sermos melhores para conosco para sermos melhores para com os outros e vice-versa, pois todos nós estamos interligados e fazemos parte do tecido da Vida, o caminho de um pode levar Luz ao caminho de outro, daí um dos sentidos de amar ao próximo como a si mesmo.

\emdash{}O pão que nos alimenta o corpo é necessário mas também o pão que nos alimenta a alma. Vida exterior e Vida interior. Consciência, Perdão, Inclusão, Razão, Coração, Vida e Deus são tudo isso e muito mais mas só concluiremos isso ao fazer o Caminho de olhos abertos, prestando atenção nessas maravilhas e buscando amar como Jesus pediu. Quando não olhamos para dentro de nós, é como se fizéssemos o Caminho de olhos fechados, vendo apenas as coisas do mundo que são efêmeras e se desfazem, mas não víssemos a imensa riqueza que Deus colocou dentro de todos nós, e que duram para sempre, são inesgotáveis e dão sentido a Tudo.

\emdash{}Jesus disse sobre esse Advogado, Espírito de verdade: ``... porque Ele vive convosco e estará dentro de vós". A pista é essa: dentro de vós, olhar para dentro para despertar, quem conversa consigo mesmo não é louco, é sábio nesse sentido. Encontrar-se para encontrar a Deus. Mesmo que no início do caminho nos percamos pois às vezes é preciso perder-se para se encontrar como aconteceu na parábola do filho pródigo. Dentro de nós, não fora.

\emdash{}Paz e bem. Deus abençoe a todos.