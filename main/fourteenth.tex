\begin{chapterpage}{O Espírito Vivifica}{c14_fourteenthchapter:cha}

\begin{myquotation}E Eu rogarei ao Pai, e Ele vos dará outro Advogado, a fim de que esteja para sempre convosco, o Espírito da verdade, que o mundo não pode receber, porque não o vê, nem o conhece; vós o conheceis, porque Ele vive convosco e estará dentro de vós.
 
\par\vspace*{15mm}
\mbox{}\hfill \emdash{}Jesus, João 14:16-17 \index{Jesus, João 14:16-17}
, %\citetitle{bibitem}\index{@\citetitle{bibitem}} %\ifxetex\label{famousperson-bibitem-quote}\else\citep[p.~123]{bibitem}\fi
\par\end{myquotation}

\end{chapterpage}

% -------------------- replace or remove text below and paste your own text ------


\section{O Espírito da Verdade Sopra dentro de Nós}\label{c1_basicformatting:sec}

\emdash{}Quando Jesus estava fisicamente no mundo os discípulos não temiam nada pois depositavam Nele toda a fé e confiança inclusive para os assuntos temporais; tinham a certeza de que nenhum mal aconteceria àqueles que o seguissem pois Ele curava a todos e demonstrava um poder divino. Jesus chegou a afirmar: ``Enquanto estou no mundo, sou a luz do mundo" (João 9:5). Qual não foi a surpresa e o medo que se apoderaram dos apóstolos quando Jesus se entregou aos soldados para ser preso, flagelado e morto.

\emdash{}Tudo melhorou com a Ressurreição e a esperança voltou a seus corações, apesar de Jesus ter anunciado todos esses fatos com antecedência. Mas novamente Jesus não ficaria fisicamente (em corpo) com a humanidade para sempre e pensando nisso Ele disse que não nos deixaria órfãos e pediria ao Pai e Ele nos enviaria o Espírito de verdade, que o mundo não conhece mas que nós conhecemos e que vive em nós.

\emdash{}Esse Espírito sopra as verdades de Deus para nós em nosso interior. Se silenciarmos poderemos ouví-lo. É importante conhecer a palavra de Deus para dar subsídios para nosso entendimento entender as intuições desse Santo Espírito. Como uma voz interior que guia a humanidade para tudo de bom que faz, como já comentado anteriormente: ``Quem olha para fora sonha, quem olha para dentro desperta!"

\emdash{}A Luz do mundo agora está dentro de cada um mas se não olharmos para ela e mergulharmos profundamente dentro de nós para isso, se ficarmos apenas na superfície e ignorando tesouros preciosos de nossa alma, a mudança edificante não ocorre. Buscar o auto-conhecimento é um passo importante: meditação, reflexão, espiritualidade saudável. E aí entram os conselhos do Evangelho de Jesus como o principal: Amar a Deus sobre todas as coisas e ao próximo como a si mesmo, pois se tivermos apenas religiosidade mas não cumprirmos esse maior mandamento, seremos apenas como um sino que retine como disse São Paulo.

\emdash{}Na meditação é possível raciocinar a fé: pensar, comparar, questionar. Se necessário buscar informações adicionais com pessoas de confiança, fazer perguntas, formar um conhecimento sólido para que sua fé não vacile nas lacunas da dúvida que podem enfraquecê-la nem caia no fanatismo por simplesmente acreditar sem entender.



\emdash{}Paz e bem. Deus abençoe a todos.