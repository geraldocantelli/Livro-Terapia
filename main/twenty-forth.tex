\begin{chapterpage}{Energia}{c24_twenty-forthchapter:cha}

\begin{myquotation}Quem tentar possuir uma flor, verá sua beleza murchando. Mas quem apenas olhar uma flor num campo, permanecerá para sempre com ela. Você nunca será minha e por isso terei você para sempre.

\par\vspace*{15mm}
\mbox{}\hfill \emdash{}Paulo Coelho \index{Coelho, Paulo}
, %\citetitle{bibitem}\index{@\citetitle{bibitem}} %\ifxetex\label{famousperson-bibitem-quote}\else\citep[p.~123]{bibitem}\fi
\par\end{myquotation}

\end{chapterpage}

% -------------------- replace or remove text below and paste your own text ------


\section{Semear, não reter para si}\label{c1_basicformatting:sec}

\emdash{}A Vida flui, não pode e não deve ser retida. Cada pedaço da vida é em si um milagre como a beleza de uma flor mas se for egoisticamente colhida, ela se desliga da Fonte e vem a murchar. Isso vale como lição para como nós nos relacionamos com  a nossa energia vital; Deus é a fonte da qual não podemos nos desligar para continuar recebendo portanto saibamos ``apreciar a santidade da vida" e compartilhá-la com todos nossos irmãos e irmãs no Amor; então saúde, paz, alegria fluirão, não sem lutas claro pois no nosso atual estado de evolução as provas são necessárias ao adiantamento e reparação de nossos ainda presentes traços egoístas. 

\emdash{}Nunca queiramos ser donos da Verdade nem donos de nada pois no momento em que separamos para nós, a coisa perde sua vitalidade, validade, nós distorcemos por exemplo. Pode-se buscar diretamente na fonte sempre que precisar pois nunca será negada Verdade, Vida, Amor a quem quer que bata na porta de Deus para as buscar. Mas essas graças alimentam nossa existência sem que a tomemos como nossa, nós apreciamos, vivemos essas graças e isso é tudo. Contentemo-nos com isso.

\emdash{}A Vida já dá tudo com Amor, pode-se reparar quantas graças recebemos: são infinitamente maiores do que os problemas. O Amor nos mantém o coração batendo, a respiração, as células funcionando; aprendemos a resolver problemas que antes pareciam insolúveis; várias coisas boas nos acontecem; Deus nos sorri o tempo todo. Mesmo os nossos entes queridos que partiram, cumpriram suas missões na Terra e estão no mundo espiritual vivos e vamos revê-los um dia; não há razão para desespero com relação a isso.

\emdash{}Qualquer tentativa de controlar a Vida falhará, primeiro porque ela é governada pela Sabedoria de Deus, que é infinitamente perfeito em tudo, e depois pois ela é grande demais para um intuito egoísta; ela pensa em todos e todos estamos unidos. Mais uma prova de que ``Deus é Amor" pois o amor pensa coletivo.

\emdash{}O Amor é a chave para a velha questão: como viver bem? É importante amar-se, muito importante, vital, essencial. A Vida vai nos ensinando a nos amar e precisamos aprender direitinho; o respeito por si, o cuidado, o trabalho, a família, quantas coisas importantes relacionadas conosco mesmo. Essa mesma sabedoria de Deus que governa a Vida nos leva a lições para valorizar tudo isso e muito mais; tudo é aprendizado.

\emdash{}E essas lições recebidas nesse permanente aprendizado vão ficando presentes em nosso ser de maneira que passamos a agir naturalmente com os novos comportamentos salutares aprendidos e a Vida vai melhorando para mais rápido para quem vive com vistas ao Amor. Como esse processo de evolução é independente da vontade das pessoas, inevitavelmente todos um dia estarão no Bem e juntos a Deus (leva quantas vidas levar, muitas) mas estar consciente desse processo e abrir-se às melhoras pode suavizar o caminho. Sem pressa.

