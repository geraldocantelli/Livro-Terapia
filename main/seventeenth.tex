\begin{chapterpage}{Iluminação}{c17_seventeenthchapter:cha}

\begin{myquotation}Aqueles que buscam a Iluminação devem sempre se lembrar da necessidade de manter constantemente puros o corpo, a fala e a mente. A mente impura segue atos impuros e estes trarão sofrimentos. Assim, é de suma importância que se conservem puros a mente e o corpo.
 
\par\vspace*{15mm}
\mbox{}\hfill \emdash{}Buda \index{Buda}
, %\citetitle{bibitem}\index{@\citetitle{bibitem}} %\ifxetex\label{famousperson-bibitem-quote}\else\citep[p.~123]{bibitem}\fi
\par\end{myquotation}

\end{chapterpage}

% -------------------- replace or remove text below and paste your own text ------


\section{Mitigar o erro original}\label{c1_basicformatting:sec}

\emdash{}O ser humano vive constantemente na Presença de Deus e tem parte com Ele mais do que imagina. Jesus disse: "vós sois deuses" por algum sério motivo e não em vão. Acredito que somos Um com o Criador e que em algum momento as pessoas que estão no nosso plano de existência se esqueceram disso e passaram a pensar que existem apenas na forma corporal e que são sozinhas e perecíveis no Universo, o que as leva ao egoísmo (raiz de todos os outros males).

\emdash{}Quando sente o medo da morte do corpo, e por se acreditar apenas matéria, pensa-se que tudo está perdido então se desespera, não acessando toda a riqueza que é sua natureza junto do Senhor da Vida que não lhe nega Vida em abundância que Jesus veio trazer, e detalhe: eterna.

\emdash{}Durante muitos anos foi mal empregado o nome de Deus e "pintado" um cenário espiritual triste e medonho do qual era preferível não acreditar em Deus (para conseguir "tocar a vida") a acreditar e deprimmir-se. E isso nada tem a ver com a pregação daquele Jesus da Galiléia que era só amor e levava esperança a todos que se chegavam a Ele.

\emdash{}Se cada vez que se pensa na Vida o coração da pessoa se enche de esperança então essa pessoa é bem-aventurada e sua bem-aventurança é a Fé. Fé constrói caráter por exemplo pois por acreditar na Justiça eterna que dá justa paga a todos e não deixa ninguém impune, nem os bons atos, a pessoa sente-se a vontade para ser boa com seu semelhante e daí nasce na comunidade uma convivência agradável e sadia.

\emdash{}Devemos manter puros a mente, a fala e o corpo para podermos enfrentar as dificuldades maior paz de espíritos pois estar no Caminho não significa ausência de empecilhos. O nosso maior e mais sério empecilho é pensarmos estar separados do Amoroso e Providente Criador e continuarmos agindo \textit{solo}. Pois apesar de não estarmos "fisicamente" separados, quando pensamos que estamos, passamos a estar "logicamente" separados, o que dá no mesmo na prática.

\emdash{}O caminho de volta à consciência da Presença pode ser mais longo do que gostaríamos de tanto que estamos acostumados com a solidão existencial. Mas Jesus é o Caminho, a Verdade e a Vida, nos agarremos a Ele para ficar tudo mais fácil, se necessário imaginemo-no ao nosso lado ou dentro de nós e meditemos suas palavras consoladoras.

\emdash{}Quando temos consciência da Presença de Deus em nosso Ser, nossa mente, corpo, palavras e atos são puros pois pensamos duas vezes antes de fazer algo desagradável ou ruim numa presença tão louvável. Pode ser este até um exercício para evoluir, parte do Caminho, pois nossa espiritualidade não significaria nada se não nos tornássemos melhores para os outros também além de melhores para nós mesmos.

\emdash{}Paz e bem.