\begin{chapterpage}{Causa e efeito - Carma}{c40_fortiethchapter:cha}
 
\begin{myquotation}Nossos atos tecem asas de libertação ou algemas de cativeiro, para nossa vitória ou nossa perda. 
\par\vspace*{15mm}
\mbox{}\hfill \emdash{}Chico Xavier\index{Xavier, Chico}
, %\citetitle{bibitem}\index{@\citetitle{bibitem}} %\ifxetex\label{famousperson-bibitem-quote}\else\citep[p.~123]{bibitem}\fi
\par\end{myquotation}

\end{chapterpage}

% -------------------- replace or remove text below and paste your own text ------


\section{Perguntas e Respostas}\label{c1_basicformatting:sec}

\emdash{}Aqui cabe uma explicação: Sabe-se que Emmanuel, na Bíblia é um dos nomes de Jesus, mas aqui também é o nome do espírito que é o mentor do médium Francisco Cândido Xavier. Ele respondeu a várias perguntas sobre para explicar o conceito de carma na prática e como se dá a aplicação da Justiça e da Misericórdia Divinas para a redenção de todos nós:


\emdash{}1) O que estrutura espiritualmente o corpo de carne?\\
\emdash{}Emmanuel: O corpo espiritual ou perispírito é o corpo básico, constituído de matéria sutil, sobre o qual se organiza o corpo de carne.\\

\emdash{}2) O erro de uma encarnação passada pode incluir na encarnação presente, predispondo o corpo físico às doenças? De que modo?\\
\emdash{}Emmanuel - A grande maioria das doenças tem a sua causa profunda na estrutura semi-material do corpo espiritual. Havendo o espírito agido erradamente, nesse ou naquele setor da experiência evolutiva, vinca o corpo espiritual com desequilíbrios ou distonias, que o predispõem à instalação de determinadas enfermidades, conforme o órgão atingido.

\emdash{}3) Quais os dois aspectos da Justiça?\\
\emdash{}Emmanuel - A Justiça na Terra pune simplesmente a crueldade manifesta, cujas conseqüências transitam nas áreas do interesse público, dilapidando a vida e induzindo à criminalidade; entretanto, esse é apenas o seu aspecto exterior, porque a Justiça é sempre manifestação constante da Lei Divina, nos processos da evolução e nas atividades da consciência.

\emdash{}4) Qual a relação existente entre doenças e a Justiça?\\
\emdash{}Emmanuel - No curso das enfermidades, é imperioso venhamos a examinar a Justiça, funcionando com todo o seu poder regenerativo, para sanar os males que acalentamos.

\emdash{}5) O que faz o Espírito, antes de reencarnar-se visando à própria melhoria?\\
\emdash{}Emmanuel - Antes da reencarnação, nós mesmos, em plenitude de responsabilidade, analisamos os pontos vulneráveis da própria alma, advogando em nosso próprio favor a concessão dos impedimentos físicos que, em tempo certo, nos imunizem, ante a possibilidade de reincidência nos erros em que estamos incursos.

\emdash{}6) Que pedem, para regenerar-se, os intelectuais que conspurcaram os tesouros da alma?\\
\emdash{}Emmanuel - Artífices do pensamento, que malversamos os patrimônios do espírito, rogam empeços cerebrais, que se façam por algum tempo alavancas coercitivas, contra as nossas tendências ao desequilíbrio intelectual.

\emdash{}7) Que medidas de reabilitação rogam os artistas que corromperam a inteligência?\\
\emdash{}Emmanuel - Artistas, que intoxicamos a sensibilidade alheia com os abusos da representação viciosa, imploramos moléstias ou mutilações, que nos incapacitem para a queda em novas culpas.

\emdash{}8) Que emendas solicitam os oradores e pessoas que influenciaram negativamente pela palavra?\\
\emdash{}Emmanuel - Tarefeiros da palavra, que nos prevalecemos dela para caluniar ou para ferir, solicitamos as deficiências dos aparelhos vocais e auditivos, que nos garantam a segregação providencial.

\emdash{}9) Que providências retificadoras pedem para si próprios aqueles que abraçaram graves compromissos do sexo?\\
\emdash{}Emmanuel - Criaturas dotadas de harmonia orgânica, que arremessamos os valores do sexo ao terreno das paixões aviltantes, enlouquecendo corações e fomentando tragédias, suplicamos as doenças e as inibições genésicas que em nos humilhando, servem por válvulas de contenção dos nossos impulsos inferiores.

\emdash{}10) Todas as enfermidades conhecidas foram solicitadas pelo Espírito do próprio enfermo, antes de renascer?\\
\emdash{}Emmanuel - Nem sempre o Espírito requisita deliberadamente determinadas enfermidades de vez que, em muitas circunstâncias quais aqueles que se verificam no suicídio ou na delinqüência, caímos, de imediato, na desagregação ou na insanidade das próprias forças, lesando o corpo espiritual, o que nos constrange a renascer no berço físico, exibindo defeitos e moléstias congênitas, em aflitivos quadros expiatórios.

\emdash{}11) Quais são os casos mais comuns de doenças compulsórias, impostas pela Lei Divina?\\
\emdash{}Emmanuel - Encontramos numerosos casos de doenças compulsórias, impostas pela Lei Divina, na maioria das criaturas que trazem as provações da idiotia ou da loucura, da cegueira ou da paralisia irreversíveis, ou ainda, nas crianças-problemas, cujos corpos, irremediavelmente frustrados, durante todo o curso da reencarnação, mostram-se na condição de celas regenerativas, para a internação compulsória daqueles que fizeram jus a semelhantes recursos drásticos da Lei. Justo acrescentar que todos esses companheiros, em transitórias, mas duras dificuldades, renascem na companhia daqueles mesmos amigos e familiares de outro tempo que, um dia, se cumpliciaram com eles na prática das ações reprováveis em que delinqüiram.

\emdash{}12) A mente invigilante pode instalar doenças no organismo? E o que pode provocar doenças de causas espirituais na vida diária?\\
\emdash{}Emmanuel - A mente é mais poderosa para instalar doenças e desarmonias do que todas as bactérias e vírus conhecidos. Necessário, pois, considerar igualmente, que desequilíbrios e moléstias surgem também da imprudência e do desmazelo, da revolta e da preguiça. Pessoas que se embriagam a ponto de arruinar a saúde; que esquecem a higiene até se tornarem presas de parasitas destruidores; que se encolerizam pelas menores razões, destrambelhando os próprios nervos; os que passam, todas as horas em redes e leitos, poltronas e janelas, sem coragem de vencer a ociosidade e o desânimo pela movimentação do trabalho, prejudicando a função dos órgãos do corpo físico, em razão da própria imobilidade, são criaturas que geram doenças para si mesmas, nas atitudes de hoje mesmo, sem qualquer ligação com causas anteriores de existências passadas.

\emdash{}13) Qual a advertência de Jesus para que nos previnamos dos males do corpo e da alma?\\
\emdash{}Emmanuel - Assinalando as causas distantes e próximas das doenças de agora, destacamos o motivo por que os ensinamentos da Doutrina Espírita nos fazem considerar, com mais senso de gravidade, a advertência do Mestre: “Orai e vigiai, para não cairdes em tentação”.

\emdash{}Chico Xavier - Emmanuel

\emdash{}(Do livro “Leis Do Amor”, Francisco Cândido Xavier E Waldo Vieira)\\
\emdash{}Fonte (livros digitados): Universo Espírita