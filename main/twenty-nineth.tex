\begin{chapterpage}{Defesa}{c29_twenty-ninethchapter:cha}

\begin{myquotation}Não se defenda com palavras. Se os seus atos não o defenderem, de nada adiantarão as palavras.

\par\vspace*{15mm}
\mbox{}\hfill \emdash{}Emmanuel, Chico Xavier \index{Chico Xavier, Emmanuel}
, %\citetitle{bibitem}\index{@\citetitle{bibitem}} %\ifxetex\label{famousperson-bibitem-quote}\else\citep[p.~123]{bibitem}\fi
\par\end{myquotation}

\end{chapterpage}

% -------------------- replace or remove text below and paste your own text ------


\section{O Conto dos Palhaços Equilibristas}\label{c1_basicformatting:sec}

\emdash{}Certa vez conta Divaldo Franco que estava com Francisco Candido Xavier à sombra de um abacateiro prestes a fazer a leitura do Evangelho para centenas de pessoas quando Chico lhe disse que Emmanuel, seu mentor, lhe pedira para contar a ele (Divaldo) uma estória: havia um circo em que trabalhavam dois palhaços, um mais velho e um mais novo, que se apresentavam andando sobre um arame num dos números. Um dia o palhaço mais novo colocou o rosto para fora das cortinas e viu a platéia e disse ao outro: ``você viu quanta gente veio nos ver?" e o palhaço mais velho respondeu: ``eles vieram ver qual de nós cairá primeiro!".

\emdash{}E o Chico perguntou: ``você entendeu, Divaldo?" e ele respondeu que sim e perfeitamente. Chico continuou: ``Muitas dessas pessoas estão querendo nos constranger nas nossas próprias palavras, usá-las contra nós. Projetam em nós seus próprios males e buscam os erros no discurso. Siga o caminho retamente para que seus atos o defendam pois suas palavras nunca serão suficientes e nunca convencerão o divergente do contrário".

\emdash{}A psicologia explica que o defeito que menos suportamos nos outros é aquele defeito que menos suportamos em nós mesmos e às vezes nem admitimos que temos por isso é mais fácil projetá-lo nos outros para censurá-lo neles. Para o Ego é muito difícil admitir que tem alguma imperfeição, digamos imperfeição a partir de agora e não defeito porque Deus não faz nada com defeito.

\emdash{}Pessoas que se dispõe a passar mensagens edificantes podem ser alvo de críticas ferrenhas mas saibam que os criticadores apenas estão criticando a projeção que fizeram e não o palestrante em si, por isso também perdoem e não deixe seu coração se perturbar por causa disso.