\begin{chapterpage}{Deus está no próximo}{c70_seventiethchapter:cha}
 
\begin{myquotation}Não devemos contentar-nos em falar do amor para com o próximo, mas praticá-lo.
\par\vspace*{15mm}
\mbox{}\hfill \emdash{}Albert Schweitzer\index{Schweitzer, Albert}
, %\citetitle{bibitem}\index{@\citetitle{bibitem}} %\ifxetex\label{famousperson-bibitem-quote}\else\citep[p.~123]{bibitem}\fi
\par\end{myquotation}

\end{chapterpage}

% -------------------- replace or remove text below and paste your own text ------


\section{Encontramos Deus no próximo}\label{c1_basicformatting:sec}

\emdash{}Muitos de nós procuram a Deus como necessidade existencial de completude, de sede de amor e para encontrar a paz. Sabemos que Deus está em todo lugar, inclusive dentro de nós mas às vezes temos dificuldade de encontrá-lo dentro de nós mesmos e sentimos como que uma necessidade de encontrá-lo em algo ou alguém que chega até nós mais diretamente do que um ato reflexivo.

\emdash{}Existe portais para acessar a Deus. Ele é excelso e transcendente portanto só podemos chegar até Ele através desses portais como o Amor ao próximo por exemplo. Se Deus está dentro de nós, então está dentro do próximo e talvez seja mais fácil vê-lo no outro do que vê-lo em nós mesmos. Isso acontece por causa de nosso estado evolutivo: não somos educados para nos valorizar e valorizar nossas verdades, isso é ensinado como se fosse arrogância e soberba mas não há nada de mais em cultivar suas verdades desde nós mesmos as passemos pelo crivo da razão e moral evangélicas.

\emdash{}Quando nos aproximamos de pessoas que nos falam de Deus, sentimos um bem estar na maioria das vezes mas nem sempre pois como nós pessoas não somos perfeitos, às vezes podemos falar coisas sobre o céu que não passariam na peneira do discernimento e então isso mais nos entristece do que felicita. Nem tudo o que se diz sobre Deus é verdadeiro mas Ele verdadeiramente é.

\emdash{}Cada um tem uma experiência com Deus e isso é sagrado em todos os sentidos. Buscar experimentá-lo é a única maneira de formarmos nosso discernimento a respeito das coisas santas. E experimentá-lo no próximo, no outro parece a maneira mais prática e acessível, afinal somos seres sociais. Na Bíblia está escrito: ``Provai e vede como Deus é bom", portanto no próximo temos uma perfeita oportunidade de reconhecermos a bondade e o amor do Pai.

\emdash{}Principalmente quando ajudamos o próximo. Há aquele cumprimento sagrado chamado ``Namastê" que significa: o Deus que está dentro de mim saúde o Deus que está dentro de você" e é verdade. O que há de melhor nas pessoas é revelado quando buscamos o céu e se todos o fizerem, então o Reino dos Céus estará implantado na Terra.