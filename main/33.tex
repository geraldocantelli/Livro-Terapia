\begin{chapterpage}{Igreja Livre}{c33_thirtieth-thirdchapter:cha}

\begin{myquotation} Mas a Jerusalém, que é de cima, é livre, a qual é mãe de todos nós.

\par\vspace*{15mm}
\mbox{}\hfill \emdash{}Paulo, Galatas 4:26 \index{Galatas 4:26, Paulo}
, %\citetitle{bibitem}\index{@\citetitle{bibitem}} %\ifxetex\label{famousperson-bibitem-quote}\else\citep[p.~123]{bibitem}\fi
\par\end{myquotation}

\end{chapterpage}

% -------------------- replace or remove text below and paste your own text ------


\section{Igreja Livre}\label{c1_basicformatting:sec}

\emdash{}O exame isolado deste versículo sugere um tema de infinita grandeza para os discípulos religiosos do Cristianismo.

\emdash{}A palavra do Apóstolo aos gentios recorda-nos a igreja liberta do Cristo, não na esfera estreita dos homens, mas no ilimitado pensamento divino.

\emdash{}O espírito orgulhoso e sectário, há tanto tempo dominante nas atividades da fé, encontra na afirmativa de Paulo de Tarso um antídoto para as suas venenosas preocupações.

\emdash{}Em todas as épocas, têm vivido na Terra os nobres excomungados, os incompreendidos valorosos e os caluniados sublimes.

\emdash{}Passaram, nos círculos das criaturas, qual acontece ainda hoje, perseguidos e desprezados, entre o sarcasmo e a indiferença.

\emdash{}Por vezes, sofrem o degredo social por não se aviltarem ante as explorações delituosas do fanatismo; em outras ocasiões, são categorizados à conta de ateus, pelas suas ideias mal interpretadas.

\emdash{}É que, de quando em quando, rajadas de ódios e dúvidas sopram nas igrejas desprevenidas da Terra. Os crentes olvidam o ``não julgueis" e confiam-se a lutas angustiosas.

\emdash{}Semelhantes atritos, contudo, não alteram a consciência tranquila dos anatematizados que se sentem sob a tutela do divino Poder. Instintivamente, reconhecem que além da esfera obscura da ação física resplandece o templo soberano e invisível em que Jesus recolhe os servidores fiéis, sem deter-se na cor ou no feitio de suas vestimentas.

\emdash{}Benfeitores e servos excomungados dos caminhos humanos, se tendes uma consciência sem mácula, não vos magoe a pedrada dos homens que se distanciam uns dos outros pelo separatismo infeliz. Há uma Igreja augusta e livre, na vida espiritual, que é acolhedora mãe de todos nós!...

\emdash{}\textit{Mensagem extraída do livro Vinha de Luz, autoria do espírito Emmanuel, psicografado pelo médium Francisco Cândido Xavier}.