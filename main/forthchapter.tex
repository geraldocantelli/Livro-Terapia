

\begin{chapterpage}{Poetry - Poems - Verses}{c4_forthchapter:cha}

\begin{myquotation}Aqueles que estão livres de pensamentos rancorosos certamente encontram a paz.\par\vspace*{15mm}
\mbox{}\hfill \emdash{}Buda\index{Buda}
% Add the source.
%, \citetitle{bibitem}\index{@\citetitle{bibitem}} \ifxetex\label{famousperson-bibitem-quote}\else\citep[p.~123]{bibitem}\fi
\par\end{myquotation}

\end{chapterpage}

% -------------------- replace or remove text below and paste your own text ------

\section{Poema das Dimensões}\label{c1_images:sec}

Deus é transcendente, todo-poderoso, multi-dimensional, bondoso, criador de tudo e todos. Nós vivemos nEle e para Ele. Nossos caminhos se cruzam de acordo com seus desígnios. Sabemos o que ele nos permite saber e somos quem nos permitimos que Ele nos ensine a ser. 

Ser e realidade são multidimensionais e o que queremos é a paz e a união, e o amor. Amar é existir, é viver em consonância com a vontade vivificadora do Pai. 

O sentido da vida pode ser encontrado dentro de nós, quando nos encontramos com Deus. A paz e a unidade no amor são filhas dEle e movem nosso espírito.  

Nada é coincidência, não existe sorte ou azar, só a realidade das muitas dimensões nossas, do universo; e da única e onipresente realidade de Deus (e amorosa). 

Quanto amor pode caber num ser humano é quanto amor Deus colocou dentro de todos nós. Uma quantia infinita pra caber em corações: a perfeição se revela em detalhes do bem e da verdade. 

Vida, luz, boa vontade e colaboração transbordam de Deus para sempre. Essa é a sintonia que edifica, que leva ao céu tão sonhado. Queremos, buscamos, somos levados para cima por Ele, dentro de cada um. 

Da nossa união como seres humanos nasce a mais linda poesia que é o perfume da vida, realizando no nosso mundo a Vida sonhada por Ele, seu Reino Eterno. 

Perdão, caminho, existência, boa vontade edificam arranha-céus que elevam o tempo presente à eternidade. 

Dons preciosos são o início deste caminho e seu conhecimento um tesouro incomensurável. Paciência, resiliência, mansidão, caridade: que pintura mística e linda da Vida. 

Um momento, agora, pra sempre. 

Promessas além dos muros do Eu, carinho. 

A vida revelada em desvelo. Mãe e pai, primos, irmãos e família do coração. Nós humanidade. 

O tempo é o tecido deste espetáculo e o espaço sua equação.  

O coração não cabe em três dimensões e tudo escrito nesta página muito menos... 

Por que as realidades caberiam? Graças a Deus que encontrou uma solução para todos os problemas antes que eles acontecessem. 

Ele criou o infinito muito além da nossa compreensão pra que tivéssemos a chance de contemplar sua beleza antes mesmo de entendê-lo, através da Sua vontade. 

Criação, boa criação, escolha do coração a florescer. 

E a perfumar os sentidos. 

\section{O Jardim}

Não existe flor mais bonita num jardim, nem jardim mais bonito de todos. A pluridade de cores e sensações esculpe e perfuma sem desbotar o resultado final. 

Dentro de cada um, uma individualidade e a presença de todos ao mesmo tempo. Singularidade é discernimento. 

Encontrar-se consigo só vale a pena se passa pelo outro. Só começamos a procurar quando sabemos que podemos encontrar e só saberemos que poderemos encontrar quando começarmos a procurar. 

Nenhuma busca é em vão, a questão é: procurar a beleza da vida é procurar a si? Somos obra prima do Criador e deveras O encontramos em nós quando escolhemos refletir Seu Espírito. 

A cada encontro: a partida de uma nova jornada e assim sucessivamente. Aí está a graça da Graça. 

Dizem ser a mudança a única certeza mas essa certeza também pode mudar. 

Não existem flores inadequadas, apenas únicas para quem sabe apreciar.  

Talvez saber apreciar a vida seja saber viver.  

A beleza está nos olhos de quem vê, não busque espelhos, busque ser espelho. 

Quem pode ousar dizer que reflete a Verdade com "V" maiúsculo? Um começo pode ser começar a refletir as verdades que admiramos...  

Ah.. se todos fazem isso não sei, preciso limpar minhas lentes até para aperceber-me disso. 

Espero que sim, aí seria só despertar para admirar. 

De fora pra dentro é imposição, de dentro pra fora é percepção. 