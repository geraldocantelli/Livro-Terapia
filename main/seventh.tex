\begin{chapterpage}{Reflexões de O Poder do Agora}{c5_fifhchapter:cha}

\begin{myquotation} Verdade está dentro de você. 
\par\vspace*{15mm}
\mbox{}\hfill \emdash{}Eckhart Tolle \index{Tolle, Eckhart}
, %\citetitle{bibitem}\index{@\citetitle{bibitem}} %\ifxetex\label{famousperson-bibitem-quote}\else\citep[p.~123]{bibitem}\fi
\par\end{myquotation}

\end{chapterpage}

% -------------------- replace or remove text below and paste your own text ------


\section{O Estado de Presença}\label{c1_basicformatting:sec}

\emdash{}Quando eu deixo o ego e me observo, observo o Agora, meus pensamentos, minhas reações, meus sentimentos, isso não sou o eu "ego" a minha mente, é o eu "divino" verdadeiro. 

No fundo, mas bem lá no fundo, nós sabemos que estamos destinados a sermos seres angelicais, a chegarmos à perfeição relativa que é possível chegar em comunhão com Deus e isso depois da ressurreição e não termos mais perispírito e voltarmos a ser apenas Espírito. 

Nosso eu verdadeiro vem de um lugar acima da mente, mais profundo, não dá pra entender com a mente nem é bom tentar entender mas vivenciar é libertador. Saber que Deus é amor puro e sabedoria infinita e está dentro de nós e nós somos Um com Ele é lindo. 

Observar a mente revela a dimensão do infinito enquanto que identificar-se com ela gera tempo (isso não é bom) e dá energia a ela. O tempo é um empecilho para que Deus se manifeste pois enquanto estamos na ilusão do passado e do futuro (sejam essas ilusões agradáveis ou não), não estamos presentes onde Deus pode nos ajudar, o Agora. 

Esse eu verdadeiro, acredito que é o Self que Jung falava. 


Então experimentamos um lugar de além da mente onde a Sabedoria e o Amor de Deus tocam profundo em nossa Alma, o Agora e quando isso acontece tiramos energia da mente e colocamos na Presença. 

É onde a sabedoria de Deus se manifesta e nossos problemas se dissipam e se resolvem e amamos mais as pessoas e as plantas e os animais e os minerais e as coisas em geral, somos mais compassivos, caridosos, amorosos, misericordiosos. 

Pensamento é da mente mas nesse lugar é mais profundo do que pensamento, é divino e podemos acessar a qualquer momento, basta focar no Agora e observar seus sentimentos e pensamentos e até como as outras pessoas te afetam sem julgar. Importante isso, apenas observar, sem julgamentos. 

Talvez a palavra sabedoria de Deus pareça um pouco pesada ou presunçosa mas é porque essa santa palavra Deus foi muito usada fora de contexto ao longo dos tempos e ficamos com uma ideia de um ser antropomórfico todo-poderoso quase inacessível mas como disse São João, Ele é Amor. 