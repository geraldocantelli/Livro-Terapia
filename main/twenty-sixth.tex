\begin{chapterpage}{Unidade}{c26_twenty-sixthchapter:cha}

\begin{myquotation}Observe as plantas e animais, aprenda com eles a aceitar aquilo que é e a se entregar ao Agora. Deixe que eles lhe ensinem o que é Ser, o que é integridade - estar em unidade, ser você mesmo, ser verdadeiro. Aprenda como viver e como morrer e como não fazer do viver e do morrer um problema.

\par\vspace*{15mm}
\mbox{}\hfill \emdash{}Eckhart Tolle \index{Tolle, Eckhart}
, %\citetitle{bibitem}\index{@\citetitle{bibitem}} %\ifxetex\label{famousperson-bibitem-quote}\else\citep[p.~123]{bibitem}\fi
\par\end{myquotation}

\end{chapterpage}

% -------------------- replace or remove text below and paste your own text ------


\section{Integridade}\label{c1_basicformatting:sec}

\emdash{}Na Bíblia está escrito: ``O temor ao Senhor é o princípio da Sabedoria". Sabedoria essa que é preciso para viver e que podemos começar a desenvolver buscando a Deus com sinceridade e suas coisas santas, sempre com discernimento. Mas pode ser que no começo não tenhamos tanto discernimento assim, isso também é parte do aprendizado, mas é preciso nos corrigir para que não firamos a consciência e a fé alheias.

\emdash{}Tudo na Vida é um aprendizado. Mesmo que buscar as coisas do Alto nos traga medo no começo mas isso também pode ser enfrentado e com o tempo aprendemos a lidar com esses assuntos também como se aprende tudo na Vida. Tem um conto budista que diz que um dragão enorme começou a aterrorizar uma vila e todos o seus moradores fugiram de medo, ficando dentro de suas casas e um menino decidiu enfrentar este dragão: a princípio tremelicando, com um escudo ele começou a andar em direção à fera e à medida que dava passos adiante, o tamanho do dragão diminuía até que ficou bem pequenininho quando ele chegou perto. Assim é quando enfrentamos nossos medos.

\emdash{}Viver não tem que ser um dilema. Por isso há a necessidade de entregarmos nosso caminho a Deus e confiarmos Nele e Ele o fará, no sentido de que receberemos intuições de soluções para nossos problemas, coragem para enfrentar as dificuldades, Sua Providência nos ajudará, enfim jamais seremos desamparados. Também está escrito: ``Provai e vede como o Senhor é bom", isso é muito valioso para lembrarmos por exemplo de todos os momentos em que Deus nos ajudou na Vida (foram incontáveis momentos - tantos quanto as estrelas do céu), somos perdoados, intuidos, ajudados, temos nossas necessidades providas, na medida de nossas provas e expiações em relação ao merecimento acumulado das vidas que vivemos, mas o importante é Agora, vivermos bem o Agora e isso pode ser obtido confiando-se a Deus.

\emdash{}Os problemas humanos podem ser resolvidos com Ele no comando, para nos livrarmos da insconsciência de ficarmos à deriva no mar da Vida e tomarmos as rédeas de nossas existências e isso é um processo em conjunto com Deus. Mas como fazer isso? Deixe que Ele te conduza, não há fórmula mágica, o Senhor conhece todas as coisas e Ele guia os filhos que se confiam a Ele pelos caminhos seguros da Vida. A boa notícia é essa: que o processo é entregar-se a Ele e que não se precisa saber mais nada a princípio para começar, é só deixar-se conduzir por Deus pois nós somos seres espirituais vivendo uma experiência carnal então sabemos sim nos deixar conduzir por Deus. Essa é uma função nobre do ser humano que precisa ser despertada e usada com mais frequência pelas pessoas.

\emdash{}Então com Deus no comando, passamos a ser os protagonistas de nossa própria Vida, conscientes de nosso papel na sociedade e atuantes. Carl Gustav Jung dizia da importância de isso acontecer e de se estar consciente para viver. Falava também de Espiritualidade num tempo em que esse termo ainda não tão era empregado como é hoje. Atualmente as pessoas estão buscando mais a saúde espiritual, a Organização Mundial da Saúde determinou que a saúde das pessoas é configurada nos níveis:

\begin{itemize}
    \item Física
    \item Psíquica
    \item Social
    \item Espiritual
\end{itemize}

\emdash{}``Aceitar aquilo que é", trecho da frase no início do capítulo significa estar consciente da realidade e vivê-la mesmo que seja necessário mudar alguma coisa podemos fazê-lo mas então façamos e não fiquemos somente com posturas negativas de não aceitação passivas que só trazem mal. Diante de uma situação que nos incomoda podemos escolher aceitá-la ou mudá-la mas nunca ficar em postura negativa, segundo o mesmo autor Eckhart Tolle. Mas essa frase tem um sentido mais profundo também: ``Torna-te o que tu és", somos filhos de Deus e não podemos viver capengando e sofrendo padecimentos que podem ser excluídos com mudança de atitude mental, claro que o caminho é longo e difícil para alcançar esse estado de melhora mas vale a pena e começa entregando-se a Deus e deixando Ele fazer o resto em nós.

\emdash{}Esse ``entregar-se a Deus e deixar Ele fazer o resto" significa que deveremos ser ativos, continuar trabalhando, produzindo, amando, perdoando, ajudando o próximo, fazendo caridade, portanto não é uma atitude passiva mas Ele está no comando, como sempre esteve. Quando um(a) paciente terminal se entrega a Deus, percebe que Deus é tudo que ele(a) sempre precisou a Vida inteira, que bom seria se as pessoas percebessem que essa entrega pode ser feita antes de um momento crítico, aliás, pode ser feita Agora mesmo.

\emdash{}Não precisa esperar-se a proximidade da morte para ter intimidade com Deus e a nossa qualidade de Vida melhora muito depois disso.  Como está escrito ``Aquele que está em Cristo, nova criatura é". Se nos entregarmos, tenhamos paciência na semeadura para poder colher os frutos e eles serão abundantes: 100 por 1, 1000 por 1, assim por diante.