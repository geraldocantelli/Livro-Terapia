\begin{chapterpage}{Castelo Interior}{c34_thirtieth-forthchapter:cha}

\begin{myquotation} Consideremos nossa alma como um castelo, feito de um só diamante ou de um cristal claríssimo, onde há muitos aposentos, à semelhança das muitas moradas que há no céu.

\par\vspace*{15mm}
\mbox{}\hfill \emdash{}Santa Teresa de Jesus\index{Teresa de Jesus, Santa}
, %\citetitle{bibitem}\index{@\citetitle{bibitem}} %\ifxetex\label{famousperson-bibitem-quote}\else\citep[p.~123]{bibitem}\fi
\par\end{myquotation}

\end{chapterpage}

% -------------------- replace or remove text below and paste your own text ------


\section{O Livro da Vida}\label{c1_basicformatting:sec}

\emdash{}Quando nascemos temos a oportunidade de escrever uma obra inteiramente nova que é nosso enredo particular de vida em que somos o personagem principal ou pelo menos deveríamos ser inclusive para poder dar o primeiro lugar em nossa Vida ao Senhor da Vida pois quem não é dono de si não consegue priorizar Deus em seus atos. Ninguém dá aquilo que não tem portanto quem não é dono de si mesmo não pode ofertar-se nem ao próximo nem ao Senhor.

\emdash{}Mas como alguém pode não estar em posse de si mesmo? Para explicar isso, cito a obra ``O Castelo Interior" de Santa Tereza de Jesus, uma monja que comparou a alma humana a uma habitação do tipo castelo com muitos aposentos mas que nem sempre são visitados e adequadamente conhecidos e valorizados. Começa ela dizendo que o portal para entrar nesse castelo é a oração mas como pode a pessoa precisar entrar nesse castelo se a pessoa já é a própria alma? É que às vezes vivemos tão absortos nas coisas do mundo e das muralhas desse castelo (que é o corpo) que não adentramos o castelo e vivemos às voltas com as feras que ficam em volta de seus limites.

\emdash{}Interessante como esse livro foi escrito entre 1515 e 1582, período de vida da autora, e já tratava de assuntos psicológicos que Jung reconheceria bem mais tarde como funções da mente humana. Sabe-se contudo que para Freud a religiosidade aflora em função de uma frustração sexual mas essa teoria é largamente contestada atualmente por muitos psiquiatras e cientistas que se seguiram a ele tanto que hoje em dia a psicologia não nega a religiosidade do paciente e também considera anti-ético interferir na religiosidade do mesmo, só o fazendo se esta o estiver causando mal.

\emdash{}Se a pessoa mal acessa seu castelo interior, se mal reflete sobre sua condição de ser humano, de pessoa, como poderia assenhorar-se da situação de sua vida? Como poderia contestar com argumentos adultos e sérios os seus adversários? É certo que por instinto de preservação muitas das vezes isso se realiza mas poderia obter-se melhor resultado se melhor planejado, pensado, arquitetado. Na Biologia como na natureza quem vence, quem sobrevive não é o mais forte mas sim quem melhor se adapta: essa é a lei da evolução de Darwin.

\emdash{}Todas as espécies atualmente presentes no planeta são igualmente evoluídas, dizia meu professor do mestrado em Bioinformática, pois todas passaram pelo que tiveram que passar com sucesso para que estivessem aqui agora, passaram pela seleção natural. Jesus nos ensina a sermos resignados nas dificuldades que se apresentam, isso nos torna mais fortes. Quanto mais humildes e resignados, mais fortes. Já quanto mais orgulhosos e desesperados, mais fracos. O orgulho é causa de queda de sociedades inteiras e de indivíduos também.

\emdash{}A vacina contra o orgulho é o Evangelho de Jesus, estudemos como Ele se comportava, sempre humilde, como ele dizia: ``quem se humilhar será exaltado e quem se exaltar será humilhado". O que nos leva à pergunta: quem é o Rei do seu castelo interior? você mesmo ou Deus? o Ego ou o Self? Você já adentrou seu castelo interior e está edificando belas páginas no livro da Vida? Você e Deus somente têm a coroa sobre a cabeça? E isso não quer dizer que humanamente (na hierarquia da profissão ou na família) não possa ser subordinado a outras pessoas, estou falando de você e Deus reinarem dentro de você, internamente, pois em sociedade temos que saber conviver com outras pessoas cordialmente e amigavelmente como disse o Mestre: ``Amar a Deus sobre todas as coisas e ao próximo como a si mesmo" (não se esqueça do como a si mesmo).