\begin{chapterpage}{Deus é a Vida}{c69_sixtieth-ninethchapter:cha}
 
\begin{myquotation}E esta é a promessa que Ele nos fez: a Vida Eterna.
\par\vspace*{15mm}
\mbox{}\hfill \emdash{}I João, 2:25\index{I João, 2:25}
, %\citetitle{bibitem}\index{@\citetitle{bibitem}} %\ifxetex\label{famousperson-bibitem-quote}\else\citep[p.~123]{bibitem}\fi
\par\end{myquotation}

\end{chapterpage}

% -------------------- replace or remove text below and paste your own text ------


\section{Deus é a própria Vida}\label{c1_basicformatting:sec}

\emdash{}Já repararam no milagre da Vida? E no mistério do surgimento da Vida (ainda não explicado). Como a Vida pode ter se criado a si mesma, em toda sua riqueza de detalhes e infinitude do Universo. Em como existem Leis naturais que a regem que são imutáveis e eternas. Em lindas realidades como o Amor, a Felicidade e a Paz que estão dentro de nós e independem, em sua pureza, de fatores externos ao ser humano, pois vêm de um lugar além do mundo.

\emdash{}Em cada uma de nossas células acontecem praticamente o tempo todo processos metabólicos e fisiológicos de altíssima complexidade e tudo isso ocorre sem nossa intervenção consciente. Existe uma Inteligência que orquestra esses processos celulares e do Universo, Inteligência essa que comandou o início de tudo, seja o Big Ben por exemplo que precisou de condições exatas para ocorrer, afinal de onde veio aquela massa de energia super concentrada que explodiu e criou o universo material?

\emdash{}Quando Jesus veio falar do Pai, ele falava do Senhor da Vida, que é a própria Vida em Si. Respeitar Deus é respeitar a Vida, a nossa, a do próximo, afinal todos somos Um. Não existe diferença entre a minha vida e a sua vida, todos estamos na Vida Única que é Deus. O Ego de cada um não gosta dessa afirmação pois é individualista mas Jesus veio falar da vida em comunidade e em perdoar setenta vezes sete. Na parábola do Bom Samaritano e do Semeador, deu exemplos lindos do que é o Reino da Vida, do Amor. 

\emdash{}Ter fé em Deus é ter fé na Vida. Por isso o suicídio é o maior de todos os crimes pois quando não se tem fé na Vida, é em Deus que não tem fé e vice-versa. Apesar dos aspectos particulares de cada religiosidade, no fundo é a fé na Vida que move todos que buscam a Deus. Buscam se completar, buscam ajuda, buscam Amor, que são os objetivos básicos da vida de todos.

\emdash{}Estar em Deus é estar na Vida. Mas às vezes temos uma ideia imperfeita de quem Ele é, imaginamos um ditador sádico e não é nada disso. Ele está em nós. Quando mostramos compaixão por exemplo, Ele está se manifestando através de nós. Deixemos para trás as ideias da Idade Média e vivamos em Paz com nossa consciência seguindo as leis da Vida, ou seja, os verdadeiros mandamentos de Deus.