\begin{chapterpage}{A parábola do filho pródigo}{c39_thirtieth-ninethchapter:cha}
 
\begin{myquotation}Cada criatura humana traz duas almas consigo: uma que olha de dentro para fora, outra que olha de fora para dentro... 
\par\vspace*{15mm}
\mbox{}\hfill \emdash{}Machado de Assis\index{Assis, Machado de}
, %\citetitle{bibitem}\index{@\citetitle{bibitem}} %\ifxetex\label{famousperson-bibitem-quote}\else\citep[p.~123]{bibitem}\fi
\par\end{myquotation}

\end{chapterpage}

% -------------------- replace or remove text below and paste your own text ------


\section{Interpretação Jungiana da parábola do filho pródigo}\label{c1_basicformatting:sec}

\emdash{}A mentora Joanna de Ângelis apresentou através da psicografia do médium Divaldo Franco uma interpretação, à luz da psicologia desenvolvida por Carl G. Jung, da parábola do filho pródigo trazida por Jesus que partilho aqui neste capítulo diante das explicações da psicóloga Íris Sinoti.

\emdash{}O filho pródigo simboliza aquele que fez o caminho de ir em busca de si mesmo, que buscou encontrar-se e perdeu-se no durante o caminho, mas que teve a coragem de realizar essa busca e portanto apesar de ter se perdido, bateu à porta de seu pai e encontrou novamente, não a sua casa mas a si mesmo. Seu pai é o Self, ou seja, o seu Eu Verdadeiro, o Eu Sou, a representação de Deus em si, o Cristo interior, que o recebeu de braços abertos quando ele reconheceu que andou errado, ou seja acolheu sua sombra.

\emdash{}A sombra é onde guardamos desde pequenos tudo o que é considerado inadequado, errado, inconveniente em nós, como se fosse uma sacola para onde vai tudo isso e depois fica guardado e não podemos mais tirar as coisas de lá nem acessar. Como não temos acesso, pra olhar para essas coisas, nós as vemos nos outros, são portanto os defeitos dos outros que reparamos que na verdade é nossa sombra ``projetada" neles. A psicologia diz que a dificuldade que menos suportamos em outra pessoa é na verdade a dificuldade que menos suportamos em nós mesmos.

\emdash{}Quando o filho pródigo está chegando perto do pai, com sua mensagem já formulada ```pai, pequei contra o céu e contra ti, já não sou digno de ser chamado de filho, trata-me como a um de seus empregados...", o pai já o recebeu de braços abertos e lhe deu uma túnica e um anel (símbolos de dignidade e de uma aliança). Quer dizer que quando buscamos com humildade nosso Self e aceitamos nossa sombra, nosso inconsciente se abre para nos ajudar plenamente no processo de melhora e integração chamado de individuação. Revelações podem vir em sonho ou podem ser captadas através de pessoas.

\emdash{}E o filho mais velho, que não fica nada feliz com o retorno do filho pródigo representa o comodismo, ou seja, ficar sem realizar a busca por si mesmo e correr portanto os riscos inerentes a esta atividade. E esse indivíduo tem muito mais dificuldade para integrar sua sombra e portanto fica apontando os erros dos outros (exatamente a atitude do filho mais velho). Já o filho pródigo que fez o seu caminho, perdeu-se mas encontrou-se e foi perdoado não tem porquê ficar apontando os erros dos outros.