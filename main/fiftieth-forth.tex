\begin{chapterpage}{Deuteronômio 18}{c54_fiftieth-forthchapter:cha}
 
\begin{myquotation}Entre ti não se achará quem faça passar pelo fogo a seu filho ou a sua filha, nem adivinhador, nem prognosticador, nem agoureiro, nem feiticeiro;
Nem encantador, nem quem consulte a um espírito adivinhador, nem mágico, nem quem consulte os mortos;
Pois todo aquele que faz tal coisa é abominação ao Senhor; e por estas abominações o Senhor teu Deus os lança fora de diante de ti.
\par\vspace*{15mm}
\mbox{}\hfill \emdash{}Deuteronômio 18:10-12\index{Deuteronômio 18:10-12}
, %\citetitle{bibitem}\index{@\citetitle{bibitem}} %\ifxetex\label{famousperson-bibitem-quote}\else\citep[p.~123]{bibitem}\fi
\par\end{myquotation}

\end{chapterpage}

% -------------------- replace or remove text below and paste your own text ------


\section{Explicação para o Deuteronômio 18}\label{c1_basicformatting:sec}

\emdash{}Essa lei foi colocada por Moisés durante o Antigo Testamento e, como todas as suas outras leis, era justificada no seu tempo. As pessoas invocavam os espíritos por motivos fúteis e até para prejudicar os outros, como é a macumba nos dias atuais. Havia muita promiscuidade nas comunicações entre o mundo físico e o mundo espiritual que desvirtuavam as pessoas e as afastavam da moral e portanto de Deus.

\emdash{}Mas essa lei de Moisés é como outras que previam pena de morte e que hoje não são mais empregadas pela religião, é como a lei que deixava o homem dar carta de divórcio a sua mulher e que Jesus disse que foi feita por causa da dureza do coração dos homens mas que no princípio não era assim... É como a lei que mandava apedrejar a mulher pega em adultério, que hoje se sabe que não se deve fazer, ou seja, algumas leis do tempo de Moisés como os 10 mandamentos são eternas pois foram enviadas diretamente por Deus e já outras tinham uso específico no seu tempo e seriam revistas com o adiantamento da sociedade e do conhecimento melhorado da pessoa de Deus.

\emdash{}Para os hebreus, Deus era implacável e punidor e hoje temos o conhecimento de um Deus amoroso e misericordioso. No final do século XIX, o  professor Hippolyte Léon Denizard Rivail começou estudos na sociedade parisiense com as famosas ``mesas girantes" que respondiam por códigos de batidas a perguntas dos presentes. Uma mesa não poderia ter inteligência para responder a perguntas então deveria haver um princípio inteligente movendo as mesas, começou uma investigação científica.

\emdash{}Das mesas passou-se a uma cesta com um lápis que escrevia e depois esses espíritos se comunicaram através de pessoas, os médiuns. A mediunidade é uma faculdade biológica ligada à glândula hipófise do cérebro, que algumas pessoas têm mais desenvolvida que outras, mas mesmo que seja através de intuições, todos temos essa ligação com o mundo espiritual.

\emdash{}Jesus disse que uma árvore boa não pode dar mau fruto e que uma árvore má não pode dar bom fruto. O Espiritismo do professor Rivail (Allan Kardec) estuda o Evangelho de Jesus e esclarece a Bíblia levando as pessoas a praticarem a caridade e o amor ao próximo e a Deus sobre todas as coisas; cura milhares de pessoas de possessões e outros males; evita suicídios ao provar que a vida não acaba no túmulo; enfim eleva a alma humana a ideiais mais elevados e santos.

\emdash{}Sem contar que se a pessoa tem uma mediunidade mais aflorade e não estuda a espiritualidade e não toma passes ou participa de um Centro Espírita, pode ficar muito prejudicada na vida, fora de si, usando um termo conhecido: louca. Todo mundo procura por paz e alegria e na nossa forma de evoluir a comunicação com o mundo dos espíritos pode ajudar muito inclusive para combater o materialismo e o ceticismo em relação ao próprio Deus.

\emdash{}As pessoas não procurariam essas comunicações se não fossem fortemente encorajadas a isso pelas dificuldades que passam por não as procurarem (no caso de quem tem mediunidade). É nossa forma de evoluir, às vezes pela dor mesmo. Mas há também aqueles que louvavelmente procuram a espiritualidade por amor.

\emdash{}Deixo aqui também uma mensagem a todos que anatematizam o Espiritismo: por favor deixem os Espíritas em paz pois eles estão apenas procurando seguir o caminho de Jesus da forma como podem e como entender ser o certo. Se toda a Lei e os profetas está resumida em ``Amar a Deus sobre todas as coisas e ao próximo como a si mesmo", por favor respeitem a fé alheia que, garanto, é muito sincera.