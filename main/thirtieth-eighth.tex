\begin{chapterpage}{Os 4 Gigantes da Alma}{c38_thirtieth-eighthchapter:cha}
 
\begin{myquotation}O melhor trabalho político, social e espiritual que podemos fazer é parar de projetar nossas sombras nos outros.
\par\vspace*{15mm}
\mbox{}\hfill \emdash{}Carl G. Jung\index{Jung, Carl G.}
, %\citetitle{bibitem}\index{@\citetitle{bibitem}} %\ifxetex\label{famousperson-bibitem-quote}\else\citep[p.~123]{bibitem}\fi
\par\end{myquotation}

\end{chapterpage}

% -------------------- replace or remove text below and paste your own text ------


\section{Os Gigantes da Alma Atuais}\label{c1_basicformatting:sec}

\emdash{}Emílio Myra e Lopes, nasceu em Santiago de Cuba, faleceu no Brasil em 1964, para onde veio em 1945, a convite da fundação Getúlio Vargas. Escreveu a obra ``Os Gigantes da Alma" em 1947 e então eles eram, segundo ele:
\begin{itemize}
    \item Amor
    \item Ódio
    \item Medo
    \item Ciúme
\end{itemize}

\emdash{}Para Joanna de Ângelis, em seu livro ``O Homem Integral", os quatro gigantes atuais são:
\begin{itemize}
    \item Rotina - asfixia/fuga-neuroses
    \item Ansiedade - sistema nervoso
    \item Medo - úlceras
    \item Solidão - solitários / não-solitários
\end{itemize}

\emdash{}Já em sua obra ``Em Busca da Verdade", Joanna de Ângelis" nos diz que três inimigos ferozes respondem por todas as misérias humanas:
\begin{itemize}
    \item Egoísmo - avaro (infeliz)
    \item Orgulho - cega
    \item Ignorância - escraviza
\end{itemize}

\emdash{}Ainda nesse livro constam as afirmações seguintes: Nascer, viver, nascer de novo - é a Lei, no entanto, é necessário descortinar algo mais profundo, que é a prática da caridade como processo de salvação, de autoiluminação. Entenda-se SALVAÇÃO como libertação da ignorância, do mal que existe no íntimo de cada pessoa, ausência da crueldade e do mal, (...).

\emdash{}Segundo a doutrina espírita, toda a humanidade foi criada simples e ignorante, ou seja, sem conhecer nada. Os indivíduos não eram originalmente bons nem maus e não sabiam tecnicamente de nada. Com as experiências das primeiras encarnações foram adquirindo tendências e fazendo, pelo livre-arbítrio, escolhas que reforçam ou não essas tendências ou criavam outras. Há seres que trilharam caminhos mais curtos e suaves de evolução e portanto escolheram o bem desde o princípio: são os que resistiram à tentação do pecado original, já outros caíram e precisam expiar as más escolhas de ações dos caminhos tortuosos que tomaram.

\emdash{}Nesse segundo grupo encontra-se grande parte das pessoas da Terra, que é um planeta da categoria de Provas e Expiações. Contudo nesse momento, ano de 2020, ela encontra-se (e já há algum tempo mas mais acentuadamente agora) em processo de Transição Planetária para planeta de Regeneração, uma nova categoria que pode ser entendida como um lugar onde os espíritos tomarão um fôlego das refregas das duras provas e expiações pelas quais passaram no período anterior.

\emdash{}Aqueles espíritos que insistirem no mal e estiverem endurecidos nesse caminho causariam sérios desequilíbrios nesse mundo de Regeneração e não poderão continuar na Terra e a exemplo do que aconteceu com os exilados de Capela há milênios atrás, haverá uma migração em massa de espíritos para um mundo primitivo. Capela é um mundo que estava em uma situação parecida com a da Terra agora e os espíritos que não reuniam condições de continuar nele foram enviados para a Terra quando esta estava no seu estado Primitivo (anterior ao de Provas e Expiações). O resultado foi que eles evoluíram os povos da Terra com sua intelectualidade mais avançada, esses povo daqui eram pois eram então mais atrasados do que eles. É a lei do Progresso agindo.

\emdash{}Precisamos pois parar de projetar nossas sombras nas outras pessoas pois segundo Joanna de Ângelis na obra Triunfo Pessoal, ``a sombra, significando o lado escuro da personalidade, pode ser analisada como herança dos atos ignóbeis (desprezíveis/baixos) ou infelizes que o Espírito gostaria de esquecer ou negar, mas que prosseguem em mecanismo de punição, dando lugar a conflitos e complexos perturbadores, expressando-se de forma densa". 

\emdash{}No seu livro Jesus e o Evangelho à Luz da Psicologia Profunda, ela diz ``O bem e o mal – essa dualidade luz e sombra – em que se debate o Espírito humano, representam o futuro e o passado de cada ser humano no trânsito evolutivo".E ``O ensinamento de Jesus fundamenta-se na evolução do Self, iluminando a sombra e vencendo-a". E o que seria o Self? Em seu livro Triunfo Pessoal, ela traz: ``O Self é o Espírito imortal, herdeiro de si mesmo,(...)".