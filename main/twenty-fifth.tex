\begin{chapterpage}{Evolução}{c25_twenty-fifthchapter:cha}

\begin{myquotation}Torne-se o sol e todos o avistarão. O dever do sol é existir, ser o que é! 

\par\vspace*{15mm}
\mbox{}\hfill \emdash{}Fiódor Dostoiévski \index{Dostoiévski, Fiódor}
, %\citetitle{bibitem}\index{@\citetitle{bibitem}} %\ifxetex\label{famousperson-bibitem-quote}\else\citep[p.~123]{bibitem}\fi
\par\end{myquotation}

\end{chapterpage}

% -------------------- replace or remove text below and paste your own text ------


\section{Efetividade da Educação}\label{c1_basicformatting:sec}

\emdash{}Durante a atual pandemia de coronavírus, aos idosos estão sendo fortemente recomendado ficar em casa pois estão no grupo de risco da doença. Uma publicação no Facebook chamou minha atenção, dizia: ``Se virem minha mãe andando pela rua me avisem pois eu vou buscar ela de chinelo na mão. Chegou a minha vez!" Alguém poderia observar que é uma brincadeira descabida pelo óbvio motivo de que os filhos não podem bater nos pais (e isso é claro!) mas eu gostaria de perguntar: e onde é que está escrito que os pais podem bater nos filhos? O ser humano é um animal para aprender por castigo físico? Aliás nem os animais deveriam sofrer castigos físicos.

\emdash{}Castigo é punição, não educação. É descontar nos filhos o descontentamento pelo comportamento deles ou que nós tínhamos na idade deles. Educar é ensinar para que aprenda com argumentos as consequências de cada ação e tornar a pessoa responsável ou seja aquele que responde pela própria vida.

\emdash{}Quando uma pessoa é presa e fica reclusa esse período serve, em teoria, para que ela se reeduque para que depois possa voltar à sociedade. Portanto essa reclusão não é punição, ou seja, não é uma simples ``desforra" pelo ato praticado mas sim medida corretiva. Não estou fazendo crítica à ação policial, que merece todo o reconhecimento da população, mas não é o fato de apanhar da polícia que faz o bandido se corrigir, isso, quando se corrige, é pela reeducação.

\emdash{}Quando os pais batem nos filhos, não estão nesse momento educando e sim ``desforrando" suas próprias frustrações e agindo como a polícia faz com os bandidos, mas seus filhos já são bandidos para precisar apanhar da polícia? Essas crianças já estão julgadas e condenadas e punidas pelos próprios pais, que às vezes usam palavras desgraçadas para, sem querer, lançar maldições durantes essas intempéries que acompanham os filhos depois.

\emdash{}Mas pessoas que não tiveram uma estrutura não tem muito para oferecer aos seus pequenos. Ninguém dá aquilo que não tem. Por isso, longe de passar a mão na cabeça de ninguém mas entregando à pessoa toda a responsabilidade pela sua própria vida é preciso conscientizá-la da realidade do mundo, da vida e dar condições para que ela possa interagir com esse mundo e essa vida de maneira que as consequências de seus atos sejam as melhores possíveis.

\emdash{}Formar-se-ia então uma reação em cascata em que pessoas melhor educadas e estruturadas multiplicariam a Educação. Como está na Bíblia: um pouco de fermento leveda a massa toda. E em muitos lugares esse fermento já age, o terreno a semear não é completamente virgem pois muito já é feito para que o Reino de Deus aconteça: ``o reino de Deus não consiste no comer e no beber, mas na justiça, na paz, e na alegria no Espírito Santo".

\emdash{}Entretanto, não podemos deixar esse trabalho apenas para o meio religioso ou para quando estamos no meio religioso. Conversar sobre filosofia e o sentido da vida e como levá-la bem, sobre lições importantes da vida que nossos pais tem para nos passar poderiam ser hábitos dentro de nossas casas para que houvesse um  preparo para a vida. Precisamos nos preparar para viver como ou melhor que alguém que se prepara para um tarefa importante. E não há receita pronta pois a vida é fluida e tem Sua própria Sabedoria. Buscar em Deus em primeiro lugar as respostas é válido mas também exige discernimento para não cairmos em armadilhas que existem nesse âmbito. 

\emdash{}Quando tomamos consciência que somos os únicos responsáveis pela nossa vida, preparar-se para ela passa a ser uma atitude desejável. Seria bom que todos soubessem um pouco de tudo e até que fossem possíveis saudáveis discussões sobre a vida em família (não brigas, mas diálogos com argumentos e questionamentos). Ousar fazer perguntas que não machuquem uns aos outros mas que levem a questionamentos que tragam respostas desejadas há muito tempo! Sair do torpor de simplesmente existir aparentemente sem sentido pois na verdade a Vida é Santa pois tem um sentido lindo, mas como encontrá-lo sem esse exercício?

\emdash{}Saiamos da zona de conforto em que nos habituamos a ficar e realizemos os Planos de Deus para nós. Cada um pode descobrir o seu Plano mas até para isso terá que pensar, refletir e partilhar. E pôr a mão na massa vai, durante o caminho, dar condições para cumprir a tarefa; pois o Caminho se faz caminhando, e nas refregas da vida somos mais que capacitados, somos curados.