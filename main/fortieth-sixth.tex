\begin{chapterpage}{Vida Interior}{c46_fortieth-sixthchapter:cha}
 
\begin{myquotation}A satisfação de uma vida plena não está na quantidade de anos que vivemos, mas na qualidade de vida interior que nutrimos no momento presente.
\par\vspace*{15mm}
\mbox{}\hfill \emdash{}Paulo Nascimento\index{Nascimento, Paulo}
, %\citetitle{bibitem}\index{@\citetitle{bibitem}} %\ifxetex\label{famousperson-bibitem-quote}\else\citep[p.~123]{bibitem}\fi
\par\end{myquotation}

\end{chapterpage}

% -------------------- replace or remove text below and paste your own text ------


\section{A Vida Interior é a Vida}\label{c1_basicformatting:sec}

\emdash{}Quantas coisas nos inquietam nesse mundo e quantas tiram nossa paz e atenção e reparem que a maioria delas é passageira e não pertence realmente a nós. Com isso quero dizer que não precisaríamos perder o que é mais precioso que é nossa qualidade de vida por praticamente coisas que passam, o que fica somos nós, nossa essência, nossa alma pois nem nosso corpo fica ou ficará, nem nossas posses, nem mesmo os acontecimentos que nos sucedem (sejam eles considerados bons ou maus).

\emdash{}Nós parecemos ter duas vidas: a interior e a exterior mas principalmente no Ocidente apenas enxergamos a exterior e coisas como posição social, orgulho, egoísmo passam a ditar nosso comportamento. Jesus falava da Vida interior por isso pregava o desprendimento e o amor ao próximo, e também a Deus sobre todas as coisas pois Ele pode ser encontrado dentro de nós. Daí a riqueza que se pode ter na vida interior: a riqueza que as traças não corroem e os ladrões não roubam: as riquezas do Reino dos céus.

\emdash{}Daí conclui-se que fazemos atos em nossa vida interior assim como fazemos atos em nossa vida exterior. Cada vez que adoramos a Deus, que somos misericordiosos e perdoamos nossos irmãos e irmãs e cada vez que fazemos caridade na nossa vida exterior deve ser porque houve um movimento equivalente na vida interior junto do Reino de Deus. Cada vez que buscamos uma meditação, um mindfullness, ou compreender melhor a Vida, a nós mesmos e colocar em prática o Amor, também.

\emdash{}Às vezes não entendemos como temos coragem de fazer alguma atitude em prol de nossa salvação, ou seja, em prol do que é acreditamos que Deus consideraria o certo a se fazer, mas isso se deve a um movimento na nossa vida interior, é um fruto desse movimento. Não digo necessariamente algo arriscado mas um passo de fé, um passo no escuro onde depois surge a luz, e ela sempre surge.

\emdash{}Pela lei do Progresso, lei Divina e imutável, continuamos seguindo nosso caminho de evolução. Um dia nossos atos da Vida Interior e da Vida Exterior serão ambos sempre a expressão do Evangelho de Jesus. Então o nosso encontro com Deus estará completo e O conheceremos tal como Ele É e viveremos plenamente a Unidade com Ele. Não estamos no momento separados dEle mas nem sempre nossos atos O espelham atualmente e é justamente isso que precisamos corrigir.

\emdash{}Com o auxílio do Evangelho de Jesus, do próprio Jesus e de outros tantos iluminados que guiam a humanidade para a comunhão com Deus, vamos nos voltar para nosso objetivo de espelhar o Criador na Criação, ou seja, praticar ações dignas de um Deus de infinita Bondade e Amor e continuar meditando para um dia entendermos o que mais há para entender das coisas que compartilhamos na nossa existência nesse vasto Universo, sem pressa nem ansiedade, pois até para com a Iluminação se deve buscá-la com paciência e prudência. Para ter alegria, aprecie as flores pelo caminho, sinta seu perfume, não foque somente no destino, pois o como se faz o caminho é tão importante ou mais do que o próprio destino.

\emdash{}Quando não souber o que fazer, faça uma prece (como disse Chico Xavier). Não precisamos fazer esse caminho sozinhos. Aliás não estamos sozinhos nunca. Enxergue que estamos com Deus o tempo todo, com Jesus e o Espírito Consolador. Os nossos atos que enxergamos com o ``consciente" são os atos da Vida Exterior e nela, às vezes estamos sozinhos, mas na Vida Interior, nunca estamos sozinhos pois conosco estão o Pai, o Filho e o Espírito Santo, que é aquele que Jesus prometeu que rogaria para Deus enviar e que nos lembraria tudo o que ele disse e nos ensinaria todas as demais coisas, onde esse Ensino acontece? Na vida Interior. Quando deixamos nossa Sabedoria Interior falar, só podemos acertar na vida, portanto ouçamos o Senhor que está dentro de nós.