\begin{chapterpage}{Graça}{c58_fiftieth-eighthchapter:cha}
 
\begin{myquotation}Eu asseguro: Quem ouve a minha palavra e crê naquele que me enviou tem a vida eterna e não será condenado, mas já passou da morte para a vida. 
\par\vspace*{15mm}
\mbox{}\hfill \emdash{}Jesus, João 5:24\index{João 5:24, Jesus}
, %\citetitle{bibitem}\index{@\citetitle{bibitem}} %\ifxetex\label{famousperson-bibitem-quote}\else\citep[p.~123]{bibitem}\fi
\par\end{myquotation}

\end{chapterpage}

% -------------------- replace or remove text below and paste your own text ------


\section{A Graça que vem de Deus}\label{c1_basicformatting:sec}

\emdash{}O Senhor Jesus deixa claro que ao aderirmos à Sua Pessoa estamos salvos, em outro trecho vai dizer: ``Eu lhes dou a vida eterna, e elas jamais perecerão; ninguém as poderá arrancar da minha mão" (João 10:28). Se ninguém pode nos tirar das mãos de Jesus, ninguém significa ninguém e que isso nunca acontecerá. Para provar isso também está escrito: ``Portanto, ele é capaz de salvar definitivamente aqueles que, por meio dele, se aproximam de Deus, pois vive sempre para interceder por eles" (Hebreus, 7:25).

\emdash{}A pessoa que se converte a Deus passa a ser Dele inteiramente e não se desviará do Caminho pois conhece a Verdade e tem a Vida em si. Jesus é o Caminho, a Verdade e a Vida por isso ``conhecereis a Verdade e ela vos libertará" ou seja conhecereis Jesus e ele vos libertará. Jesus morreu e ressuscitou para salvar a humanidade toda e sua missão foi cumprida com sucesso pois ele não falha. O que acontece é que alguns despertaram para a maravilha da Salvação enquanto muitos ainda dormitam nas sombras.

\emdash{}Se Jesus salvou a humanidade inteira, então todos chegarão ao Céu, ou seja, à comunhão completa com Deus mas sobre o tempo que levará para isso, uma vida apenas não é suficiente, daí a necessidade da tese da Reencarnação, que é era um dogma da Igreja Católica até meados do século V. Era ensinado nas missas que as pessoas teriam que voltar a viver na carne depois da morte para quitar seus débitos por más ações e evoluir. O próprio Jesus, logo após o acontecimento narrado como transfiguração diante de Pedro, Tiago e João, respondeu à pergunta deles: porque está escrito que Elias devia voltar? Ele respondeu na Verdade Elias já voltou e vocês não o reconheceram. Está escrito então que os apóstolos entenderam que Ele falava de João Batista. Em outro Evangelho sinótico Jesus diz que João Batista tem o espírito de Elias, ora, como se chama quando uma pessoa deixa o corpo, essa vida e vai para o mundo espiritual e depois nasce de novo (de Isabel) com outro nome e vida corpórea mas mantendo a identidade de espírito? Reencarnação.

\emdash{}Foi o que os discípulos entenderam e era assim que foi pregado na Igreja por 5 séculos até uma mulher de nome Teodora que era mulher do imperador de Roma ficar muito incomodada com o conceito de Reencarnação. Antes de ser mulher do imperador ela havia sido prostituta e as antigas amigas queriam lhe procurar no ``palácio" e ela tinha vergonha e asco por elas então mandou matar centenas de pessoas ligadas ao seu passado e depois ouvia na missa que voltaríamos a viver na carne para expiar nossas faltas e ela com centenas de homicídios nas costas, se incomodava. Como era filha de um importante bispo (o celibato na Igreja só veio depois de um milênio) ordenou o Concílio de Constantinopla e só uma ou duas dúzias de religiosos de Roma foram a esse concílio, os outros eram todos da parte Oriental. Esse evento foi feito com o objetivo de tirar o dogma da reencarnação e anatematizar quem o pregasse daí por diante. E assim foi feito. A narrativa desses fatos encontra-se na História.

\emdash{}Jesus salvou a todos, ou seja, abriu os caminhos para que pudéssemos chegar à perfeição relativa (porque perfeição absoluta só Deus). Por isso ele disse ``Sede perfeitos como vosso Pai Celestial é perfeito", nesse sentido. A Bíblia é muito clara quando diz que só quem é puro pode entrar no Reino dos Céus mas não é possível atingir essa pureza toda em uma só vida e ainda que fosse muitos morrem em má situação moral e para eles a salvação que Jesus obteve não teria valido? Como se diz então que Ele salvou toda a humanidade?

\emdash{}A obra de Jesus é completa e perfeita e atingirá seus efeitos em todas as pessoas. Umas se abrem mais cedo para o Amor e seu Caminho para a perfeição é mais asfaltado, menos longo, já para outros que insistem no mal, o caminho é mais acidentado, sofrido e longo. Está nas nossas mãos escolher fazer um caminho suave e feliz guiados pelo Senhor até esse Céu. E como Jesus disse, depois que estivermos em Suas mãos ninguém poderá nos tirar delas, e para estar em Suas mão desde já basta se entregar a Ele.