\begin{chapterpage}{Justiça}{c16_sixteenthchapter:cha}

\begin{myquotation}Por conseguinte, se alguém declara que a justiça significa restituir a cada um o que lhe é devido, e se por isso entende que o homem justo deve prejudicar os inimigos e ajudar os amigos, não é sábio quem expõe tais ideias. Pois a verdade é bem outra: não é lícito fazer o mal a ninguém em nenhuma ocasião.
 
\par\vspace*{15mm}
\mbox{}\hfill \emdash{}Sócrates, República \index{Sócrates, República}
, %\citetitle{bibitem}\index{@\citetitle{bibitem}} %\ifxetex\label{famousperson-bibitem-quote}\else\citep[p.~123]{bibitem}\fi
\par\end{myquotation}

\end{chapterpage}

% -------------------- replace or remove text below and paste your own text ------


\section{Desfazer-se do egoísmo}\label{c1_basicformatting:sec}

\emdash{}Como poderei ir para as delícias eternas do céu um dia, se este for o caso, sabendo que bilhões de irmãos meus (seres humanos) padecem inenarráveis tormentos eternamente num lugar horrível cheio de verdugos impiedosos armados até os dentes, lugar este supostamente criado pelo Deus de infinito Amor incondicional? Essa teoria não é nem aceitável, mas é a que é passada de geração em geração sem ser nem sequer questionada pelas pessoas.

\emdash{}Parece normal tomar o ilógico por lógico em nome da praticidade do dia a dia e dos costumes mas não se percebe que esse conceito só reforça o egoísmo da nossa sociedade pois se cada um só cuida do próprio céu sem se importar com o destino dos demais então cada um vive para si, na realidade, e não foi isso que Jesus pregou quando disse que toda a Lei e os profetas se resumem em amar a Deus sobre todas as coisas e ao próximo como a si mesmo.

\emdash{}Para os hebreus, portanto para o ensino religioso da época de Jesus, não havia o conceito de inferno. Para eles havia sim um fogo eterno em que os espíritos dos falecidos purgavam seus erros mas depois disso seguiam para a vida eterna e só o fogo era eterno e não as penas individuais de cada pessoa. Mas quando o texto foi traduzido para o latim e o grego esse fogo eterno foi interpretado como se a pena fosse eterna e a pessoa não saísse mais de lá para sempre. Foi um erro literal que levou a um erro conceitual que vai contra a lógica e o coração de Deus e tudo o que Jesus pregou.

\emdash{}E esse conceito errôneo foi incorporado pelo inconsciente coletivo levando as pessoas a duvidarem do amor de Deus ou até mesmo a não querem a acreditar nele pois se houvesse um Deus assim tão terrível melhor seria ser descrente a esperar um juízo tão severo no fim de uma vida já tão sofrida.

\emdash{}Jesus falava de amor, perdão, de reerguer-se todo aquele que se arrependesse de seus maus atos e consertar vidas e realidades, tornar retos os caminhos. Se a pessoa crê que pode ajustar-se, abre-se para a esperança e a mudança e sabe que pagará apenas pelo mal que houver feito como justa reparação (pois também não se apagam as consequências do que fazemos, simplesmente). Longe de promover mudança, a teoria do inferno promove descrença num Deus de amor que a acolhe e mantém a pessoa no erro.

\emdash{}Como sempre a solução é a educação. Vamos parar de assustar as pessoas com imagens terríveis e vamos falar do amor, do perdão, da possibilidade que têm de tornar retos seus caminhos nas veredas do Senhor a qualquer momento e que quanto mais se demoram, mais sofrerão as dores da reparação que lhes será imposta. Não existe crime sem castigo, cada um é responsável por tudo o que lhe acontece e atrai o próprio destino, então semeemos o bem desde já para colhermos o bem num futuro próximo.

\emdash{}Se as pessoas estivessem conscientes de que a vida não acaba com a morte e que, pela Lei Divina de Causa e Efeito colherão tudo o que plantarem, pensariam duas vezes antes de cometer maldades e maior seria o número daqueles que endireitam seus caminhos.


\emdash{}Que Deus nos ajude. Amém.