\begin{chapterpage}{Carta de São Paulo}{c65_sixtieth-fifthchapter:cha}
 
\begin{myquotation} Gravitar para a unidade divina, eis o fim da Humanidade. Para atingi-lo, três coisas são necessárias: a Justiça, o Amor e a Ciência. Três coisas lhe são opostas e contrárias: a ignorância, o ódio e a injustiça. 
\par\vspace*{15mm}
\mbox{}\hfill \emdash{}S. Paulo\index{S. Paulo}
, %\citetitle{bibitem}\index{@\citetitle{bibitem}} %\ifxetex\label{famousperson-bibitem-quote}\else\citep[p.~123]{bibitem}\fi
\par\end{myquotation}

\end{chapterpage}

% -------------------- replace or remove text below and paste your own text ------


\section{Carta Póstuma de São Paulo}\label{c1_basicformatting:sec}

\emdash{}Gravitar para a unidade divina, eis o fim da Humanidade. Para atingi-lo, três coisas são necessárias: a Justiça,
o Amor e a Ciência. Três coisas lhe são opostas e contrárias: a ignorância, o ódio e a injustiça. Pois bem! digo-vos,
em verdade, que mentis a estes princípios fundamentais,
comprometendo a idéia de Deus, com o lhe exagerardes a
severidade. Duplamente a comprometeis, deixando que no
Espírito da criatura penetre a suposição de que há nela
mais clemência, mais virtude, amor e verdadeira justiça,
do que atribuís ao ser infinito. Destruís mesmo a idéia do
inferno, tornando-o ridículo e inadmissível às vossas crenças, como o é aos vossos corações o horrendo espetáculo
das execuções, das fogueiras e das torturas da Idade Média! Pois que! Quando banida se acha para sempre das legislações humanas a era das cegas represálias, é que
esperais mantê-la no ideal? Oh! crede-me, crede-me, irmãos
em Deus e em Jesus-Cristo, crede-me: ou vos resignais a
deixar que pereçam nas vossas mãos todos os vossos
dogmas, de preferência a que se modifiquem, ou, então,
vivificai-os, abrindo-os aos benfazejos eflúvios que os Bons,
neste momento, derramam neles. A idéia do inferno, com
as suas fornalhas ardentes, com as suas caldeiras a ferver,
pôde ser tolerada, isto é, perdoável num século de ferro;
porém, no século dezenove, não passa de vão fantasma,
próprio, quando muito, para amedrontar criancinhas e em que estas, crescendo um pouco, logo deixam de crer. Se
persistirdes nessa mitologia aterradora, engendrareis a incredulidade, mãe de toda a desorganização social. Tremo,
entrevendo toda uma ordem social abalada e a ruir sobre
os seus fundamentos, por falta de sanção penal. Homens
de fé ardente e viva, vanguardeiros do dia da luz, mãos à
obra, não para manter fábulas que envelheceram e se desa-
creditaram, mas para reavivar, revivificar a verdadeira sanção penal, sob formas condizentes com os vossos costu-
mes, os vossos sentimentos e as luzes da vossa época.

\emdash{}Quem é, com efeito, o culpado? É aquele que, por um
desvio, por um falso movimento da alma, se afasta do objetivo da criação, que consiste no culto harmonioso do belo,
do bem, idealizados pelo arquétipo humano, pelo Homem-Deus, por Jesus-Cristo.

\emdash{}Que é o castigo? A conseqüência natural, derivada
desse falso movimento; uma certa soma de dores necessária a desgostá-lo da sua deformidade, pela experimentação
do sofrimento. O castigo é o aguilhão que estimula a alma,
pela amargura, a se dobrar sobre si mesma e a buscar o
porto de salvação. O castigo só tem por fim a reabilitação, a
redenção. Querê-lo eterno, por uma falta não eterna, é
negar-lhe toda a razão de ser.

\emdash{}Oh! em verdade vos digo, cessai, cessai de pôr em paralelo, na sua eternidade, o Bem, essência do Criador, com
o Mal, essência da criatura. Fora criar uma penalidade
injustificável. Afirmai, ao contrário, o abrandamento gradual dos castigos e das penas pelas transmigrações e
consagrareis a unidade divina, tendo unidos o sentimento
e a razão.

\emdash{}Paulo, apóstolo.

\emdash{}Mensagem retirada de O Livro dos Espíritos, capítulo ``Das penas e gozos futuros".