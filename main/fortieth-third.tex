\begin{chapterpage}{Meditação}{c43_fortieth-thirdchapter:cha}
 
\begin{myquotation}Somos o que pensamos. Tudo o que somos surge com nossos pensamentos. Com nossos pensamentos, fazemos o nosso mundo.
\par\vspace*{15mm}
\mbox{}\hfill \emdash{}Buda\index{Buda}
, %\citetitle{bibitem}\index{@\citetitle{bibitem}} %\ifxetex\label{famousperson-bibitem-quote}\else\citep[p.~123]{bibitem}\fi
\par\end{myquotation}

\end{chapterpage}

% -------------------- replace or remove text below and paste your own text ------


\section{A importância da Meditação}\label{c1_basicformatting:sec}

\emdash{}Certa vez em uma palestra a Monja Coen disse que meditar não é apenas ficar numa determinada posição tentando não pensar em nada mas que sim se pensar por exemplo sobre o que que é o sentimento, o que é a vida, como se sente diante de determinadas situações e por quê?

\emdash{}Passamos a vida ignorando questões fundamentais a nós mesmos por simplesmente por não termos o hábito de formulá-las a nós mesmos em momentos como a meditação. E nos faria um bem enorme se tivéssemos a atitude de voltarmos para o nosso interior, que é o lugar onde estão as nossas respostas, e buscássemos a essência de nosso ser.

\emdash{}Hoje em dia parece muito comum as pessoas dizerem que estão na ``correria" e muitas apresentam quadros de ansiedade e sentimento de vazio na alma ou um descontentamento com suas vidas no sentido geral, sentem que não estão plenamente satisfeitas com seu desempenho pessoal na vida. Isso pode ser falta de um encontro pessoal consigo mesmo.

\emdash{}Há uma sabedoria interior em todo ser humano, uma riqueza inestimável que pode ser acessada. Sem saber disso é como se fôssemos mendigos sofrendo privações mas com um diamante no bolso e não soubéssemos disso e nunca olhássemos dentro desse bolso. Pessoas como Jesus e Buda vieram para abrir os olhos da humanidade e fazer com que acessássemos essa sabedoria e como que despertássemos e fôssemos salvos de nossa própria ignorância.

\emdash{}Longe de perder tempo, estaremos ganhando tempo. Na verdade, a questão do tempo é bem interessante: tudo o que temos é o Agora. O passado não existe mais e o futuro só existirá quando for o Agora. A mente não sabe trabalhar bem em outro tempo que não seja o Agora por isso nos afligimos quando ficamos projetando possíveis realidades futuras e imaginando cenários, geralmente negativos, que poderão nos ocorrer. Ou ainda pode nos ocorrer de nos castigarmos com cenas do passado igualmente dolorosas; ambos comportamentos roubam o presente e o bem estar e a Vida que é tudo o que temos.

\emdash{}Meditar é uma forma de estar no Presente  pois não é possível meditar no passado nem no futuro. Quando meditar foque no Agora. Coloque, se preferir, uma música instrumental suave de fundo ou preste atenção ao som ambiente, preste atenção a sua respiração, inspire, expire. Pode ser por 5 minutos no começo (deixe o timer do celular regulado para 5 minutos) uma vez por dia.

\emdash{}Isso tudo é independente de religião, lembrando que religiosidade é diferente de espiritualidade pois uma pessoa pode ter muita religiosidade e pouca espiritualidade, se apenas segue os preceitos mas não se deixa tocar moralmente e ensinamentos espirituais da religião em questão. 

\emdash{}Meditar não é colocar-se em xeque e sim dar-se oportunidades de sua sabedoria interior falar o que você mesmo precisa ouvir. É uma forma de prover o encontro do Pai bondoso e compassivo com o filho pródigo da parábola e esses dois personagens somos nós mesmos. E somos o filho mais velho também que fica apontando as falhas e não aceita, e o Pai o chama pra dentro para integrá-lo também para sermos completos, integrais. Assim nenhuma parte nossa fica excluída, na sombra.